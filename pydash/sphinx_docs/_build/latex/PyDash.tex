%% Generated by Sphinx.
\def\sphinxdocclass{report}
\documentclass[letterpaper,10pt,english]{sphinxmanual}
\ifdefined\pdfpxdimen
   \let\sphinxpxdimen\pdfpxdimen\else\newdimen\sphinxpxdimen
\fi \sphinxpxdimen=.75bp\relax

\PassOptionsToPackage{warn}{textcomp}
\usepackage[utf8]{inputenc}
\ifdefined\DeclareUnicodeCharacter
 \ifdefined\DeclareUnicodeCharacterAsOptional
  \DeclareUnicodeCharacter{"00A0}{\nobreakspace}
  \DeclareUnicodeCharacter{"2500}{\sphinxunichar{2500}}
  \DeclareUnicodeCharacter{"2502}{\sphinxunichar{2502}}
  \DeclareUnicodeCharacter{"2514}{\sphinxunichar{2514}}
  \DeclareUnicodeCharacter{"251C}{\sphinxunichar{251C}}
  \DeclareUnicodeCharacter{"2572}{\textbackslash}
 \else
  \DeclareUnicodeCharacter{00A0}{\nobreakspace}
  \DeclareUnicodeCharacter{2500}{\sphinxunichar{2500}}
  \DeclareUnicodeCharacter{2502}{\sphinxunichar{2502}}
  \DeclareUnicodeCharacter{2514}{\sphinxunichar{2514}}
  \DeclareUnicodeCharacter{251C}{\sphinxunichar{251C}}
  \DeclareUnicodeCharacter{2572}{\textbackslash}
 \fi
\fi
\usepackage{cmap}
\usepackage[T1]{fontenc}
\usepackage{amsmath,amssymb,amstext}
\usepackage{babel}
\usepackage{times}
\usepackage[Bjarne]{fncychap}
\usepackage{sphinx}

\usepackage{geometry}

% Include hyperref last.
\usepackage{hyperref}
% Fix anchor placement for figures with captions.
\usepackage{hypcap}% it must be loaded after hyperref.
% Set up styles of URL: it should be placed after hyperref.
\urlstyle{same}
\addto\captionsenglish{\renewcommand{\contentsname}{Contents:}}

\addto\captionsenglish{\renewcommand{\figurename}{Fig.}}
\addto\captionsenglish{\renewcommand{\tablename}{Table}}
\addto\captionsenglish{\renewcommand{\literalblockname}{Listing}}

\addto\captionsenglish{\renewcommand{\literalblockcontinuedname}{continued from previous page}}
\addto\captionsenglish{\renewcommand{\literalblockcontinuesname}{continues on next page}}

\addto\extrasenglish{\def\pageautorefname{page}}

\setcounter{tocdepth}{1}



\title{PyDash Documentation}
\date{Jun 24, 2018}
\release{0.4.0}
\author{The PyDash Team}
\newcommand{\sphinxlogo}{\vbox{}}
\renewcommand{\releasename}{Release}
\makeindex

\begin{document}

\maketitle
\sphinxtableofcontents
\phantomsection\label{\detokenize{index::doc}}



\chapter{pydash}
\label{\detokenize{modules::doc}}\label{\detokenize{modules:pydash}}\label{\detokenize{modules:welcome-to-pydash-s-documentation}}

\section{flask\_monitoring\_dashboard\_client package}
\label{\detokenize{flask_monitoring_dashboard_client::doc}}\label{\detokenize{flask_monitoring_dashboard_client:module-flask_monitoring_dashboard_client}}\label{\detokenize{flask_monitoring_dashboard_client:flask-monitoring-dashboard-client-package}}\index{flask\_monitoring\_dashboard\_client (module)}
Performs the remote requests to the flask-monitoring-dashboard.

The method names in this module 1:1 reflect the names of the flask-monitoring-dashboard API
(but without the word ‘JSON’ in them, because conversion from JSON to Python dictionaries/lists
is one of the thing this module handles for you.)
\index{get\_data() (in module flask\_monitoring\_dashboard\_client)}

\begin{fulllineitems}
\phantomsection\label{\detokenize{flask_monitoring_dashboard_client:flask_monitoring_dashboard_client.get_data}}\pysiglinewithargsret{\sphinxcode{\sphinxupquote{flask\_monitoring\_dashboard\_client.}}\sphinxbfcode{\sphinxupquote{get\_data}}}{\emph{dashboard\_url}, \emph{dashboard\_token}, \emph{time\_from=None}, \emph{time\_to=None}, \emph{timeout=1}}{}
Get data from a deployed flask-monitoring-dashboard
:param dashboard\_url: The base URL for the deployed dashboard, without trailing slash
:param dashboard\_token: The secret token for the dashboard, used to decode the Json Web Token response
:param time\_from: An optional datetime indicating only data since that moment should be included
:param time\_to: An optional datetime indicating only data up to that point should be included;
only valid if time\_from is also specified
:param timeout: Optional timeout to wait for a response from the dashboard
:return: A dict containing all monitoring data, possibly limited to the given time range

\end{fulllineitems}

\index{get\_details() (in module flask\_monitoring\_dashboard\_client)}

\begin{fulllineitems}
\phantomsection\label{\detokenize{flask_monitoring_dashboard_client:flask_monitoring_dashboard_client.get_details}}\pysiglinewithargsret{\sphinxcode{\sphinxupquote{flask\_monitoring\_dashboard\_client.}}\sphinxbfcode{\sphinxupquote{get\_details}}}{\emph{dashboard\_url}, \emph{timeout=1}}{}
Get details from a deployed flask-monitoring-dashboard
:param dashboard\_url: The base URL for the deployed dashboard, without trailing slash
:param timeout: Optional timeout to wait for a response from the dashboard
:return: A dict containing details from the dashboard, or None if the request was unsuccessful

\end{fulllineitems}

\index{get\_monitor\_rules() (in module flask\_monitoring\_dashboard\_client)}

\begin{fulllineitems}
\phantomsection\label{\detokenize{flask_monitoring_dashboard_client:flask_monitoring_dashboard_client.get_monitor_rules}}\pysiglinewithargsret{\sphinxcode{\sphinxupquote{flask\_monitoring\_dashboard\_client.}}\sphinxbfcode{\sphinxupquote{get\_monitor\_rules}}}{\emph{dashboard\_url}, \emph{dashboard\_token}, \emph{timeout=1}}{}
Get monitor rules from a deployed flask-monitoring-dashboard
:param dashboard\_url: The base URL for the deployed dashboard, without trailing slash
:param dashboard\_token: The secret token for the dashboard, used to decode the Json Web Token response
:param timeout: Optional timeout to wait for a response from the dashboard
:return: A dict containing monitor rules of the dashboard, or None if the request was unsuccessful

\end{fulllineitems}



\section{periodic\_tasks package}
\label{\detokenize{periodic_tasks::doc}}\label{\detokenize{periodic_tasks:module-periodic_tasks}}\label{\detokenize{periodic_tasks:periodic-tasks-package}}\index{periodic\_tasks (module)}
Allows for the running of tasks in the background, as well as periodically.
Tasks can either be added to the \sphinxtitleref{default\_task\_scheduler}, or multiple schedulers can be created.

Tasks are run in a process pool of subprocesses (See \sphinxtitleref{multiprocessing.Pool}).
The task scheduler itself, which passes tasks on to this process pool, runs its scheduling loop in a separate subprocess as well.
This means that there is no computational overhead for the main process at runtime.

Internally, an indexable priority queue (c.f. the \sphinxtitleref{pqdict} package) is used to keep track of the next tasks to run.
This makes the scheduling loop quite efficient, because tasks are already ordered (so only the oldest task’s desired execution moment needs to be compared to the current timestamp).
Because the priority queue is indexed, adding and removing a task is also done in \sphinxtitleref{O(log(n))}.

Adding/updating/removing tasks is possible by using the same name as used previously for the task.
Names can be strings, but also any other hashable object, so referring to a task based on a tuple of strings + integers is also possible.

Tasks can be added/updated/removed at any time, including before the scheduler is started.

The scheduler will be started by calling the \sphinxtitleref{start()} function. It will stop scheduling and tear down the spawned processes when calling the \sphinxtitleref{stop()} function.
This function will also (in most cases) be automatically called when the main process finishes execution.

Example code with default scheduler:

\fvset{hllines={, ,}}%
\begin{sphinxVerbatim}[commandchars=\\\{\}]
\PYG{g+gp}{\PYGZgt{}\PYGZgt{}\PYGZgt{} }\PYG{k+kn}{import} \PYG{n+nn}{periodic\PYGZus{}tasks} \PYG{k}{as} \PYG{n+nn}{pt}
\PYG{g+gp}{\PYGZgt{}\PYGZgt{}\PYGZgt{} }\PYG{k+kn}{import} \PYG{n+nn}{datetime}
\PYG{g+gp}{\PYGZgt{}\PYGZgt{}\PYGZgt{} }\PYG{n}{pt}\PYG{o}{.}\PYG{n}{start\PYGZus{}default\PYGZus{}scheduler}\PYG{p}{(}\PYG{p}{)}
\PYG{g+gp}{\PYGZgt{}\PYGZgt{}\PYGZgt{} }\PYG{n}{pt}\PYG{o}{.}\PYG{n}{add\PYGZus{}periodic\PYGZus{}task}\PYG{p}{(}\PYG{l+s+s1}{\PYGZsq{}}\PYG{l+s+s1}{foo}\PYG{l+s+s1}{\PYGZsq{}}\PYG{p}{,} \PYG{n}{datetime}\PYG{o}{.}\PYG{n}{timedelta}\PYG{p}{(}\PYG{n}{seconds}\PYG{o}{=}\PYG{l+m+mi}{3}\PYG{p}{)}\PYG{p}{,} \PYG{n}{pt}\PYG{o}{.}\PYG{n}{foo}\PYG{p}{)}
\PYG{g+gp}{\PYGZgt{}\PYGZgt{}\PYGZgt{} }\PYG{n}{pt}\PYG{o}{.}\PYG{n}{add\PYGZus{}periodic\PYGZus{}task}\PYG{p}{(}\PYG{l+s+s1}{\PYGZsq{}}\PYG{l+s+s1}{bar}\PYG{l+s+s1}{\PYGZsq{}}\PYG{p}{,} \PYG{n}{datetime}\PYG{o}{.}\PYG{n}{timedelta}\PYG{p}{(}\PYG{n}{seconds}\PYG{o}{=}\PYG{l+m+mi}{5}\PYG{p}{)}\PYG{p}{,} \PYG{n}{pt}\PYG{o}{.}\PYG{n}{bar}\PYG{p}{)}
\PYG{g+gp}{\PYGZgt{}\PYGZgt{}\PYGZgt{} }\PYG{n}{pt}\PYG{o}{.}\PYG{n}{add\PYGZus{}background\PYGZus{}task}\PYG{p}{(}\PYG{l+s+s1}{\PYGZsq{}}\PYG{l+s+s1}{baz}\PYG{l+s+s1}{\PYGZsq{}}\PYG{p}{,} \PYG{n}{pt}\PYG{o}{.}\PYG{n}{baz}\PYG{p}{)}
\PYG{g+gp}{\PYGZgt{}\PYGZgt{}\PYGZgt{} }\PYG{n}{pt}\PYG{o}{.}\PYG{n}{add\PYGZus{}periodic\PYGZus{}task}\PYG{p}{(}\PYG{l+s+s1}{\PYGZsq{}}\PYG{l+s+s1}{bar}\PYG{l+s+s1}{\PYGZsq{}}\PYG{p}{,} \PYG{n}{datetime}\PYG{o}{.}\PYG{n}{timedelta}\PYG{p}{(}\PYG{n}{seconds}\PYG{o}{=}\PYG{l+m+mi}{1}\PYG{p}{)}\PYG{p}{,} \PYG{n}{pt}\PYG{o}{.}\PYG{n}{bar}\PYG{p}{)} \PYG{c+c1}{\PYGZsh{} overrides previous {}`bar{}` task with new settings}
\PYG{g+gp}{\PYGZgt{}\PYGZgt{}\PYGZgt{} }\PYG{n}{pt}\PYG{o}{.}\PYG{n}{remove\PYGZus{}task}\PYG{p}{(}\PYG{l+s+s1}{\PYGZsq{}}\PYG{l+s+s1}{foo}\PYG{l+s+s1}{\PYGZsq{}}\PYG{p}{)}
\PYG{g+gp}{\PYGZgt{}\PYGZgt{}\PYGZgt{} }\PYG{n}{pt}\PYG{o}{.}\PYG{n}{default\PYGZus{}task\PYGZus{}scheduler}\PYG{o}{.}\PYG{n}{stop}\PYG{p}{(}\PYG{p}{)}
\end{sphinxVerbatim}

Example code with custom scheduler:

\fvset{hllines={, ,}}%
\begin{sphinxVerbatim}[commandchars=\\\{\}]
\PYG{g+gp}{\PYGZgt{}\PYGZgt{}\PYGZgt{} }\PYG{k+kn}{import} \PYG{n+nn}{periodic\PYGZus{}tasks} \PYG{k}{as} \PYG{n+nn}{pt}
\PYG{g+gp}{\PYGZgt{}\PYGZgt{}\PYGZgt{} }\PYG{n}{ts} \PYG{o}{=} \PYG{n}{pt}\PYG{o}{.}\PYG{n}{TaskScheduler}\PYG{p}{(}\PYG{p}{)}
\PYG{g+gp}{\PYGZgt{}\PYGZgt{}\PYGZgt{} }\PYG{k+kn}{import} \PYG{n+nn}{datetime}\PYG{o}{,} \PYG{n+nn}{time}
\PYG{g+gp}{\PYGZgt{}\PYGZgt{}\PYGZgt{} }\PYG{n}{ts}\PYG{o}{.}\PYG{n}{start}\PYG{p}{(}\PYG{p}{)}
\PYG{g+gp}{\PYGZgt{}\PYGZgt{}\PYGZgt{} }\PYG{n}{ts}\PYG{o}{.}\PYG{n}{add\PYGZus{}periodic\PYGZus{}task}\PYG{p}{(}\PYG{l+s+s1}{\PYGZsq{}}\PYG{l+s+s1}{foo}\PYG{l+s+s1}{\PYGZsq{}}\PYG{p}{,} \PYG{n}{datetime}\PYG{o}{.}\PYG{n}{timedelta}\PYG{p}{(}\PYG{n}{milliseconds}\PYG{o}{=}\PYG{l+m+mi}{1}\PYG{p}{)}\PYG{p}{,} \PYG{n}{pt}\PYG{o}{.}\PYG{n}{foo}\PYG{p}{)}
\PYG{g+gp}{\PYGZgt{}\PYGZgt{}\PYGZgt{} }\PYG{n}{ts}\PYG{o}{.}\PYG{n}{add\PYGZus{}periodic\PYGZus{}task}\PYG{p}{(}\PYG{l+s+s1}{\PYGZsq{}}\PYG{l+s+s1}{bar}\PYG{l+s+s1}{\PYGZsq{}}\PYG{p}{,} \PYG{n}{datetime}\PYG{o}{.}\PYG{n}{timedelta}\PYG{p}{(}\PYG{n}{milliseconds}\PYG{o}{=}\PYG{l+m+mi}{5}\PYG{p}{)}\PYG{p}{,} \PYG{n}{pt}\PYG{o}{.}\PYG{n}{bar}\PYG{p}{)}
\PYG{g+gp}{\PYGZgt{}\PYGZgt{}\PYGZgt{} }\PYG{n}{time}\PYG{o}{.}\PYG{n}{sleep}\PYG{p}{(}\PYG{l+m+mi}{2}\PYG{p}{)}
\PYG{g+gp}{\PYGZgt{}\PYGZgt{}\PYGZgt{} }\PYG{n}{ts}\PYG{o}{.}\PYG{n}{stop}\PYG{p}{(}\PYG{p}{)}
\end{sphinxVerbatim}
\index{add\_background\_task() (in module periodic\_tasks)}

\begin{fulllineitems}
\phantomsection\label{\detokenize{periodic_tasks:periodic_tasks.add_background_task}}\pysiglinewithargsret{\sphinxcode{\sphinxupquote{periodic\_tasks.}}\sphinxbfcode{\sphinxupquote{add\_background\_task}}}{\emph{name}, \emph{task}, \emph{scheduler=\textless{}periodic\_tasks.task\_scheduler.TaskScheduler object\textgreater{}}}{}
Adds a task to be run only once (and as soon as possible) to the given \sphinxtitleref{scheduler}, which defaults to the global \sphinxtitleref{default\_task\_scheduler} that this module provides.
\begin{quote}\begin{description}
\item[{Name}] \leavevmode
An identifier to find this task again later (and e.g. remove or alter it). Can be any hashable (using a string or a tuple of strings/integers is common.)

\end{description}\end{quote}

(Calling this function again with the same name will override the earlier task).
:target: A function (or other callable) that will perform this task’s functionality.
:scheduler: Which TaskScheduler to run the task on. It defaults to the global \sphinxtitleref{default\_task\_scheduler} that this module provides.

\end{fulllineitems}

\index{add\_periodic\_task() (in module periodic\_tasks)}

\begin{fulllineitems}
\phantomsection\label{\detokenize{periodic_tasks:periodic_tasks.add_periodic_task}}\pysiglinewithargsret{\sphinxcode{\sphinxupquote{periodic\_tasks.}}\sphinxbfcode{\sphinxupquote{add\_periodic\_task}}}{\emph{name}, \emph{interval}, \emph{task}, \emph{run\_at\_start=False}, \emph{scheduler=\textless{}periodic\_tasks.task\_scheduler.TaskScheduler object\textgreater{}}}{}
Adds a task to be run periodically to the given \sphinxtitleref{scheduler}, which defaults to the global \sphinxtitleref{default\_task\_scheduler} that this module provides.
\begin{quote}\begin{description}
\item[{Name}] \leavevmode
An identifier to find this task again later (and e.g. remove or alter it). Can be any hashable (using a string or a tuple of strings/integers is common.)

\end{description}\end{quote}

(Calling this function again with the same name will override the earlier task).
:target: A function (or other callable) that will perform this task’s functionality.
:interval: A datetime.timedelta representing how frequently to run the given target.
:run\_at\_start: If true, runs task right after it was added to the scheduler, rather than only after the first interval has passed.
:scheduler: Which TaskScheduler to run the task on. It defaults to the global \sphinxtitleref{default\_task\_scheduler} that this module provides.

\end{fulllineitems}

\index{bar() (in module periodic\_tasks)}

\begin{fulllineitems}
\phantomsection\label{\detokenize{periodic_tasks:periodic_tasks.bar}}\pysiglinewithargsret{\sphinxcode{\sphinxupquote{periodic\_tasks.}}\sphinxbfcode{\sphinxupquote{bar}}}{}{}
\end{fulllineitems}

\index{baz() (in module periodic\_tasks)}

\begin{fulllineitems}
\phantomsection\label{\detokenize{periodic_tasks:periodic_tasks.baz}}\pysiglinewithargsret{\sphinxcode{\sphinxupquote{periodic\_tasks.}}\sphinxbfcode{\sphinxupquote{baz}}}{}{}
\end{fulllineitems}

\index{foo() (in module periodic\_tasks)}

\begin{fulllineitems}
\phantomsection\label{\detokenize{periodic_tasks:periodic_tasks.foo}}\pysiglinewithargsret{\sphinxcode{\sphinxupquote{periodic\_tasks.}}\sphinxbfcode{\sphinxupquote{foo}}}{}{}
\end{fulllineitems}

\index{periodic\_task() (in module periodic\_tasks)}

\begin{fulllineitems}
\phantomsection\label{\detokenize{periodic_tasks:periodic_tasks.periodic_task}}\pysiglinewithargsret{\sphinxcode{\sphinxupquote{periodic\_tasks.}}\sphinxbfcode{\sphinxupquote{periodic\_task}}}{\emph{name}, \emph{interval}, \emph{run\_at\_start=False}, \emph{scheduler=\textless{}periodic\_tasks.task\_scheduler.TaskScheduler object\textgreater{}}}{}
Function decorator to specify that the following function
should be called periodically;
It accepts the same arguments as \sphinxtitleref{add\_periodic\_task} (with the \sphinxtitleref{target} argument filled in by the function being decorated.)

Usage:
\begin{quote}

@periodic\_task(‘qux’, datetime.timedelta(seconds=2))
def qux():
\begin{quote}

print(‘qux’)
\end{quote}

@periodic\_task(‘qux’, datetime.timedelta(seconds=2), run\_at\_start=True, scheduler = your\_scheduler)
def qux():
\begin{quote}

print(‘qux’)
\end{quote}
\end{quote}

\end{fulllineitems}

\index{qux() (in module periodic\_tasks)}

\begin{fulllineitems}
\phantomsection\label{\detokenize{periodic_tasks:periodic_tasks.qux}}\pysiglinewithargsret{\sphinxcode{\sphinxupquote{periodic\_tasks.}}\sphinxbfcode{\sphinxupquote{qux}}}{}{}
\end{fulllineitems}

\index{remove\_task() (in module periodic\_tasks)}

\begin{fulllineitems}
\phantomsection\label{\detokenize{periodic_tasks:periodic_tasks.remove_task}}\pysiglinewithargsret{\sphinxcode{\sphinxupquote{periodic\_tasks.}}\sphinxbfcode{\sphinxupquote{remove\_task}}}{\emph{name}, \emph{scheduler=\textless{}periodic\_tasks.task\_scheduler.TaskScheduler object\textgreater{}}}{}
Removes a task that was previously added from the given \sphinxtitleref{scheduler}, which defaults to the global \sphinxtitleref{default\_task\_scheduler} that this module provides..
Will do nothing if there is no task with the given name.
\begin{quote}\begin{description}
\item[{Name}] \leavevmode
The task with this name will be removed.

\item[{Scheduler}] \leavevmode
Which TaskScheduler to remove the task from. It defaults to the global \sphinxtitleref{default\_task\_scheduler} that this module provides.

\end{description}\end{quote}

\end{fulllineitems}

\index{start\_default\_scheduler() (in module periodic\_tasks)}

\begin{fulllineitems}
\phantomsection\label{\detokenize{periodic_tasks:periodic_tasks.start_default_scheduler}}\pysiglinewithargsret{\sphinxcode{\sphinxupquote{periodic\_tasks.}}\sphinxbfcode{\sphinxupquote{start\_default\_scheduler}}}{}{}
Starts the default (global) scheduler that this module provides.

\end{fulllineitems}



\subsection{Submodules}
\label{\detokenize{periodic_tasks:submodules}}

\subsubsection{periodic\_tasks.pqdict\_iter\_upto\_priority module}
\label{\detokenize{periodic_tasks.pqdict_iter_upto_priority::doc}}\label{\detokenize{periodic_tasks.pqdict_iter_upto_priority:periodic-tasks-pqdict-iter-upto-priority-module}}\label{\detokenize{periodic_tasks.pqdict_iter_upto_priority:module-periodic_tasks.pqdict_iter_upto_priority}}\index{periodic\_tasks.pqdict\_iter\_upto\_priority (module)}\index{pqdict\_iter\_upto\_priority (class in periodic\_tasks.pqdict\_iter\_upto\_priority)}

\begin{fulllineitems}
\phantomsection\label{\detokenize{periodic_tasks.pqdict_iter_upto_priority:periodic_tasks.pqdict_iter_upto_priority.pqdict_iter_upto_priority}}\pysiglinewithargsret{\sphinxbfcode{\sphinxupquote{class }}\sphinxcode{\sphinxupquote{periodic\_tasks.pqdict\_iter\_upto\_priority.}}\sphinxbfcode{\sphinxupquote{pqdict\_iter\_upto\_priority}}}{\emph{pqueue}, \emph{priority}}{}
Bases: \sphinxhref{https://docs.python.org/3/library/functions.html\#object}{\sphinxcode{\sphinxupquote{object}}}

Wrapper around \sphinxtitleref{pqdict} to implement an iterator
that returns items up to the given \sphinxtitleref{priority} (exclusive).
The rest of the pqdict is kept unchanged.
\begin{quote}\begin{description}
\item[{Pqueue}] \leavevmode
An instance of the \sphinxtitleref{pqdict.pqdict} class.

\item[{Priority}] \leavevmode
The threshold priority.

\end{description}\end{quote}

The comparison function that the pqueue itself uses is used to cutoff this iterator,
so it will automatically work with both min-queues as wel as max-queues.

\end{fulllineitems}



\subsubsection{periodic\_tasks.queue\_nonblocking\_iter module}
\label{\detokenize{periodic_tasks.queue_nonblocking_iter::doc}}\label{\detokenize{periodic_tasks.queue_nonblocking_iter:periodic-tasks-queue-nonblocking-iter-module}}\label{\detokenize{periodic_tasks.queue_nonblocking_iter:module-periodic_tasks.queue_nonblocking_iter}}\index{periodic\_tasks.queue\_nonblocking\_iter (module)}\index{queue\_nonblocking\_iter (class in periodic\_tasks.queue\_nonblocking\_iter)}

\begin{fulllineitems}
\phantomsection\label{\detokenize{periodic_tasks.queue_nonblocking_iter:periodic_tasks.queue_nonblocking_iter.queue_nonblocking_iter}}\pysiglinewithargsret{\sphinxbfcode{\sphinxupquote{class }}\sphinxcode{\sphinxupquote{periodic\_tasks.queue\_nonblocking\_iter.}}\sphinxbfcode{\sphinxupquote{queue\_nonblocking\_iter}}}{\emph{queue}}{}
Bases: \sphinxhref{https://docs.python.org/3/library/functions.html\#object}{\sphinxcode{\sphinxupquote{object}}}

This iterator wraps the queue.Queue/multiprocessing.Queue objects,
which provide both a blocking API and a non-blocking API
that raises errors when attempting to retrieve an item while it is empty.

Since these queues exist on multiple threads/processes,
checking for (non)emptyness before attempting an action is not good enough,
because its state might change in-between.

So instead, we handle the \sphinxtitleref{queue.Empty} that is raised
when attempting to retrieve the next item from an emtpy queue.

\end{fulllineitems}



\subsubsection{periodic\_tasks.task\_scheduler module}
\label{\detokenize{periodic_tasks.task_scheduler::doc}}\label{\detokenize{periodic_tasks.task_scheduler:periodic-tasks-task-scheduler-module}}\label{\detokenize{periodic_tasks.task_scheduler:module-periodic_tasks.task_scheduler}}\index{periodic\_tasks.task\_scheduler (module)}
Contains the meat of the task scheduling: The TaskScheduler class,
and a couple of classes that it uses under the hood.
\index{TaskScheduler (class in periodic\_tasks.task\_scheduler)}

\begin{fulllineitems}
\phantomsection\label{\detokenize{periodic_tasks.task_scheduler:periodic_tasks.task_scheduler.TaskScheduler}}\pysiglinewithargsret{\sphinxbfcode{\sphinxupquote{class }}\sphinxcode{\sphinxupquote{periodic\_tasks.task\_scheduler.}}\sphinxbfcode{\sphinxupquote{TaskScheduler}}}{\emph{granularity=0.1}, \emph{pool\_settings=\{\}}}{}
Bases: \sphinxhref{https://docs.python.org/3/library/functions.html\#object}{\sphinxcode{\sphinxupquote{object}}}

Runs tasks in a process pool of subprocesses (See \sphinxtitleref{multiprocessing.Pool}).
The task scheduler itself, which passes tasks on to this process pool, runs its scheduling loop in a separate subprocess as well.
This means that there is no computational overhead for the main process at runtime.

Internally, an indexable priority queue (c.f. the \sphinxtitleref{pqdict} package) is used to keep track of the next tasks to run.
This makes the scheduling loop quite efficient, because tasks are already ordered (so only the oldest task’s desired execution moment needs to be compared to the current timestamp).
Because the priority queue is indexed, adding and removing a task is also done in \sphinxtitleref{O(log(n))}.

Adding/updating/removing tasks is possible by using the same name as used previously for the task.
Names can be strings, but also any other hashable object, so referring to a task based on a tuple of strings + integers is also possible.

Tasks can be added/updated/removed at any time, including before the scheduler is started.

The scheduler will be started by calling the \sphinxtitleref{start()} function. It will stop scheduling and tear down the spawned processes when calling the \sphinxtitleref{stop()} function.
This function will also (in most cases) be automatically called when the main process finishes execution.
\index{add\_background\_task() (periodic\_tasks.task\_scheduler.TaskScheduler method)}

\begin{fulllineitems}
\phantomsection\label{\detokenize{periodic_tasks.task_scheduler:periodic_tasks.task_scheduler.TaskScheduler.add_background_task}}\pysiglinewithargsret{\sphinxbfcode{\sphinxupquote{add\_background\_task}}}{\emph{name}, \emph{task}}{}
Adds a task to be run only once (and as soon as possible) to the scheduler.
\begin{quote}\begin{description}
\item[{Name}] \leavevmode
An identifier to find this task again later (and e.g. remove or alter it). Can be any hashable (using a string or a tuple of strings/integers is common.)

\end{description}\end{quote}

(Calling this function again with the same name will override the earlier task).
:target: A function (or other callable) that will perform this task’s functionality.

\end{fulllineitems}

\index{add\_periodic\_task() (periodic\_tasks.task\_scheduler.TaskScheduler method)}

\begin{fulllineitems}
\phantomsection\label{\detokenize{periodic_tasks.task_scheduler:periodic_tasks.task_scheduler.TaskScheduler.add_periodic_task}}\pysiglinewithargsret{\sphinxbfcode{\sphinxupquote{add\_periodic\_task}}}{\emph{name}, \emph{interval}, \emph{task}, \emph{run\_at\_start=False}}{}
Adds a task to be run periodically to the scheduler.
\begin{quote}\begin{description}
\item[{Name}] \leavevmode
An identifier to find this task again later (and e.g. remove or alter it). Can be any hashable (using a string or a tuple of strings/integers is common.)

\end{description}\end{quote}

(Calling this function again with the same name will override the earlier task).
:target: A function (or other callable) that will perform this task’s functionality.
:interval: A datetime.timedelta representing how frequently to run the given target.
:run\_at\_start: If true, runs task right after it was added to the scheduler, rather than only after the first interval has passed.

\end{fulllineitems}

\index{remove\_task() (periodic\_tasks.task\_scheduler.TaskScheduler method)}

\begin{fulllineitems}
\phantomsection\label{\detokenize{periodic_tasks.task_scheduler:periodic_tasks.task_scheduler.TaskScheduler.remove_task}}\pysiglinewithargsret{\sphinxbfcode{\sphinxupquote{remove\_task}}}{\emph{name}}{}
Removes a task that was previously added from the scheduler.
Will do nothing if there is no task with the given name.
\begin{quote}\begin{description}
\item[{Name}] \leavevmode
The task with this name will be removed.

\end{description}\end{quote}

\end{fulllineitems}

\index{start() (periodic\_tasks.task\_scheduler.TaskScheduler method)}

\begin{fulllineitems}
\phantomsection\label{\detokenize{periodic_tasks.task_scheduler:periodic_tasks.task_scheduler.TaskScheduler.start}}\pysiglinewithargsret{\sphinxbfcode{\sphinxupquote{start}}}{}{}
Starts the scheduler scheduling loop on a separate process.

Should only be called once per scheduler.

\fvset{hllines={, ,}}%
\begin{sphinxVerbatim}[commandchars=\\\{\}]
\PYG{g+gp}{\PYGZgt{}\PYGZgt{}\PYGZgt{} }\PYG{k+kn}{import} \PYG{n+nn}{periodic\PYGZus{}tasks} \PYG{k}{as} \PYG{n+nn}{pt}
\PYG{g+gp}{\PYGZgt{}\PYGZgt{}\PYGZgt{} }\PYG{n}{ts} \PYG{o}{=} \PYG{n}{pt}\PYG{o}{.}\PYG{n}{TaskScheduler}\PYG{p}{(}\PYG{p}{)}
\PYG{g+gp}{\PYGZgt{}\PYGZgt{}\PYGZgt{} }\PYG{n}{ts}\PYG{o}{.}\PYG{n}{start}\PYG{p}{(}\PYG{p}{)}
\PYG{g+gp}{\PYGZgt{}\PYGZgt{}\PYGZgt{} }\PYG{n}{ts}\PYG{o}{.}\PYG{n}{start}\PYG{p}{(}\PYG{p}{)}
\PYG{g+gt}{Traceback (most recent call last):}
   \PYG{c}{...}
\PYG{g+gr}{Exception}
\end{sphinxVerbatim}

\end{fulllineitems}

\index{stop() (periodic\_tasks.task\_scheduler.TaskScheduler method)}

\begin{fulllineitems}
\phantomsection\label{\detokenize{periodic_tasks.task_scheduler:periodic_tasks.task_scheduler.TaskScheduler.stop}}\pysiglinewithargsret{\sphinxbfcode{\sphinxupquote{stop}}}{}{}
Stops the scheduler scheduling loop.

Should only be called once per scheduler, and only after \sphinxtitleref{start()} was called.
When the program exits suddenly, this function will (in most cases) automatically be called
to clean up the scheduling process.

\fvset{hllines={, ,}}%
\begin{sphinxVerbatim}[commandchars=\\\{\}]
\PYG{g+gp}{\PYGZgt{}\PYGZgt{}\PYGZgt{} }\PYG{k+kn}{import} \PYG{n+nn}{periodic\PYGZus{}tasks} \PYG{k}{as} \PYG{n+nn}{pt}
\PYG{g+gp}{\PYGZgt{}\PYGZgt{}\PYGZgt{} }\PYG{n}{ts} \PYG{o}{=} \PYG{n}{pt}\PYG{o}{.}\PYG{n}{TaskScheduler}\PYG{p}{(}\PYG{p}{)}
\PYG{g+gp}{\PYGZgt{}\PYGZgt{}\PYGZgt{} }\PYG{n}{ts}\PYG{o}{.}\PYG{n}{stop}\PYG{p}{(}\PYG{p}{)}
\PYG{g+gt}{Traceback (most recent call last):}
   \PYG{c}{...}
\PYG{g+gr}{Exception}
\end{sphinxVerbatim}

\end{fulllineitems}


\end{fulllineitems}



\section{pydash module}
\label{\detokenize{pydash::doc}}\label{\detokenize{pydash:pydash-module}}\label{\detokenize{pydash:module-pydash}}\index{pydash (module)}

\section{pydash\_app package}
\label{\detokenize{pydash_app::doc}}\label{\detokenize{pydash_app:pydash-app-package}}\label{\detokenize{pydash_app:module-pydash_app}}\index{pydash\_app (module)}
The \sphinxtitleref{pydash\_app} package contains all business domain logic of the PyDash application: Everything that is not part of rendering a set of webpages.
\index{schedule\_periodic\_tasks() (in module pydash\_app)}

\begin{fulllineitems}
\phantomsection\label{\detokenize{pydash_app:pydash_app.schedule_periodic_tasks}}\pysiglinewithargsret{\sphinxcode{\sphinxupquote{pydash\_app.}}\sphinxbfcode{\sphinxupquote{schedule\_periodic\_tasks}}}{}{}
\end{fulllineitems}

\index{seed\_datastructures() (in module pydash\_app)}

\begin{fulllineitems}
\phantomsection\label{\detokenize{pydash_app:pydash_app.seed_datastructures}}\pysiglinewithargsret{\sphinxcode{\sphinxupquote{pydash\_app.}}\sphinxbfcode{\sphinxupquote{seed\_datastructures}}}{}{}
\end{fulllineitems}

\index{start\_task\_scheduler() (in module pydash\_app)}

\begin{fulllineitems}
\phantomsection\label{\detokenize{pydash_app:pydash_app.start_task_scheduler}}\pysiglinewithargsret{\sphinxcode{\sphinxupquote{pydash\_app.}}\sphinxbfcode{\sphinxupquote{start\_task\_scheduler}}}{}{}
\end{fulllineitems}

\index{stop\_task\_scheduler() (in module pydash\_app)}

\begin{fulllineitems}
\phantomsection\label{\detokenize{pydash_app:pydash_app.stop_task_scheduler}}\pysiglinewithargsret{\sphinxcode{\sphinxupquote{pydash\_app.}}\sphinxbfcode{\sphinxupquote{stop\_task\_scheduler}}}{}{}
\end{fulllineitems}



\subsection{Subpackages}
\label{\detokenize{pydash_app:subpackages}}

\subsubsection{pydash\_app.dashboard package}
\label{\detokenize{pydash_app.dashboard::doc}}\label{\detokenize{pydash_app.dashboard:module-pydash_app.dashboard}}\label{\detokenize{pydash_app.dashboard:pydash-app-dashboard-package}}\index{pydash\_app.dashboard (module)}
This module is the public interface (available to the web-application pydash\_web)
for interacting with Dashboards.
\index{add\_to\_repository() (in module pydash\_app.dashboard)}

\begin{fulllineitems}
\phantomsection\label{\detokenize{pydash_app.dashboard:pydash_app.dashboard.add_to_repository}}\pysiglinewithargsret{\sphinxcode{\sphinxupquote{pydash\_app.dashboard.}}\sphinxbfcode{\sphinxupquote{add\_to\_repository}}}{\emph{dashboard}}{}
\end{fulllineitems}

\index{dashboards\_of\_user() (in module pydash\_app.dashboard)}

\begin{fulllineitems}
\phantomsection\label{\detokenize{pydash_app.dashboard:pydash_app.dashboard.dashboards_of_user}}\pysiglinewithargsret{\sphinxcode{\sphinxupquote{pydash\_app.dashboard.}}\sphinxbfcode{\sphinxupquote{dashboards\_of\_user}}}{\emph{user\_id}}{}
Returns a list of Dashboard-entities that are connected to the given user.
:param user\_id: The UUID of the user whose dashboards we’re requesting.
:return: A list of Dashboard-entities.

\end{fulllineitems}

\index{find() (in module pydash\_app.dashboard)}

\begin{fulllineitems}
\phantomsection\label{\detokenize{pydash_app.dashboard:pydash_app.dashboard.find}}\pysiglinewithargsret{\sphinxcode{\sphinxupquote{pydash\_app.dashboard.}}\sphinxbfcode{\sphinxupquote{find}}}{\emph{dashboard\_id}}{}
Returns a single Dashboard-entity with the given UUID or None if it could not be found.
:param dashboard\_id: UUID of the dashboard we hope to find.
:return: The Dashboard-entity with the given UUID or raises an Exception if it could not be found.

\end{fulllineitems}

\index{find\_verified\_dashboard() (in module pydash\_app.dashboard)}

\begin{fulllineitems}
\phantomsection\label{\detokenize{pydash_app.dashboard:pydash_app.dashboard.find_verified_dashboard}}\pysiglinewithargsret{\sphinxcode{\sphinxupquote{pydash\_app.dashboard.}}\sphinxbfcode{\sphinxupquote{find\_verified\_dashboard}}}{\emph{dashboard\_id}}{}
Verifies if a given dashboard\_id is correct and if the current user has access
to the dashboard.
:param dashboard\_id: The UUID of the dashboard to be validated.
:return: True if the dashboard is valid, else False followed by the result and
the http error code.

\end{fulllineitems}

\index{remove\_from\_repository() (in module pydash\_app.dashboard)}

\begin{fulllineitems}
\phantomsection\label{\detokenize{pydash_app.dashboard:pydash_app.dashboard.remove_from_repository}}\pysiglinewithargsret{\sphinxcode{\sphinxupquote{pydash\_app.dashboard.}}\sphinxbfcode{\sphinxupquote{remove\_from\_repository}}}{\emph{dashboard}}{}
\end{fulllineitems}



\paragraph{Subpackages}
\label{\detokenize{pydash_app.dashboard:subpackages}}

\subparagraph{pydash\_app.dashboard.aggregator package}
\label{\detokenize{pydash_app.dashboard.aggregator::doc}}\label{\detokenize{pydash_app.dashboard.aggregator:pydash-app-dashboard-aggregator-package}}\label{\detokenize{pydash_app.dashboard.aggregator:module-pydash_app.dashboard.aggregator}}\index{pydash\_app.dashboard.aggregator (module)}\index{Aggregator (class in pydash\_app.dashboard.aggregator)}

\begin{fulllineitems}
\phantomsection\label{\detokenize{pydash_app.dashboard.aggregator:pydash_app.dashboard.aggregator.Aggregator}}\pysiglinewithargsret{\sphinxbfcode{\sphinxupquote{class }}\sphinxcode{\sphinxupquote{pydash\_app.dashboard.aggregator.}}\sphinxbfcode{\sphinxupquote{Aggregator}}}{\emph{endpoint\_calls={[}{]}}}{}
Bases: \sphinxcode{\sphinxupquote{persistent.Persistent}}

Maintains aggregate data for either a dashboard or a single endpoint.
This data is updated every time a new endpoint call is added.
\index{add\_endpoint\_call() (pydash\_app.dashboard.aggregator.Aggregator method)}

\begin{fulllineitems}
\phantomsection\label{\detokenize{pydash_app.dashboard.aggregator:pydash_app.dashboard.aggregator.Aggregator.add_endpoint_call}}\pysiglinewithargsret{\sphinxbfcode{\sphinxupquote{add\_endpoint\_call}}}{\emph{endpoint\_call}}{}
Add an endpoint call and update aggregated data
:param endpoint\_call: \sphinxtitleref{EndpointCall} instance to add

\end{fulllineitems}

\index{as\_dict() (pydash\_app.dashboard.aggregator.Aggregator method)}

\begin{fulllineitems}
\phantomsection\label{\detokenize{pydash_app.dashboard.aggregator:pydash_app.dashboard.aggregator.Aggregator.as_dict}}\pysiglinewithargsret{\sphinxbfcode{\sphinxupquote{as\_dict}}}{}{}
Return aggregated data in a dict. Only includes statistics that should be rendered.
:return: A dict containing several aggregated data points

\end{fulllineitems}

\index{contained\_statistics\_classes (pydash\_app.dashboard.aggregator.Aggregator attribute)}

\begin{fulllineitems}
\phantomsection\label{\detokenize{pydash_app.dashboard.aggregator:pydash_app.dashboard.aggregator.Aggregator.contained_statistics_classes}}\pysigline{\sphinxbfcode{\sphinxupquote{contained\_statistics\_classes}}\sphinxbfcode{\sphinxupquote{ = OrderedSet({[}\textless{}class 'pydash\_app.dashboard.aggregator.statistics.TotalVisits'\textgreater{}, \textless{}class 'pydash\_app.dashboard.aggregator.statistics.AverageExecutionTime'\textgreater{}, \textless{}class 'pydash\_app.dashboard.aggregator.statistics.VisitsPerIP'\textgreater{}, \textless{}class 'pydash\_app.dashboard.aggregator.statistics.UniqueVisitorsAllTime'\textgreater{}, \textless{}class 'pydash\_app.dashboard.aggregator.statistics.FastestExecutionTime'\textgreater{}, \textless{}class 'pydash\_app.dashboard.aggregator.statistics.FastestQuartileExecutionTime'\textgreater{}, \textless{}class 'pydash\_app.dashboard.aggregator.statistics.MedianExecutionTime'\textgreater{}, \textless{}class 'pydash\_app.dashboard.aggregator.statistics.SlowestQuartileExecutionTime'\textgreater{}, \textless{}class 'pydash\_app.dashboard.aggregator.statistics.NinetiethPercentileExecutionTime'\textgreater{}, \textless{}class 'pydash\_app.dashboard.aggregator.statistics.NinetyNinthPercentileExecutionTime'\textgreater{}, \textless{}class 'pydash\_app.dashboard.aggregator.statistics.SlowestExecutionTime'\textgreater{}, \textless{}class 'pydash\_app.dashboard.aggregator.statistics.Versions'\textgreater{}{]})}}}
\end{fulllineitems}

\index{statistic (pydash\_app.dashboard.aggregator.Aggregator attribute)}

\begin{fulllineitems}
\phantomsection\label{\detokenize{pydash_app.dashboard.aggregator:pydash_app.dashboard.aggregator.Aggregator.statistic}}\pysigline{\sphinxbfcode{\sphinxupquote{statistic}}}
alias of {\hyperref[\detokenize{pydash_app.dashboard.aggregator.statistics:pydash_app.dashboard.aggregator.statistics.Versions}]{\sphinxcrossref{\sphinxcode{\sphinxupquote{pydash\_app.dashboard.aggregator.statistics.Versions}}}}}

\end{fulllineitems}

\index{statistics\_classes\_with\_dependencies (pydash\_app.dashboard.aggregator.Aggregator attribute)}

\begin{fulllineitems}
\phantomsection\label{\detokenize{pydash_app.dashboard.aggregator:pydash_app.dashboard.aggregator.Aggregator.statistics_classes_with_dependencies}}\pysigline{\sphinxbfcode{\sphinxupquote{statistics\_classes\_with\_dependencies}}\sphinxbfcode{\sphinxupquote{ = OrderedSet({[}\textless{}class 'pydash\_app.dashboard.aggregator.statistics.TotalVisits'\textgreater{}, \textless{}class 'pydash\_app.dashboard.aggregator.statistics.TotalExecutionTime'\textgreater{}, \textless{}class 'pydash\_app.dashboard.aggregator.statistics.AverageExecutionTime'\textgreater{}, \textless{}class 'pydash\_app.dashboard.aggregator.statistics.VisitsPerIP'\textgreater{}, \textless{}class 'pydash\_app.dashboard.aggregator.statistics.UniqueVisitorsAllTime'\textgreater{}, \textless{}class 'pydash\_app.dashboard.aggregator.statistics.ExecutionTimeTDigest'\textgreater{}, \textless{}class 'pydash\_app.dashboard.aggregator.statistics.FastestExecutionTime'\textgreater{}, \textless{}class 'pydash\_app.dashboard.aggregator.statistics.FastestQuartileExecutionTime'\textgreater{}, \textless{}class 'pydash\_app.dashboard.aggregator.statistics.MedianExecutionTime'\textgreater{}, \textless{}class 'pydash\_app.dashboard.aggregator.statistics.SlowestQuartileExecutionTime'\textgreater{}, \textless{}class 'pydash\_app.dashboard.aggregator.statistics.NinetiethPercentileExecutionTime'\textgreater{}, \textless{}class 'pydash\_app.dashboard.aggregator.statistics.NinetyNinthPercentileExecutionTime'\textgreater{}, \textless{}class 'pydash\_app.dashboard.aggregator.statistics.SlowestExecutionTime'\textgreater{}, \textless{}class 'pydash\_app.dashboard.aggregator.statistics.Versions'\textgreater{}{]})}}}
\end{fulllineitems}


\end{fulllineitems}



\subparagraph{Submodules}
\label{\detokenize{pydash_app.dashboard.aggregator:submodules}}

\subparagraph{pydash\_app.dashboard.aggregator.aggregator\_group module}
\label{\detokenize{pydash_app.dashboard.aggregator.aggregator_group::doc}}\label{\detokenize{pydash_app.dashboard.aggregator.aggregator_group:module-pydash_app.dashboard.aggregator.aggregator_group}}\label{\detokenize{pydash_app.dashboard.aggregator.aggregator_group:pydash-app-dashboard-aggregator-aggregator-group-module}}\index{pydash\_app.dashboard.aggregator.aggregator\_group (module)}\index{AggregatorGroup (class in pydash\_app.dashboard.aggregator.aggregator\_group)}

\begin{fulllineitems}
\phantomsection\label{\detokenize{pydash_app.dashboard.aggregator.aggregator_group:pydash_app.dashboard.aggregator.aggregator_group.AggregatorGroup}}\pysiglinewithargsret{\sphinxbfcode{\sphinxupquote{class }}\sphinxcode{\sphinxupquote{pydash\_app.dashboard.aggregator.aggregator\_group.}}\sphinxbfcode{\sphinxupquote{AggregatorGroup}}}{\emph{endpoint\_calls={[}{]}}}{}
Bases: \sphinxcode{\sphinxupquote{persistent.Persistent}}

Maintains a powerset of dicts of aggregators,
such that we can filter based on:
- time
- IP
- FMD’s group\_by
- etc.

Involved usage example:
\textgreater{}\textgreater{}\textgreater{} from datetime import datetime
\textgreater{}\textgreater{}\textgreater{} from pydash\_app.dashboard.endpoint\_call import EndpointCall
\textgreater{}\textgreater{}\textgreater{} from pydash\_app.dashboard.aggregator.aggregator\_group import AggregatorGroup
\textgreater{}\textgreater{}\textgreater{} ag = AggregatorGroup()
\textgreater{}\textgreater{}\textgreater{} ec1 = EndpointCall(“foo”, 0.5, datetime.strptime(“2018-04-25 15:29:23”, “\%Y-\%m-\%d \%H:\%M:\%S”), “0.1”, “None”, “127.0.0.1”)
\textgreater{}\textgreater{}\textgreater{} ec2 = EndpointCall(“foo”, 0.5, datetime.strptime(“2018-04-26 15:29:23”, “\%Y-\%m-\%d \%H:\%M:\%S”), “0.1”, “None”, “127.0.0.1”)
\textgreater{}\textgreater{}\textgreater{} ec3 = EndpointCall(“foo”, 0.5, datetime.strptime(“2018-04-25 15:29:23”, “\%Y-\%m-\%d \%H:\%M:\%S”), “0.1”, “None”, “127.0.0.2”)
\textgreater{}\textgreater{}\textgreater{} ag.add\_endpoint\_call(ec1)
\textgreater{}\textgreater{}\textgreater{} ag.add\_endpoint\_call(ec2)
\textgreater{}\textgreater{}\textgreater{} ag.add\_endpoint\_call(ec3)
\textgreater{}\textgreater{}\textgreater{}
\textgreater{}\textgreater{}\textgreater{} \# Filter by day
… a\_day = ag.fetch\_aggregator(\{‘day’:‘2018-04-25’\})
\textgreater{}\textgreater{}\textgreater{} a\_day.as\_dict(){[}‘total\_visits’{]} == 2
True
\textgreater{}\textgreater{}\textgreater{}
\textgreater{}\textgreater{}\textgreater{} \# Filter by week
… a\_week = ag.fetch\_aggregator(\{‘week’:‘2018-W17’\})
\textgreater{}\textgreater{}\textgreater{} a\_week.as\_dict(){[}‘total\_visits’{]} == 3
True
\textgreater{}\textgreater{}\textgreater{}
\textgreater{}\textgreater{}\textgreater{} \# Filter by day and ip
… a\_day\_ip = ag.fetch\_aggregator(\{‘day’:‘2018-04-25’, ‘ip’:‘127.0.0.1’\})
\textgreater{}\textgreater{}\textgreater{} a\_day\_ip.as\_dict(){[}‘total\_visits’{]} == 1
True
\textgreater{}\textgreater{}\textgreater{}
\textgreater{}\textgreater{}\textgreater{} \# No filtering (all endpoint calls are included in this aggregator)
… a\_all = ag.fetch\_aggregator(\{\})
\textgreater{}\textgreater{}\textgreater{} a\_all.as\_dict(){[}‘total\_visits’{]} == 3
True
\textgreater{}\textgreater{}\textgreater{}
\textgreater{}\textgreater{}\textgreater{} \# Filter over a datetime range
… start\_datetime = datetime(ec1.time.year, ec1.time.month, ec1.time.day)
\textgreater{}\textgreater{}\textgreater{} end\_datetime = datetime(ec2.time.year, ec2.time.month, ec2.time.day + 1)
\textgreater{}\textgreater{}\textgreater{} a\_all2 = ag.fetch\_aggregator\_daterange(\{\}, start\_datetime, end\_datetime)
\textgreater{}\textgreater{}\textgreater{} a\_all2.as\_dict(){[}‘total\_visits’{]} == 3
True
\textgreater{}\textgreater{}\textgreater{} a\_all.as\_dict() == a\_all2.as\_dict()
True
\index{add\_endpoint\_call() (pydash\_app.dashboard.aggregator.aggregator\_group.AggregatorGroup method)}

\begin{fulllineitems}
\phantomsection\label{\detokenize{pydash_app.dashboard.aggregator.aggregator_group:pydash_app.dashboard.aggregator.aggregator_group.AggregatorGroup.add_endpoint_call}}\pysiglinewithargsret{\sphinxbfcode{\sphinxupquote{add\_endpoint\_call}}}{\emph{endpoint\_call}}{}
Adds the given endpoint call to the right aggregators within the group.

\end{fulllineitems}

\index{fetch\_aggregator() (pydash\_app.dashboard.aggregator.aggregator\_group.AggregatorGroup method)}

\begin{fulllineitems}
\phantomsection\label{\detokenize{pydash_app.dashboard.aggregator.aggregator_group:pydash_app.dashboard.aggregator.aggregator_group.AggregatorGroup.fetch_aggregator}}\pysiglinewithargsret{\sphinxbfcode{\sphinxupquote{fetch\_aggregator}}}{\emph{filter\_dict=\{\}}}{}
Filters the internal collection of aggregators and returns the right one depending on filter\_dict.
:param filter\_dict: A dictionary containing property\_name-value pairs to filter on.
\begin{quote}

This is in the gist of \sphinxtitleref{\{‘day’:‘2018-05-20’, ‘ip’:‘127.0.0.1’\}}
\begin{description}
\item[{The current filter\_names are:}] \leavevmode\begin{itemize}
\item {} 
Time:
* ‘year’   - e.g. ‘2018’
* ‘month’  - e.g. ‘2018-05’
* ‘week’   - e.g. ‘2018-W17’
* ‘day’    - e.g. ‘2018-05-20’
* ‘hour’   - e.g. ‘2018-05-20T20’
* ‘minute’ - e.g. ‘2018-05-20T20-10’

\end{itemize}

Note that for Time filter-values, the formatting is crucial.
\begin{itemize}
\item {} 
Version:
* ‘version’ - e.g. ‘1.0.1’

\item {} 
IP:
* ‘ip’ - e.g. ‘127.0.0.1’

\item {} 
Group-by:
* ‘group\_by’ - e.g. ‘None’

\end{itemize}

\end{description}

Note that when providing two filters of the same type, a ValueError is raised.
\end{quote}
\begin{quote}\begin{description}
\item[{Returns}] \leavevmode
An Aggregator instance that contains the right aggregated data for this query.
Note that if an invalid value is given, a new (and empty) Aggregator is returned, due to the lazy addition.

\end{description}\end{quote}

\end{fulllineitems}

\index{fetch\_aggregator\_daterange() (pydash\_app.dashboard.aggregator.aggregator\_group.AggregatorGroup method)}

\begin{fulllineitems}
\phantomsection\label{\detokenize{pydash_app.dashboard.aggregator.aggregator_group:pydash_app.dashboard.aggregator.aggregator_group.AggregatorGroup.fetch_aggregator_daterange}}\pysiglinewithargsret{\sphinxbfcode{\sphinxupquote{fetch\_aggregator\_daterange}}}{\emph{filters}, \emph{datetime\_begin}, \emph{datetime\_end}}{}
Fetches an aggregator over the entire provided datetime range.
:param filters: A dictionary that contains property\_name-value pairs to filter on.
\begin{quote}

This is in the gist of \{‘ip’: ‘127.0.0.1’, ‘version’: ‘1.0.1’\}
For the complete set of possible filters, see AggregatorGroup.fetch\_aggregator.
Note: may not contain time-based filters, for obvious reasons.
\end{quote}
\begin{quote}\begin{description}
\item[{Parameters}] \leavevmode\begin{itemize}
\item {} 
\sphinxstyleliteralstrong{\sphinxupquote{datetime\_begin}} \textendash{} A datetime object indicating the inclusive lower bound for the datetime range to
aggregate over.

\item {} 
\sphinxstyleliteralstrong{\sphinxupquote{datetime\_end}} \textendash{} A datetime object indicating the exclusive upper bound for the datetime range to
aggregate over.

\end{itemize}

\item[{Returns}] \leavevmode
An Aggregator object that contains the aggregated data over the entirety of the specified datetime
range.

\end{description}\end{quote}

\end{fulllineitems}

\index{fetch\_aggregator\_inclusive\_daterange() (pydash\_app.dashboard.aggregator.aggregator\_group.AggregatorGroup method)}

\begin{fulllineitems}
\phantomsection\label{\detokenize{pydash_app.dashboard.aggregator.aggregator_group:pydash_app.dashboard.aggregator.aggregator_group.AggregatorGroup.fetch_aggregator_inclusive_daterange}}\pysiglinewithargsret{\sphinxbfcode{\sphinxupquote{fetch\_aggregator\_inclusive\_daterange}}}{\emph{filters}, \emph{datetime\_begin}, \emph{datetime\_end}, \emph{granularity}}{}
Fetches an aggregator over the entire provided datetime range.
:param filters: A dictionary that contains property\_name-value pairs to filter on.
\begin{quote}

This is in the gist of \{‘ip’: ‘127.0.0.1’, ‘version’: ‘1.0.1’\}
For the complete set of possible filters, see AggregatorGroup.fetch\_aggregator.
Note: May not contain time-based filters, for obvious reasons.
\end{quote}
\begin{quote}\begin{description}
\item[{Parameters}] \leavevmode\begin{itemize}
\item {} 
\sphinxstyleliteralstrong{\sphinxupquote{datetime\_begin}} \textendash{} A datetime object indicating the inclusive lower bound for the datetime range to
aggregate over.

\item {} 
\sphinxstyleliteralstrong{\sphinxupquote{datetime\_end}} \textendash{} A datetime object indicating the inclusive upper bound for the datetime range to
aggregate over.

\item {} 
\sphinxstyleliteralstrong{\sphinxupquote{granularity}} \textendash{} A string denoting the granularity of the daterange.

\end{itemize}

\item[{Returns}] \leavevmode
An Aggregator object that contains the aggregated data over the entirety of the specified datetime
range.

\end{description}\end{quote}

\end{fulllineitems}

\index{fetch\_aggregators\_per\_timeslice() (pydash\_app.dashboard.aggregator.aggregator\_group.AggregatorGroup method)}

\begin{fulllineitems}
\phantomsection\label{\detokenize{pydash_app.dashboard.aggregator.aggregator_group:pydash_app.dashboard.aggregator.aggregator_group.AggregatorGroup.fetch_aggregators_per_timeslice}}\pysiglinewithargsret{\sphinxbfcode{\sphinxupquote{fetch\_aggregators\_per\_timeslice}}}{\emph{filters}, \emph{timeslice}, \emph{start\_datetime}, \emph{end\_datetime}}{}
These datetimes are treated as inclusive boundaries of a datetime range (e.g. {[}start\_datetime, end\_datetime{]}.
Assumes start\_datetime and end\_datetime are both from utc.
:param filters: A dictionary that contains property\_name-value pairs to filter on.
\begin{quote}

This is in the gist of \{‘ip’: ‘127.0.0.1’, ‘version’: ‘1.0.1’\}
For the complete set of possible filters, see AggregatorGroup.fetch\_aggregator.
Note: May not contain time-based filters, for obvious reasons.
\end{quote}
\begin{quote}\begin{description}
\item[{Parameters}] \leavevmode\begin{itemize}
\item {} 
\sphinxstyleliteralstrong{\sphinxupquote{timeslice}} \textendash{} A string denoting at what granularity the indicated datetime range should be split.
The currently supported values for this are: ‘year’, ‘month’, ‘week’, ‘day’, ‘hour’ and ‘minute’.

\item {} 
\sphinxstyleliteralstrong{\sphinxupquote{start\_datetime}} \textendash{} A datetime object indicating the inclusive lower bound for the datetime range to
aggregate over.

\item {} 
\sphinxstyleliteralstrong{\sphinxupquote{end\_datetime}} \textendash{} A datetime object indicating the inclusive upper bound for the datetime range to
aggregate over.

\end{itemize}

\item[{Returns}] \leavevmode
A list of tuples consisting of a datetime string (formatted according to the ISO-8601 standard)
and the corresponding aggregator, over the specified datetime range.

\end{description}\end{quote}

\end{fulllineitems}

\index{partition\_funs (pydash\_app.dashboard.aggregator.aggregator\_group.AggregatorGroup attribute)}

\begin{fulllineitems}
\phantomsection\label{\detokenize{pydash_app.dashboard.aggregator.aggregator_group:pydash_app.dashboard.aggregator.aggregator_group.AggregatorGroup.partition_funs}}\pysigline{\sphinxbfcode{\sphinxupquote{partition\_funs}}\sphinxbfcode{\sphinxupquote{ = {[}\textless{}AggregatorPartitionFun field\_name=year category=time \textgreater{}, \textless{}AggregatorPartitionFun field\_name=month category=time \textgreater{}, \textless{}AggregatorPartitionFun field\_name=week category=time \textgreater{}, \textless{}AggregatorPartitionFun field\_name=day category=time \textgreater{}, \textless{}AggregatorPartitionFun field\_name=hour category=time \textgreater{}, \textless{}AggregatorPartitionFun field\_name=minute category=time \textgreater{}, \textless{}AggregatorPartitionFun field\_name=group\_by category=group\_by \textgreater{}, \textless{}AggregatorPartitionFun field\_name=ip category=ip \textgreater{}, \textless{}AggregatorPartitionFun field\_name=version category=version \textgreater{}{]}}}}
Note to our internal dev team:
To add more partitions to filter on, a corresponding AggregatorPartitionFun class instance should be created
(together with its corresponding ‘{\color{red}\bfseries{}partition\_by\_}’ function) and added to the \sphinxtitleref{partition\_funs} list above.

\end{fulllineitems}

\index{partition\_powerset (pydash\_app.dashboard.aggregator.aggregator\_group.AggregatorGroup attribute)}

\begin{fulllineitems}
\phantomsection\label{\detokenize{pydash_app.dashboard.aggregator.aggregator_group:pydash_app.dashboard.aggregator.aggregator_group.AggregatorGroup.partition_powerset}}\pysigline{\sphinxbfcode{\sphinxupquote{partition\_powerset}}\sphinxbfcode{\sphinxupquote{ = \textless{}generator object powerset\_generator\textgreater{}}}}
\end{fulllineitems}

\index{partitions\_set (pydash\_app.dashboard.aggregator.aggregator\_group.AggregatorGroup attribute)}

\begin{fulllineitems}
\phantomsection\label{\detokenize{pydash_app.dashboard.aggregator.aggregator_group:pydash_app.dashboard.aggregator.aggregator_group.AggregatorGroup.partitions_set}}\pysigline{\sphinxbfcode{\sphinxupquote{partitions\_set}}\sphinxbfcode{\sphinxupquote{ = frozenset(\{frozenset(\{\textless{}AggregatorPartitionFun field\_name=group\_by category=group\_by \textgreater{}, \textless{}AggregatorPartitionFun field\_name=day category=time \textgreater{}, \textless{}AggregatorPartitionFun field\_name=version category=version \textgreater{}\}), frozenset(\{\textless{}AggregatorPartitionFun field\_name=minute category=time \textgreater{}, \textless{}AggregatorPartitionFun field\_name=group\_by category=group\_by \textgreater{}\}), frozenset(\{\textless{}AggregatorPartitionFun field\_name=minute category=time \textgreater{}, \textless{}AggregatorPartitionFun field\_name=group\_by category=group\_by \textgreater{}, \textless{}AggregatorPartitionFun field\_name=version category=version \textgreater{}\}), frozenset(), frozenset(\{\textless{}AggregatorPartitionFun field\_name=ip category=ip \textgreater{}, \textless{}AggregatorPartitionFun field\_name=group\_by category=group\_by \textgreater{}, \textless{}AggregatorPartitionFun field\_name=hour category=time \textgreater{}, \textless{}AggregatorPartitionFun field\_name=version category=version \textgreater{}\}), frozenset(\{\textless{}AggregatorPartitionFun field\_name=ip category=ip \textgreater{}, \textless{}AggregatorPartitionFun field\_name=hour category=time \textgreater{}\}), frozenset(\{\textless{}AggregatorPartitionFun field\_name=group\_by category=group\_by \textgreater{}, \textless{}AggregatorPartitionFun field\_name=hour category=time \textgreater{}\}), frozenset(\{\textless{}AggregatorPartitionFun field\_name=group\_by category=group\_by \textgreater{}, \textless{}AggregatorPartitionFun field\_name=month category=time \textgreater{}, \textless{}AggregatorPartitionFun field\_name=version category=version \textgreater{}\}), frozenset(\{\textless{}AggregatorPartitionFun field\_name=ip category=ip \textgreater{}, \textless{}AggregatorPartitionFun field\_name=month category=time \textgreater{}, \textless{}AggregatorPartitionFun field\_name=version category=version \textgreater{}\}), frozenset(\{\textless{}AggregatorPartitionFun field\_name=ip category=ip \textgreater{}, \textless{}AggregatorPartitionFun field\_name=minute category=time \textgreater{}, \textless{}AggregatorPartitionFun field\_name=group\_by category=group\_by \textgreater{}\}), frozenset(\{\textless{}AggregatorPartitionFun field\_name=hour category=time \textgreater{}\}), frozenset(\{\textless{}AggregatorPartitionFun field\_name=group\_by category=group\_by \textgreater{}, \textless{}AggregatorPartitionFun field\_name=version category=version \textgreater{}, \textless{}AggregatorPartitionFun field\_name=week category=time \textgreater{}\}), frozenset(\{\textless{}AggregatorPartitionFun field\_name=hour category=time \textgreater{}, \textless{}AggregatorPartitionFun field\_name=version category=version \textgreater{}\}), frozenset(\{\textless{}AggregatorPartitionFun field\_name=ip category=ip \textgreater{}, \textless{}AggregatorPartitionFun field\_name=group\_by category=group\_by \textgreater{}, \textless{}AggregatorPartitionFun field\_name=month category=time \textgreater{}\}), frozenset(\{\textless{}AggregatorPartitionFun field\_name=ip category=ip \textgreater{}, \textless{}AggregatorPartitionFun field\_name=minute category=time \textgreater{}, \textless{}AggregatorPartitionFun field\_name=version category=version \textgreater{}\}), frozenset(\{\textless{}AggregatorPartitionFun field\_name=ip category=ip \textgreater{}, \textless{}AggregatorPartitionFun field\_name=group\_by category=group\_by \textgreater{}, \textless{}AggregatorPartitionFun field\_name=day category=time \textgreater{}, \textless{}AggregatorPartitionFun field\_name=version category=version \textgreater{}\}), frozenset(\{\textless{}AggregatorPartitionFun field\_name=minute category=time \textgreater{}\}), frozenset(\{\textless{}AggregatorPartitionFun field\_name=ip category=ip \textgreater{}, \textless{}AggregatorPartitionFun field\_name=group\_by category=group\_by \textgreater{}, \textless{}AggregatorPartitionFun field\_name=week category=time \textgreater{}\}), frozenset(\{\textless{}AggregatorPartitionFun field\_name=ip category=ip \textgreater{}, \textless{}AggregatorPartitionFun field\_name=month category=time \textgreater{}\}), frozenset(\{\textless{}AggregatorPartitionFun field\_name=group\_by category=group\_by \textgreater{}, \textless{}AggregatorPartitionFun field\_name=month category=time \textgreater{}\}), frozenset(\{\textless{}AggregatorPartitionFun field\_name=group\_by category=group\_by \textgreater{}, \textless{}AggregatorPartitionFun field\_name=hour category=time \textgreater{}, \textless{}AggregatorPartitionFun field\_name=version category=version \textgreater{}\}), frozenset(\{\textless{}AggregatorPartitionFun field\_name=ip category=ip \textgreater{}, \textless{}AggregatorPartitionFun field\_name=hour category=time \textgreater{}, \textless{}AggregatorPartitionFun field\_name=version category=version \textgreater{}\}), frozenset(\{\textless{}AggregatorPartitionFun field\_name=month category=time \textgreater{}\}), frozenset(\{\textless{}AggregatorPartitionFun field\_name=ip category=ip \textgreater{}, \textless{}AggregatorPartitionFun field\_name=version category=version \textgreater{}, \textless{}AggregatorPartitionFun field\_name=week category=time \textgreater{}\}), frozenset(\{\textless{}AggregatorPartitionFun field\_name=ip category=ip \textgreater{}, \textless{}AggregatorPartitionFun field\_name=week category=time \textgreater{}\}), frozenset(\{\textless{}AggregatorPartitionFun field\_name=month category=time \textgreater{}, \textless{}AggregatorPartitionFun field\_name=version category=version \textgreater{}\}), frozenset(\{\textless{}AggregatorPartitionFun field\_name=ip category=ip \textgreater{}, \textless{}AggregatorPartitionFun field\_name=group\_by category=group\_by \textgreater{}, \textless{}AggregatorPartitionFun field\_name=hour category=time \textgreater{}\}), frozenset(\{\textless{}AggregatorPartitionFun field\_name=ip category=ip \textgreater{}, \textless{}AggregatorPartitionFun field\_name=group\_by category=group\_by \textgreater{}, \textless{}AggregatorPartitionFun field\_name=version category=version \textgreater{}\}), frozenset(\{\textless{}AggregatorPartitionFun field\_name=version category=version \textgreater{}, \textless{}AggregatorPartitionFun field\_name=week category=time \textgreater{}\}), frozenset(\{\textless{}AggregatorPartitionFun field\_name=group\_by category=group\_by \textgreater{}, \textless{}AggregatorPartitionFun field\_name=year category=time \textgreater{}, \textless{}AggregatorPartitionFun field\_name=version category=version \textgreater{}\}), frozenset(\{\textless{}AggregatorPartitionFun field\_name=ip category=ip \textgreater{}, \textless{}AggregatorPartitionFun field\_name=year category=time \textgreater{}, \textless{}AggregatorPartitionFun field\_name=version category=version \textgreater{}\}), frozenset(\{\textless{}AggregatorPartitionFun field\_name=group\_by category=group\_by \textgreater{}, \textless{}AggregatorPartitionFun field\_name=day category=time \textgreater{}\}), frozenset(\{\textless{}AggregatorPartitionFun field\_name=ip category=ip \textgreater{}, \textless{}AggregatorPartitionFun field\_name=version category=version \textgreater{}\}), frozenset(\{\textless{}AggregatorPartitionFun field\_name=group\_by category=group\_by \textgreater{}, \textless{}AggregatorPartitionFun field\_name=week category=time \textgreater{}\}), frozenset(\{\textless{}AggregatorPartitionFun field\_name=ip category=ip \textgreater{}, \textless{}AggregatorPartitionFun field\_name=group\_by category=group\_by \textgreater{}, \textless{}AggregatorPartitionFun field\_name=year category=time \textgreater{}\}), frozenset(\{\textless{}AggregatorPartitionFun field\_name=version category=version \textgreater{}\}), frozenset(\{\textless{}AggregatorPartitionFun field\_name=day category=time \textgreater{}, \textless{}AggregatorPartitionFun field\_name=version category=version \textgreater{}\}), frozenset(\{\textless{}AggregatorPartitionFun field\_name=year category=time \textgreater{}\}), frozenset(\{\textless{}AggregatorPartitionFun field\_name=ip category=ip \textgreater{}, \textless{}AggregatorPartitionFun field\_name=group\_by category=group\_by \textgreater{}\}), frozenset(\{\textless{}AggregatorPartitionFun field\_name=day category=time \textgreater{}\}), frozenset(\{\textless{}AggregatorPartitionFun field\_name=group\_by category=group\_by \textgreater{}\}), frozenset(\{\textless{}AggregatorPartitionFun field\_name=ip category=ip \textgreater{}\}), frozenset(\{\textless{}AggregatorPartitionFun field\_name=group\_by category=group\_by \textgreater{}, \textless{}AggregatorPartitionFun field\_name=year category=time \textgreater{}\}), frozenset(\{\textless{}AggregatorPartitionFun field\_name=ip category=ip \textgreater{}, \textless{}AggregatorPartitionFun field\_name=year category=time \textgreater{}\}), frozenset(\{\textless{}AggregatorPartitionFun field\_name=ip category=ip \textgreater{}, \textless{}AggregatorPartitionFun field\_name=day category=time \textgreater{}\}), frozenset(\{\textless{}AggregatorPartitionFun field\_name=group\_by category=group\_by \textgreater{}, \textless{}AggregatorPartitionFun field\_name=version category=version \textgreater{}\}), frozenset(\{\textless{}AggregatorPartitionFun field\_name=ip category=ip \textgreater{}, \textless{}AggregatorPartitionFun field\_name=day category=time \textgreater{}, \textless{}AggregatorPartitionFun field\_name=version category=version \textgreater{}\}), frozenset(\{\textless{}AggregatorPartitionFun field\_name=ip category=ip \textgreater{}, \textless{}AggregatorPartitionFun field\_name=group\_by category=group\_by \textgreater{}, \textless{}AggregatorPartitionFun field\_name=year category=time \textgreater{}, \textless{}AggregatorPartitionFun field\_name=version category=version \textgreater{}\}), frozenset(\{\textless{}AggregatorPartitionFun field\_name=ip category=ip \textgreater{}, \textless{}AggregatorPartitionFun field\_name=minute category=time \textgreater{}, \textless{}AggregatorPartitionFun field\_name=group\_by category=group\_by \textgreater{}, \textless{}AggregatorPartitionFun field\_name=version category=version \textgreater{}\}), frozenset(\{\textless{}AggregatorPartitionFun field\_name=ip category=ip \textgreater{}, \textless{}AggregatorPartitionFun field\_name=minute category=time \textgreater{}\}), frozenset(\{\textless{}AggregatorPartitionFun field\_name=week category=time \textgreater{}\}), frozenset(\{\textless{}AggregatorPartitionFun field\_name=year category=time \textgreater{}, \textless{}AggregatorPartitionFun field\_name=version category=version \textgreater{}\}), frozenset(\{\textless{}AggregatorPartitionFun field\_name=ip category=ip \textgreater{}, \textless{}AggregatorPartitionFun field\_name=group\_by category=group\_by \textgreater{}, \textless{}AggregatorPartitionFun field\_name=month category=time \textgreater{}, \textless{}AggregatorPartitionFun field\_name=version category=version \textgreater{}\}), frozenset(\{\textless{}AggregatorPartitionFun field\_name=ip category=ip \textgreater{}, \textless{}AggregatorPartitionFun field\_name=group\_by category=group\_by \textgreater{}, \textless{}AggregatorPartitionFun field\_name=day category=time \textgreater{}\}), frozenset(\{\textless{}AggregatorPartitionFun field\_name=minute category=time \textgreater{}, \textless{}AggregatorPartitionFun field\_name=version category=version \textgreater{}\}), frozenset(\{\textless{}AggregatorPartitionFun field\_name=ip category=ip \textgreater{}, \textless{}AggregatorPartitionFun field\_name=group\_by category=group\_by \textgreater{}, \textless{}AggregatorPartitionFun field\_name=version category=version \textgreater{}, \textless{}AggregatorPartitionFun field\_name=week category=time \textgreater{}\})\})}}}
\end{fulllineitems}


\end{fulllineitems}

\index{AggregatorPartitionFun (class in pydash\_app.dashboard.aggregator.aggregator\_group)}

\begin{fulllineitems}
\phantomsection\label{\detokenize{pydash_app.dashboard.aggregator.aggregator_group:pydash_app.dashboard.aggregator.aggregator_group.AggregatorPartitionFun}}\pysiglinewithargsret{\sphinxbfcode{\sphinxupquote{class }}\sphinxcode{\sphinxupquote{pydash\_app.dashboard.aggregator.aggregator\_group.}}\sphinxbfcode{\sphinxupquote{AggregatorPartitionFun}}}{\emph{field\_name}, \emph{category}, \emph{fun}}{}
Bases: \sphinxhref{https://docs.python.org/3/library/functions.html\#object}{\sphinxcode{\sphinxupquote{object}}}

\end{fulllineitems}

\index{calc\_endpoint\_call\_identifier() (in module pydash\_app.dashboard.aggregator.aggregator\_group)}

\begin{fulllineitems}
\phantomsection\label{\detokenize{pydash_app.dashboard.aggregator.aggregator_group:pydash_app.dashboard.aggregator.aggregator_group.calc_endpoint_call_identifier}}\pysiglinewithargsret{\sphinxcode{\sphinxupquote{pydash\_app.dashboard.aggregator.aggregator\_group.}}\sphinxbfcode{\sphinxupquote{calc\_endpoint\_call\_identifier}}}{\emph{partition}, \emph{endpoint\_call}}{}
\end{fulllineitems}

\index{partition\_by\_day\_fun() (in module pydash\_app.dashboard.aggregator.aggregator\_group)}

\begin{fulllineitems}
\phantomsection\label{\detokenize{pydash_app.dashboard.aggregator.aggregator_group:pydash_app.dashboard.aggregator.aggregator_group.partition_by_day_fun}}\pysiglinewithargsret{\sphinxcode{\sphinxupquote{pydash\_app.dashboard.aggregator.aggregator\_group.}}\sphinxbfcode{\sphinxupquote{partition\_by\_day\_fun}}}{\emph{endpoint\_call}}{}
\end{fulllineitems}

\index{partition\_by\_group\_by\_fun() (in module pydash\_app.dashboard.aggregator.aggregator\_group)}

\begin{fulllineitems}
\phantomsection\label{\detokenize{pydash_app.dashboard.aggregator.aggregator_group:pydash_app.dashboard.aggregator.aggregator_group.partition_by_group_by_fun}}\pysiglinewithargsret{\sphinxcode{\sphinxupquote{pydash\_app.dashboard.aggregator.aggregator\_group.}}\sphinxbfcode{\sphinxupquote{partition\_by\_group\_by\_fun}}}{\emph{endpoint\_call}}{}
\end{fulllineitems}

\index{partition\_by\_hour\_fun() (in module pydash\_app.dashboard.aggregator.aggregator\_group)}

\begin{fulllineitems}
\phantomsection\label{\detokenize{pydash_app.dashboard.aggregator.aggregator_group:pydash_app.dashboard.aggregator.aggregator_group.partition_by_hour_fun}}\pysiglinewithargsret{\sphinxcode{\sphinxupquote{pydash\_app.dashboard.aggregator.aggregator\_group.}}\sphinxbfcode{\sphinxupquote{partition\_by\_hour\_fun}}}{\emph{endpoint\_call}}{}
\end{fulllineitems}

\index{partition\_by\_ip\_fun() (in module pydash\_app.dashboard.aggregator.aggregator\_group)}

\begin{fulllineitems}
\phantomsection\label{\detokenize{pydash_app.dashboard.aggregator.aggregator_group:pydash_app.dashboard.aggregator.aggregator_group.partition_by_ip_fun}}\pysiglinewithargsret{\sphinxcode{\sphinxupquote{pydash\_app.dashboard.aggregator.aggregator\_group.}}\sphinxbfcode{\sphinxupquote{partition\_by\_ip\_fun}}}{\emph{endpoint\_call}}{}
\end{fulllineitems}

\index{partition\_by\_minute\_fun() (in module pydash\_app.dashboard.aggregator.aggregator\_group)}

\begin{fulllineitems}
\phantomsection\label{\detokenize{pydash_app.dashboard.aggregator.aggregator_group:pydash_app.dashboard.aggregator.aggregator_group.partition_by_minute_fun}}\pysiglinewithargsret{\sphinxcode{\sphinxupquote{pydash\_app.dashboard.aggregator.aggregator\_group.}}\sphinxbfcode{\sphinxupquote{partition\_by\_minute\_fun}}}{\emph{endpoint\_call}}{}
\end{fulllineitems}

\index{partition\_by\_month\_fun() (in module pydash\_app.dashboard.aggregator.aggregator\_group)}

\begin{fulllineitems}
\phantomsection\label{\detokenize{pydash_app.dashboard.aggregator.aggregator_group:pydash_app.dashboard.aggregator.aggregator_group.partition_by_month_fun}}\pysiglinewithargsret{\sphinxcode{\sphinxupquote{pydash\_app.dashboard.aggregator.aggregator\_group.}}\sphinxbfcode{\sphinxupquote{partition\_by\_month\_fun}}}{\emph{endpoint\_call}}{}
\end{fulllineitems}

\index{partition\_by\_version\_fun() (in module pydash\_app.dashboard.aggregator.aggregator\_group)}

\begin{fulllineitems}
\phantomsection\label{\detokenize{pydash_app.dashboard.aggregator.aggregator_group:pydash_app.dashboard.aggregator.aggregator_group.partition_by_version_fun}}\pysiglinewithargsret{\sphinxcode{\sphinxupquote{pydash\_app.dashboard.aggregator.aggregator\_group.}}\sphinxbfcode{\sphinxupquote{partition\_by\_version\_fun}}}{\emph{endpoint\_call}}{}
\end{fulllineitems}

\index{partition\_by\_week\_fun() (in module pydash\_app.dashboard.aggregator.aggregator\_group)}

\begin{fulllineitems}
\phantomsection\label{\detokenize{pydash_app.dashboard.aggregator.aggregator_group:pydash_app.dashboard.aggregator.aggregator_group.partition_by_week_fun}}\pysiglinewithargsret{\sphinxcode{\sphinxupquote{pydash\_app.dashboard.aggregator.aggregator\_group.}}\sphinxbfcode{\sphinxupquote{partition\_by\_week\_fun}}}{\emph{endpoint\_call}}{}
\end{fulllineitems}

\index{partition\_by\_year\_fun() (in module pydash\_app.dashboard.aggregator.aggregator\_group)}

\begin{fulllineitems}
\phantomsection\label{\detokenize{pydash_app.dashboard.aggregator.aggregator_group:pydash_app.dashboard.aggregator.aggregator_group.partition_by_year_fun}}\pysiglinewithargsret{\sphinxcode{\sphinxupquote{pydash\_app.dashboard.aggregator.aggregator\_group.}}\sphinxbfcode{\sphinxupquote{partition\_by\_year\_fun}}}{\emph{endpoint\_call}}{}
\end{fulllineitems}

\index{partition\_field\_names() (in module pydash\_app.dashboard.aggregator.aggregator\_group)}

\begin{fulllineitems}
\phantomsection\label{\detokenize{pydash_app.dashboard.aggregator.aggregator_group:pydash_app.dashboard.aggregator.aggregator_group.partition_field_names}}\pysiglinewithargsret{\sphinxcode{\sphinxupquote{pydash\_app.dashboard.aggregator.aggregator\_group.}}\sphinxbfcode{\sphinxupquote{partition\_field\_names}}}{\emph{partition}}{}
\end{fulllineitems}

\index{powerset\_generator() (in module pydash\_app.dashboard.aggregator.aggregator\_group)}

\begin{fulllineitems}
\phantomsection\label{\detokenize{pydash_app.dashboard.aggregator.aggregator_group:pydash_app.dashboard.aggregator.aggregator_group.powerset_generator}}\pysiglinewithargsret{\sphinxcode{\sphinxupquote{pydash\_app.dashboard.aggregator.aggregator\_group.}}\sphinxbfcode{\sphinxupquote{powerset\_generator}}}{\emph{i}}{}
\end{fulllineitems}

\index{remove\_duplicate\_categories() (in module pydash\_app.dashboard.aggregator.aggregator\_group)}

\begin{fulllineitems}
\phantomsection\label{\detokenize{pydash_app.dashboard.aggregator.aggregator_group:pydash_app.dashboard.aggregator.aggregator_group.remove_duplicate_categories}}\pysiglinewithargsret{\sphinxcode{\sphinxupquote{pydash\_app.dashboard.aggregator.aggregator\_group.}}\sphinxbfcode{\sphinxupquote{remove\_duplicate\_categories}}}{\emph{partition\_funs}}{}
\end{fulllineitems}



\subparagraph{pydash\_app.dashboard.aggregator.statistics module}
\label{\detokenize{pydash_app.dashboard.aggregator.statistics::doc}}\label{\detokenize{pydash_app.dashboard.aggregator.statistics:pydash-app-dashboard-aggregator-statistics-module}}\label{\detokenize{pydash_app.dashboard.aggregator.statistics:module-pydash_app.dashboard.aggregator.statistics}}\index{pydash\_app.dashboard.aggregator.statistics (module)}\index{AverageExecutionTime (class in pydash\_app.dashboard.aggregator.statistics)}

\begin{fulllineitems}
\phantomsection\label{\detokenize{pydash_app.dashboard.aggregator.statistics:pydash_app.dashboard.aggregator.statistics.AverageExecutionTime}}\pysigline{\sphinxbfcode{\sphinxupquote{class }}\sphinxcode{\sphinxupquote{pydash\_app.dashboard.aggregator.statistics.}}\sphinxbfcode{\sphinxupquote{AverageExecutionTime}}}
Bases: {\hyperref[\detokenize{pydash_app.dashboard.aggregator.statistics:pydash_app.dashboard.aggregator.statistics.FloatStatisticABC}]{\sphinxcrossref{\sphinxcode{\sphinxupquote{pydash\_app.dashboard.aggregator.statistics.FloatStatisticABC}}}}}

Keeps track of the average execution time of all endpoints that have been appended to it.
Rendered value is rounded to 3 decimal places by default.
\index{add\_together() (pydash\_app.dashboard.aggregator.statistics.AverageExecutionTime method)}

\begin{fulllineitems}
\phantomsection\label{\detokenize{pydash_app.dashboard.aggregator.statistics:pydash_app.dashboard.aggregator.statistics.AverageExecutionTime.add_together}}\pysiglinewithargsret{\sphinxbfcode{\sphinxupquote{add\_together}}}{\emph{other}, \emph{dependencies\_self}, \emph{dependencies\_other}}{}
Should return a new statistic where the internals of self and other are added together.

\end{fulllineitems}

\index{dependencies (pydash\_app.dashboard.aggregator.statistics.AverageExecutionTime attribute)}

\begin{fulllineitems}
\phantomsection\label{\detokenize{pydash_app.dashboard.aggregator.statistics:pydash_app.dashboard.aggregator.statistics.AverageExecutionTime.dependencies}}\pysigline{\sphinxbfcode{\sphinxupquote{dependencies}}\sphinxbfcode{\sphinxupquote{ = {[}\textless{}class 'pydash\_app.dashboard.aggregator.statistics.TotalVisits'\textgreater{}, \textless{}class 'pydash\_app.dashboard.aggregator.statistics.TotalExecutionTime'\textgreater{}{]}}}}
\end{fulllineitems}

\index{empty() (pydash\_app.dashboard.aggregator.statistics.AverageExecutionTime method)}

\begin{fulllineitems}
\phantomsection\label{\detokenize{pydash_app.dashboard.aggregator.statistics:pydash_app.dashboard.aggregator.statistics.AverageExecutionTime.empty}}\pysiglinewithargsret{\sphinxbfcode{\sphinxupquote{empty}}}{}{}
\end{fulllineitems}

\index{field\_name() (pydash\_app.dashboard.aggregator.statistics.AverageExecutionTime method)}

\begin{fulllineitems}
\phantomsection\label{\detokenize{pydash_app.dashboard.aggregator.statistics:pydash_app.dashboard.aggregator.statistics.AverageExecutionTime.field_name}}\pysiglinewithargsret{\sphinxbfcode{\sphinxupquote{field\_name}}}{}{}
\end{fulllineitems}

\index{perform\_append() (pydash\_app.dashboard.aggregator.statistics.AverageExecutionTime method)}

\begin{fulllineitems}
\phantomsection\label{\detokenize{pydash_app.dashboard.aggregator.statistics:pydash_app.dashboard.aggregator.statistics.AverageExecutionTime.perform_append}}\pysiglinewithargsret{\sphinxbfcode{\sphinxupquote{perform\_append}}}{\emph{endpoint\_call}, \emph{dependencies}}{}
\end{fulllineitems}

\index{should\_be\_rendered() (pydash\_app.dashboard.aggregator.statistics.AverageExecutionTime method)}

\begin{fulllineitems}
\phantomsection\label{\detokenize{pydash_app.dashboard.aggregator.statistics:pydash_app.dashboard.aggregator.statistics.AverageExecutionTime.should_be_rendered}}\pysiglinewithargsret{\sphinxbfcode{\sphinxupquote{should\_be\_rendered}}}{}{}
Note: implementing subclasses should add the @property decorator.
There was some strange behaviour where without adding the decorator,
subclasses implementing it as \sphinxtitleref{return True} behaved normally, but those implementing it as \sphinxtitleref{return False} still
were treated as if it returned True. Adding the @property decorator fixed it.

\end{fulllineitems}


\end{fulllineitems}

\index{ExecutionTimePercentileABC (class in pydash\_app.dashboard.aggregator.statistics)}

\begin{fulllineitems}
\phantomsection\label{\detokenize{pydash_app.dashboard.aggregator.statistics:pydash_app.dashboard.aggregator.statistics.ExecutionTimePercentileABC}}\pysigline{\sphinxbfcode{\sphinxupquote{class }}\sphinxcode{\sphinxupquote{pydash\_app.dashboard.aggregator.statistics.}}\sphinxbfcode{\sphinxupquote{ExecutionTimePercentileABC}}}
Bases: {\hyperref[\detokenize{pydash_app.dashboard.aggregator.statistics:pydash_app.dashboard.aggregator.statistics.FloatStatisticABC}]{\sphinxcrossref{\sphinxcode{\sphinxupquote{pydash\_app.dashboard.aggregator.statistics.FloatStatisticABC}}}}}

Abstract base class for execution time percentile statistics.
\index{add\_together() (pydash\_app.dashboard.aggregator.statistics.ExecutionTimePercentileABC method)}

\begin{fulllineitems}
\phantomsection\label{\detokenize{pydash_app.dashboard.aggregator.statistics:pydash_app.dashboard.aggregator.statistics.ExecutionTimePercentileABC.add_together}}\pysiglinewithargsret{\sphinxbfcode{\sphinxupquote{add\_together}}}{\emph{other}, \emph{dependencies\_self}, \emph{dependencies\_other}}{}
Should return a new statistic where the internals of self and other are added together.

\end{fulllineitems}

\index{dependencies (pydash\_app.dashboard.aggregator.statistics.ExecutionTimePercentileABC attribute)}

\begin{fulllineitems}
\phantomsection\label{\detokenize{pydash_app.dashboard.aggregator.statistics:pydash_app.dashboard.aggregator.statistics.ExecutionTimePercentileABC.dependencies}}\pysigline{\sphinxbfcode{\sphinxupquote{dependencies}}\sphinxbfcode{\sphinxupquote{ = {[}\textless{}class 'pydash\_app.dashboard.aggregator.statistics.ExecutionTimeTDigest'\textgreater{}{]}}}}
\end{fulllineitems}

\index{empty() (pydash\_app.dashboard.aggregator.statistics.ExecutionTimePercentileABC method)}

\begin{fulllineitems}
\phantomsection\label{\detokenize{pydash_app.dashboard.aggregator.statistics:pydash_app.dashboard.aggregator.statistics.ExecutionTimePercentileABC.empty}}\pysiglinewithargsret{\sphinxbfcode{\sphinxupquote{empty}}}{}{}
\end{fulllineitems}

\index{percentile\_nr (pydash\_app.dashboard.aggregator.statistics.ExecutionTimePercentileABC attribute)}

\begin{fulllineitems}
\phantomsection\label{\detokenize{pydash_app.dashboard.aggregator.statistics:pydash_app.dashboard.aggregator.statistics.ExecutionTimePercentileABC.percentile_nr}}\pysigline{\sphinxbfcode{\sphinxupquote{percentile\_nr}}}
\end{fulllineitems}

\index{perform\_append() (pydash\_app.dashboard.aggregator.statistics.ExecutionTimePercentileABC method)}

\begin{fulllineitems}
\phantomsection\label{\detokenize{pydash_app.dashboard.aggregator.statistics:pydash_app.dashboard.aggregator.statistics.ExecutionTimePercentileABC.perform_append}}\pysiglinewithargsret{\sphinxbfcode{\sphinxupquote{perform\_append}}}{\emph{endpoint\_call}, \emph{dependencies}}{}
\end{fulllineitems}

\index{should\_be\_rendered() (pydash\_app.dashboard.aggregator.statistics.ExecutionTimePercentileABC method)}

\begin{fulllineitems}
\phantomsection\label{\detokenize{pydash_app.dashboard.aggregator.statistics:pydash_app.dashboard.aggregator.statistics.ExecutionTimePercentileABC.should_be_rendered}}\pysiglinewithargsret{\sphinxbfcode{\sphinxupquote{should\_be\_rendered}}}{}{}
Note: implementing subclasses should add the @property decorator.
There was some strange behaviour where without adding the decorator,
subclasses implementing it as \sphinxtitleref{return True} behaved normally, but those implementing it as \sphinxtitleref{return False} still
were treated as if it returned True. Adding the @property decorator fixed it.

\end{fulllineitems}


\end{fulllineitems}

\index{ExecutionTimeTDigest (class in pydash\_app.dashboard.aggregator.statistics)}

\begin{fulllineitems}
\phantomsection\label{\detokenize{pydash_app.dashboard.aggregator.statistics:pydash_app.dashboard.aggregator.statistics.ExecutionTimeTDigest}}\pysigline{\sphinxbfcode{\sphinxupquote{class }}\sphinxcode{\sphinxupquote{pydash\_app.dashboard.aggregator.statistics.}}\sphinxbfcode{\sphinxupquote{ExecutionTimeTDigest}}}
Bases: {\hyperref[\detokenize{pydash_app.dashboard.aggregator.statistics:pydash_app.dashboard.aggregator.statistics.Statistic}]{\sphinxcrossref{\sphinxcode{\sphinxupquote{pydash\_app.dashboard.aggregator.statistics.Statistic}}}}}

Acts as the general execution time tdigest, from which its dependants take their data from.
This class is supposed to be instantiated, but not rendered.
\index{add\_together() (pydash\_app.dashboard.aggregator.statistics.ExecutionTimeTDigest method)}

\begin{fulllineitems}
\phantomsection\label{\detokenize{pydash_app.dashboard.aggregator.statistics:pydash_app.dashboard.aggregator.statistics.ExecutionTimeTDigest.add_together}}\pysiglinewithargsret{\sphinxbfcode{\sphinxupquote{add\_together}}}{\emph{other}, \emph{dependencies\_self}, \emph{dependencies\_other}}{}
Should return a new statistic where the internals of self and other are added together.

\end{fulllineitems}

\index{empty() (pydash\_app.dashboard.aggregator.statistics.ExecutionTimeTDigest method)}

\begin{fulllineitems}
\phantomsection\label{\detokenize{pydash_app.dashboard.aggregator.statistics:pydash_app.dashboard.aggregator.statistics.ExecutionTimeTDigest.empty}}\pysiglinewithargsret{\sphinxbfcode{\sphinxupquote{empty}}}{}{}
\end{fulllineitems}

\index{field\_name() (pydash\_app.dashboard.aggregator.statistics.ExecutionTimeTDigest method)}

\begin{fulllineitems}
\phantomsection\label{\detokenize{pydash_app.dashboard.aggregator.statistics:pydash_app.dashboard.aggregator.statistics.ExecutionTimeTDigest.field_name}}\pysiglinewithargsret{\sphinxbfcode{\sphinxupquote{field\_name}}}{}{}
\end{fulllineitems}

\index{perform\_append() (pydash\_app.dashboard.aggregator.statistics.ExecutionTimeTDigest method)}

\begin{fulllineitems}
\phantomsection\label{\detokenize{pydash_app.dashboard.aggregator.statistics:pydash_app.dashboard.aggregator.statistics.ExecutionTimeTDigest.perform_append}}\pysiglinewithargsret{\sphinxbfcode{\sphinxupquote{perform\_append}}}{\emph{endpoint\_call}, \emph{dependencies}}{}
\end{fulllineitems}

\index{should\_be\_rendered (pydash\_app.dashboard.aggregator.statistics.ExecutionTimeTDigest attribute)}

\begin{fulllineitems}
\phantomsection\label{\detokenize{pydash_app.dashboard.aggregator.statistics:pydash_app.dashboard.aggregator.statistics.ExecutionTimeTDigest.should_be_rendered}}\pysigline{\sphinxbfcode{\sphinxupquote{should\_be\_rendered}}}
Note: implementing subclasses should add the @property decorator.
There was some strange behaviour where without adding the decorator,
subclasses implementing it as \sphinxtitleref{return True} behaved normally, but those implementing it as \sphinxtitleref{return False} still
were treated as if it returned True. Adding the @property decorator fixed it.

\end{fulllineitems}


\end{fulllineitems}

\index{FastestExecutionTime (class in pydash\_app.dashboard.aggregator.statistics)}

\begin{fulllineitems}
\phantomsection\label{\detokenize{pydash_app.dashboard.aggregator.statistics:pydash_app.dashboard.aggregator.statistics.FastestExecutionTime}}\pysigline{\sphinxbfcode{\sphinxupquote{class }}\sphinxcode{\sphinxupquote{pydash\_app.dashboard.aggregator.statistics.}}\sphinxbfcode{\sphinxupquote{FastestExecutionTime}}}
Bases: {\hyperref[\detokenize{pydash_app.dashboard.aggregator.statistics:pydash_app.dashboard.aggregator.statistics.ExecutionTimePercentileABC}]{\sphinxcrossref{\sphinxcode{\sphinxupquote{pydash\_app.dashboard.aggregator.statistics.ExecutionTimePercentileABC}}}}}
\index{field\_name() (pydash\_app.dashboard.aggregator.statistics.FastestExecutionTime method)}

\begin{fulllineitems}
\phantomsection\label{\detokenize{pydash_app.dashboard.aggregator.statistics:pydash_app.dashboard.aggregator.statistics.FastestExecutionTime.field_name}}\pysiglinewithargsret{\sphinxbfcode{\sphinxupquote{field\_name}}}{}{}
\end{fulllineitems}

\index{percentile\_nr() (pydash\_app.dashboard.aggregator.statistics.FastestExecutionTime method)}

\begin{fulllineitems}
\phantomsection\label{\detokenize{pydash_app.dashboard.aggregator.statistics:pydash_app.dashboard.aggregator.statistics.FastestExecutionTime.percentile_nr}}\pysiglinewithargsret{\sphinxbfcode{\sphinxupquote{percentile\_nr}}}{}{}
\end{fulllineitems}


\end{fulllineitems}

\index{FastestQuartileExecutionTime (class in pydash\_app.dashboard.aggregator.statistics)}

\begin{fulllineitems}
\phantomsection\label{\detokenize{pydash_app.dashboard.aggregator.statistics:pydash_app.dashboard.aggregator.statistics.FastestQuartileExecutionTime}}\pysigline{\sphinxbfcode{\sphinxupquote{class }}\sphinxcode{\sphinxupquote{pydash\_app.dashboard.aggregator.statistics.}}\sphinxbfcode{\sphinxupquote{FastestQuartileExecutionTime}}}
Bases: {\hyperref[\detokenize{pydash_app.dashboard.aggregator.statistics:pydash_app.dashboard.aggregator.statistics.ExecutionTimePercentileABC}]{\sphinxcrossref{\sphinxcode{\sphinxupquote{pydash\_app.dashboard.aggregator.statistics.ExecutionTimePercentileABC}}}}}
\index{field\_name() (pydash\_app.dashboard.aggregator.statistics.FastestQuartileExecutionTime method)}

\begin{fulllineitems}
\phantomsection\label{\detokenize{pydash_app.dashboard.aggregator.statistics:pydash_app.dashboard.aggregator.statistics.FastestQuartileExecutionTime.field_name}}\pysiglinewithargsret{\sphinxbfcode{\sphinxupquote{field\_name}}}{}{}
\end{fulllineitems}

\index{percentile\_nr() (pydash\_app.dashboard.aggregator.statistics.FastestQuartileExecutionTime method)}

\begin{fulllineitems}
\phantomsection\label{\detokenize{pydash_app.dashboard.aggregator.statistics:pydash_app.dashboard.aggregator.statistics.FastestQuartileExecutionTime.percentile_nr}}\pysiglinewithargsret{\sphinxbfcode{\sphinxupquote{percentile\_nr}}}{}{}
\end{fulllineitems}


\end{fulllineitems}

\index{FloatStatisticABC (class in pydash\_app.dashboard.aggregator.statistics)}

\begin{fulllineitems}
\phantomsection\label{\detokenize{pydash_app.dashboard.aggregator.statistics:pydash_app.dashboard.aggregator.statistics.FloatStatisticABC}}\pysigline{\sphinxbfcode{\sphinxupquote{class }}\sphinxcode{\sphinxupquote{pydash\_app.dashboard.aggregator.statistics.}}\sphinxbfcode{\sphinxupquote{FloatStatisticABC}}}
Bases: {\hyperref[\detokenize{pydash_app.dashboard.aggregator.statistics:pydash_app.dashboard.aggregator.statistics.Statistic}]{\sphinxcrossref{\sphinxcode{\sphinxupquote{pydash\_app.dashboard.aggregator.statistics.Statistic}}}}}

The FloatStatisticABC is the abstract base class for statistics that render a single floating point number.
It specifies the default amount of digits to round its rendered value to as 3.
(E.g. 2.54, 123, 0.3, but not 0.123)
\index{nr\_of\_digits (pydash\_app.dashboard.aggregator.statistics.FloatStatisticABC attribute)}

\begin{fulllineitems}
\phantomsection\label{\detokenize{pydash_app.dashboard.aggregator.statistics:pydash_app.dashboard.aggregator.statistics.FloatStatisticABC.nr_of_digits}}\pysigline{\sphinxbfcode{\sphinxupquote{nr\_of\_digits}}}
\end{fulllineitems}

\index{rendered\_value() (pydash\_app.dashboard.aggregator.statistics.FloatStatisticABC method)}

\begin{fulllineitems}
\phantomsection\label{\detokenize{pydash_app.dashboard.aggregator.statistics:pydash_app.dashboard.aggregator.statistics.FloatStatisticABC.rendered_value}}\pysiglinewithargsret{\sphinxbfcode{\sphinxupquote{rendered\_value}}}{}{}
\end{fulllineitems}


\end{fulllineitems}

\index{MedianExecutionTime (class in pydash\_app.dashboard.aggregator.statistics)}

\begin{fulllineitems}
\phantomsection\label{\detokenize{pydash_app.dashboard.aggregator.statistics:pydash_app.dashboard.aggregator.statistics.MedianExecutionTime}}\pysigline{\sphinxbfcode{\sphinxupquote{class }}\sphinxcode{\sphinxupquote{pydash\_app.dashboard.aggregator.statistics.}}\sphinxbfcode{\sphinxupquote{MedianExecutionTime}}}
Bases: {\hyperref[\detokenize{pydash_app.dashboard.aggregator.statistics:pydash_app.dashboard.aggregator.statistics.ExecutionTimePercentileABC}]{\sphinxcrossref{\sphinxcode{\sphinxupquote{pydash\_app.dashboard.aggregator.statistics.ExecutionTimePercentileABC}}}}}
\index{field\_name() (pydash\_app.dashboard.aggregator.statistics.MedianExecutionTime method)}

\begin{fulllineitems}
\phantomsection\label{\detokenize{pydash_app.dashboard.aggregator.statistics:pydash_app.dashboard.aggregator.statistics.MedianExecutionTime.field_name}}\pysiglinewithargsret{\sphinxbfcode{\sphinxupquote{field\_name}}}{}{}
\end{fulllineitems}

\index{percentile\_nr() (pydash\_app.dashboard.aggregator.statistics.MedianExecutionTime method)}

\begin{fulllineitems}
\phantomsection\label{\detokenize{pydash_app.dashboard.aggregator.statistics:pydash_app.dashboard.aggregator.statistics.MedianExecutionTime.percentile_nr}}\pysiglinewithargsret{\sphinxbfcode{\sphinxupquote{percentile\_nr}}}{}{}
\end{fulllineitems}


\end{fulllineitems}

\index{NinetiethPercentileExecutionTime (class in pydash\_app.dashboard.aggregator.statistics)}

\begin{fulllineitems}
\phantomsection\label{\detokenize{pydash_app.dashboard.aggregator.statistics:pydash_app.dashboard.aggregator.statistics.NinetiethPercentileExecutionTime}}\pysigline{\sphinxbfcode{\sphinxupquote{class }}\sphinxcode{\sphinxupquote{pydash\_app.dashboard.aggregator.statistics.}}\sphinxbfcode{\sphinxupquote{NinetiethPercentileExecutionTime}}}
Bases: {\hyperref[\detokenize{pydash_app.dashboard.aggregator.statistics:pydash_app.dashboard.aggregator.statistics.ExecutionTimePercentileABC}]{\sphinxcrossref{\sphinxcode{\sphinxupquote{pydash\_app.dashboard.aggregator.statistics.ExecutionTimePercentileABC}}}}}
\index{field\_name() (pydash\_app.dashboard.aggregator.statistics.NinetiethPercentileExecutionTime method)}

\begin{fulllineitems}
\phantomsection\label{\detokenize{pydash_app.dashboard.aggregator.statistics:pydash_app.dashboard.aggregator.statistics.NinetiethPercentileExecutionTime.field_name}}\pysiglinewithargsret{\sphinxbfcode{\sphinxupquote{field\_name}}}{}{}
\end{fulllineitems}

\index{percentile\_nr() (pydash\_app.dashboard.aggregator.statistics.NinetiethPercentileExecutionTime method)}

\begin{fulllineitems}
\phantomsection\label{\detokenize{pydash_app.dashboard.aggregator.statistics:pydash_app.dashboard.aggregator.statistics.NinetiethPercentileExecutionTime.percentile_nr}}\pysiglinewithargsret{\sphinxbfcode{\sphinxupquote{percentile\_nr}}}{}{}
\end{fulllineitems}


\end{fulllineitems}

\index{NinetyNinthPercentileExecutionTime (class in pydash\_app.dashboard.aggregator.statistics)}

\begin{fulllineitems}
\phantomsection\label{\detokenize{pydash_app.dashboard.aggregator.statistics:pydash_app.dashboard.aggregator.statistics.NinetyNinthPercentileExecutionTime}}\pysigline{\sphinxbfcode{\sphinxupquote{class }}\sphinxcode{\sphinxupquote{pydash\_app.dashboard.aggregator.statistics.}}\sphinxbfcode{\sphinxupquote{NinetyNinthPercentileExecutionTime}}}
Bases: {\hyperref[\detokenize{pydash_app.dashboard.aggregator.statistics:pydash_app.dashboard.aggregator.statistics.ExecutionTimePercentileABC}]{\sphinxcrossref{\sphinxcode{\sphinxupquote{pydash\_app.dashboard.aggregator.statistics.ExecutionTimePercentileABC}}}}}
\index{field\_name() (pydash\_app.dashboard.aggregator.statistics.NinetyNinthPercentileExecutionTime method)}

\begin{fulllineitems}
\phantomsection\label{\detokenize{pydash_app.dashboard.aggregator.statistics:pydash_app.dashboard.aggregator.statistics.NinetyNinthPercentileExecutionTime.field_name}}\pysiglinewithargsret{\sphinxbfcode{\sphinxupquote{field\_name}}}{}{}
\end{fulllineitems}

\index{percentile\_nr() (pydash\_app.dashboard.aggregator.statistics.NinetyNinthPercentileExecutionTime method)}

\begin{fulllineitems}
\phantomsection\label{\detokenize{pydash_app.dashboard.aggregator.statistics:pydash_app.dashboard.aggregator.statistics.NinetyNinthPercentileExecutionTime.percentile_nr}}\pysiglinewithargsret{\sphinxbfcode{\sphinxupquote{percentile\_nr}}}{}{}
\end{fulllineitems}


\end{fulllineitems}

\index{SlowestExecutionTime (class in pydash\_app.dashboard.aggregator.statistics)}

\begin{fulllineitems}
\phantomsection\label{\detokenize{pydash_app.dashboard.aggregator.statistics:pydash_app.dashboard.aggregator.statistics.SlowestExecutionTime}}\pysigline{\sphinxbfcode{\sphinxupquote{class }}\sphinxcode{\sphinxupquote{pydash\_app.dashboard.aggregator.statistics.}}\sphinxbfcode{\sphinxupquote{SlowestExecutionTime}}}
Bases: {\hyperref[\detokenize{pydash_app.dashboard.aggregator.statistics:pydash_app.dashboard.aggregator.statistics.ExecutionTimePercentileABC}]{\sphinxcrossref{\sphinxcode{\sphinxupquote{pydash\_app.dashboard.aggregator.statistics.ExecutionTimePercentileABC}}}}}
\index{field\_name() (pydash\_app.dashboard.aggregator.statistics.SlowestExecutionTime method)}

\begin{fulllineitems}
\phantomsection\label{\detokenize{pydash_app.dashboard.aggregator.statistics:pydash_app.dashboard.aggregator.statistics.SlowestExecutionTime.field_name}}\pysiglinewithargsret{\sphinxbfcode{\sphinxupquote{field\_name}}}{}{}
\end{fulllineitems}

\index{percentile\_nr() (pydash\_app.dashboard.aggregator.statistics.SlowestExecutionTime method)}

\begin{fulllineitems}
\phantomsection\label{\detokenize{pydash_app.dashboard.aggregator.statistics:pydash_app.dashboard.aggregator.statistics.SlowestExecutionTime.percentile_nr}}\pysiglinewithargsret{\sphinxbfcode{\sphinxupquote{percentile\_nr}}}{}{}
\end{fulllineitems}


\end{fulllineitems}

\index{SlowestQuartileExecutionTime (class in pydash\_app.dashboard.aggregator.statistics)}

\begin{fulllineitems}
\phantomsection\label{\detokenize{pydash_app.dashboard.aggregator.statistics:pydash_app.dashboard.aggregator.statistics.SlowestQuartileExecutionTime}}\pysigline{\sphinxbfcode{\sphinxupquote{class }}\sphinxcode{\sphinxupquote{pydash\_app.dashboard.aggregator.statistics.}}\sphinxbfcode{\sphinxupquote{SlowestQuartileExecutionTime}}}
Bases: {\hyperref[\detokenize{pydash_app.dashboard.aggregator.statistics:pydash_app.dashboard.aggregator.statistics.ExecutionTimePercentileABC}]{\sphinxcrossref{\sphinxcode{\sphinxupquote{pydash\_app.dashboard.aggregator.statistics.ExecutionTimePercentileABC}}}}}
\index{field\_name() (pydash\_app.dashboard.aggregator.statistics.SlowestQuartileExecutionTime method)}

\begin{fulllineitems}
\phantomsection\label{\detokenize{pydash_app.dashboard.aggregator.statistics:pydash_app.dashboard.aggregator.statistics.SlowestQuartileExecutionTime.field_name}}\pysiglinewithargsret{\sphinxbfcode{\sphinxupquote{field\_name}}}{}{}
\end{fulllineitems}

\index{percentile\_nr() (pydash\_app.dashboard.aggregator.statistics.SlowestQuartileExecutionTime method)}

\begin{fulllineitems}
\phantomsection\label{\detokenize{pydash_app.dashboard.aggregator.statistics:pydash_app.dashboard.aggregator.statistics.SlowestQuartileExecutionTime.percentile_nr}}\pysiglinewithargsret{\sphinxbfcode{\sphinxupquote{percentile\_nr}}}{}{}
\end{fulllineitems}


\end{fulllineitems}

\index{Statistic (class in pydash\_app.dashboard.aggregator.statistics)}

\begin{fulllineitems}
\phantomsection\label{\detokenize{pydash_app.dashboard.aggregator.statistics:pydash_app.dashboard.aggregator.statistics.Statistic}}\pysigline{\sphinxbfcode{\sphinxupquote{class }}\sphinxcode{\sphinxupquote{pydash\_app.dashboard.aggregator.statistics.}}\sphinxbfcode{\sphinxupquote{Statistic}}}
Bases: \sphinxcode{\sphinxupquote{persistent.Persistent}}, \sphinxhref{https://docs.python.org/3/library/abc.html\#abc.ABC}{\sphinxcode{\sphinxupquote{abc.ABC}}}
\index{add\_to\_collection() (pydash\_app.dashboard.aggregator.statistics.Statistic class method)}

\begin{fulllineitems}
\phantomsection\label{\detokenize{pydash_app.dashboard.aggregator.statistics:pydash_app.dashboard.aggregator.statistics.Statistic.add_to_collection}}\pysiglinewithargsret{\sphinxbfcode{\sphinxupquote{classmethod }}\sphinxbfcode{\sphinxupquote{add\_to\_collection}}}{\emph{collection}}{}
cls should only be a class instead of an instance.

\end{fulllineitems}

\index{add\_together() (pydash\_app.dashboard.aggregator.statistics.Statistic method)}

\begin{fulllineitems}
\phantomsection\label{\detokenize{pydash_app.dashboard.aggregator.statistics:pydash_app.dashboard.aggregator.statistics.Statistic.add_together}}\pysiglinewithargsret{\sphinxbfcode{\sphinxupquote{add\_together}}}{\emph{other}, \emph{dependencies\_self}, \emph{dependencies\_other}}{}
Should return a new statistic where the internals of self and other are added together.

\end{fulllineitems}

\index{append() (pydash\_app.dashboard.aggregator.statistics.Statistic method)}

\begin{fulllineitems}
\phantomsection\label{\detokenize{pydash_app.dashboard.aggregator.statistics:pydash_app.dashboard.aggregator.statistics.Statistic.append}}\pysiglinewithargsret{\sphinxbfcode{\sphinxupquote{append}}}{\emph{endpoint\_call}, \emph{dependencies}}{}
\end{fulllineitems}

\index{dependencies (pydash\_app.dashboard.aggregator.statistics.Statistic attribute)}

\begin{fulllineitems}
\phantomsection\label{\detokenize{pydash_app.dashboard.aggregator.statistics:pydash_app.dashboard.aggregator.statistics.Statistic.dependencies}}\pysigline{\sphinxbfcode{\sphinxupquote{dependencies}}\sphinxbfcode{\sphinxupquote{ = {[}{]}}}}
\end{fulllineitems}

\index{empty() (pydash\_app.dashboard.aggregator.statistics.Statistic method)}

\begin{fulllineitems}
\phantomsection\label{\detokenize{pydash_app.dashboard.aggregator.statistics:pydash_app.dashboard.aggregator.statistics.Statistic.empty}}\pysiglinewithargsret{\sphinxbfcode{\sphinxupquote{empty}}}{}{}
\end{fulllineitems}

\index{field\_name() (pydash\_app.dashboard.aggregator.statistics.Statistic class method)}

\begin{fulllineitems}
\phantomsection\label{\detokenize{pydash_app.dashboard.aggregator.statistics:pydash_app.dashboard.aggregator.statistics.Statistic.field_name}}\pysiglinewithargsret{\sphinxbfcode{\sphinxupquote{classmethod }}\sphinxbfcode{\sphinxupquote{field\_name}}}{}{}
\end{fulllineitems}

\index{perform\_append() (pydash\_app.dashboard.aggregator.statistics.Statistic method)}

\begin{fulllineitems}
\phantomsection\label{\detokenize{pydash_app.dashboard.aggregator.statistics:pydash_app.dashboard.aggregator.statistics.Statistic.perform_append}}\pysiglinewithargsret{\sphinxbfcode{\sphinxupquote{perform\_append}}}{\emph{endpoint\_call}, \emph{dependencies}}{}
\end{fulllineitems}

\index{rendered\_value() (pydash\_app.dashboard.aggregator.statistics.Statistic method)}

\begin{fulllineitems}
\phantomsection\label{\detokenize{pydash_app.dashboard.aggregator.statistics:pydash_app.dashboard.aggregator.statistics.Statistic.rendered_value}}\pysiglinewithargsret{\sphinxbfcode{\sphinxupquote{rendered\_value}}}{}{}
\end{fulllineitems}

\index{should\_be\_rendered (pydash\_app.dashboard.aggregator.statistics.Statistic attribute)}

\begin{fulllineitems}
\phantomsection\label{\detokenize{pydash_app.dashboard.aggregator.statistics:pydash_app.dashboard.aggregator.statistics.Statistic.should_be_rendered}}\pysigline{\sphinxbfcode{\sphinxupquote{should\_be\_rendered}}}
Note: implementing subclasses should add the @property decorator.
There was some strange behaviour where without adding the decorator,
subclasses implementing it as \sphinxtitleref{return True} behaved normally, but those implementing it as \sphinxtitleref{return False} still
were treated as if it returned True. Adding the @property decorator fixed it.

\end{fulllineitems}


\end{fulllineitems}

\index{TotalExecutionTime (class in pydash\_app.dashboard.aggregator.statistics)}

\begin{fulllineitems}
\phantomsection\label{\detokenize{pydash_app.dashboard.aggregator.statistics:pydash_app.dashboard.aggregator.statistics.TotalExecutionTime}}\pysigline{\sphinxbfcode{\sphinxupquote{class }}\sphinxcode{\sphinxupquote{pydash\_app.dashboard.aggregator.statistics.}}\sphinxbfcode{\sphinxupquote{TotalExecutionTime}}}
Bases: {\hyperref[\detokenize{pydash_app.dashboard.aggregator.statistics:pydash_app.dashboard.aggregator.statistics.FloatStatisticABC}]{\sphinxcrossref{\sphinxcode{\sphinxupquote{pydash\_app.dashboard.aggregator.statistics.FloatStatisticABC}}}}}
\index{add\_together() (pydash\_app.dashboard.aggregator.statistics.TotalExecutionTime method)}

\begin{fulllineitems}
\phantomsection\label{\detokenize{pydash_app.dashboard.aggregator.statistics:pydash_app.dashboard.aggregator.statistics.TotalExecutionTime.add_together}}\pysiglinewithargsret{\sphinxbfcode{\sphinxupquote{add\_together}}}{\emph{other}, \emph{dependencies\_self}, \emph{dependencies\_other}}{}
Should return a new statistic where the internals of self and other are added together.

\end{fulllineitems}

\index{empty() (pydash\_app.dashboard.aggregator.statistics.TotalExecutionTime method)}

\begin{fulllineitems}
\phantomsection\label{\detokenize{pydash_app.dashboard.aggregator.statistics:pydash_app.dashboard.aggregator.statistics.TotalExecutionTime.empty}}\pysiglinewithargsret{\sphinxbfcode{\sphinxupquote{empty}}}{}{}
\end{fulllineitems}

\index{field\_name() (pydash\_app.dashboard.aggregator.statistics.TotalExecutionTime method)}

\begin{fulllineitems}
\phantomsection\label{\detokenize{pydash_app.dashboard.aggregator.statistics:pydash_app.dashboard.aggregator.statistics.TotalExecutionTime.field_name}}\pysiglinewithargsret{\sphinxbfcode{\sphinxupquote{field\_name}}}{}{}
\end{fulllineitems}

\index{perform\_append() (pydash\_app.dashboard.aggregator.statistics.TotalExecutionTime method)}

\begin{fulllineitems}
\phantomsection\label{\detokenize{pydash_app.dashboard.aggregator.statistics:pydash_app.dashboard.aggregator.statistics.TotalExecutionTime.perform_append}}\pysiglinewithargsret{\sphinxbfcode{\sphinxupquote{perform\_append}}}{\emph{endpoint\_call}, \emph{dependencies}}{}
\end{fulllineitems}

\index{should\_be\_rendered() (pydash\_app.dashboard.aggregator.statistics.TotalExecutionTime method)}

\begin{fulllineitems}
\phantomsection\label{\detokenize{pydash_app.dashboard.aggregator.statistics:pydash_app.dashboard.aggregator.statistics.TotalExecutionTime.should_be_rendered}}\pysiglinewithargsret{\sphinxbfcode{\sphinxupquote{should\_be\_rendered}}}{}{}
Note: implementing subclasses should add the @property decorator.
There was some strange behaviour where without adding the decorator,
subclasses implementing it as \sphinxtitleref{return True} behaved normally, but those implementing it as \sphinxtitleref{return False} still
were treated as if it returned True. Adding the @property decorator fixed it.

\end{fulllineitems}


\end{fulllineitems}

\index{TotalVisits (class in pydash\_app.dashboard.aggregator.statistics)}

\begin{fulllineitems}
\phantomsection\label{\detokenize{pydash_app.dashboard.aggregator.statistics:pydash_app.dashboard.aggregator.statistics.TotalVisits}}\pysigline{\sphinxbfcode{\sphinxupquote{class }}\sphinxcode{\sphinxupquote{pydash\_app.dashboard.aggregator.statistics.}}\sphinxbfcode{\sphinxupquote{TotalVisits}}}
Bases: {\hyperref[\detokenize{pydash_app.dashboard.aggregator.statistics:pydash_app.dashboard.aggregator.statistics.Statistic}]{\sphinxcrossref{\sphinxcode{\sphinxupquote{pydash\_app.dashboard.aggregator.statistics.Statistic}}}}}
\index{add\_together() (pydash\_app.dashboard.aggregator.statistics.TotalVisits method)}

\begin{fulllineitems}
\phantomsection\label{\detokenize{pydash_app.dashboard.aggregator.statistics:pydash_app.dashboard.aggregator.statistics.TotalVisits.add_together}}\pysiglinewithargsret{\sphinxbfcode{\sphinxupquote{add\_together}}}{\emph{other}, \emph{dependencies\_self}, \emph{dependencies\_other}}{}
Should return a new statistic where the internals of self and other are added together.

\end{fulllineitems}

\index{empty() (pydash\_app.dashboard.aggregator.statistics.TotalVisits method)}

\begin{fulllineitems}
\phantomsection\label{\detokenize{pydash_app.dashboard.aggregator.statistics:pydash_app.dashboard.aggregator.statistics.TotalVisits.empty}}\pysiglinewithargsret{\sphinxbfcode{\sphinxupquote{empty}}}{}{}
\end{fulllineitems}

\index{field\_name() (pydash\_app.dashboard.aggregator.statistics.TotalVisits method)}

\begin{fulllineitems}
\phantomsection\label{\detokenize{pydash_app.dashboard.aggregator.statistics:pydash_app.dashboard.aggregator.statistics.TotalVisits.field_name}}\pysiglinewithargsret{\sphinxbfcode{\sphinxupquote{field\_name}}}{}{}
\end{fulllineitems}

\index{perform\_append() (pydash\_app.dashboard.aggregator.statistics.TotalVisits method)}

\begin{fulllineitems}
\phantomsection\label{\detokenize{pydash_app.dashboard.aggregator.statistics:pydash_app.dashboard.aggregator.statistics.TotalVisits.perform_append}}\pysiglinewithargsret{\sphinxbfcode{\sphinxupquote{perform\_append}}}{\emph{endpoint\_call}, \emph{dependencies}}{}
\end{fulllineitems}

\index{should\_be\_rendered() (pydash\_app.dashboard.aggregator.statistics.TotalVisits method)}

\begin{fulllineitems}
\phantomsection\label{\detokenize{pydash_app.dashboard.aggregator.statistics:pydash_app.dashboard.aggregator.statistics.TotalVisits.should_be_rendered}}\pysiglinewithargsret{\sphinxbfcode{\sphinxupquote{should\_be\_rendered}}}{}{}
Note: implementing subclasses should add the @property decorator.
There was some strange behaviour where without adding the decorator,
subclasses implementing it as \sphinxtitleref{return True} behaved normally, but those implementing it as \sphinxtitleref{return False} still
were treated as if it returned True. Adding the @property decorator fixed it.

\end{fulllineitems}


\end{fulllineitems}

\index{UniqueVisitorsAllTime (class in pydash\_app.dashboard.aggregator.statistics)}

\begin{fulllineitems}
\phantomsection\label{\detokenize{pydash_app.dashboard.aggregator.statistics:pydash_app.dashboard.aggregator.statistics.UniqueVisitorsAllTime}}\pysigline{\sphinxbfcode{\sphinxupquote{class }}\sphinxcode{\sphinxupquote{pydash\_app.dashboard.aggregator.statistics.}}\sphinxbfcode{\sphinxupquote{UniqueVisitorsAllTime}}}
Bases: {\hyperref[\detokenize{pydash_app.dashboard.aggregator.statistics:pydash_app.dashboard.aggregator.statistics.Statistic}]{\sphinxcrossref{\sphinxcode{\sphinxupquote{pydash\_app.dashboard.aggregator.statistics.Statistic}}}}}
\index{add\_together() (pydash\_app.dashboard.aggregator.statistics.UniqueVisitorsAllTime method)}

\begin{fulllineitems}
\phantomsection\label{\detokenize{pydash_app.dashboard.aggregator.statistics:pydash_app.dashboard.aggregator.statistics.UniqueVisitorsAllTime.add_together}}\pysiglinewithargsret{\sphinxbfcode{\sphinxupquote{add\_together}}}{\emph{other}, \emph{dependencies\_self}, \emph{dependencies\_other}}{}
Should return a new statistic where the internals of self and other are added together.

\end{fulllineitems}

\index{empty() (pydash\_app.dashboard.aggregator.statistics.UniqueVisitorsAllTime method)}

\begin{fulllineitems}
\phantomsection\label{\detokenize{pydash_app.dashboard.aggregator.statistics:pydash_app.dashboard.aggregator.statistics.UniqueVisitorsAllTime.empty}}\pysiglinewithargsret{\sphinxbfcode{\sphinxupquote{empty}}}{}{}
\end{fulllineitems}

\index{field\_name() (pydash\_app.dashboard.aggregator.statistics.UniqueVisitorsAllTime method)}

\begin{fulllineitems}
\phantomsection\label{\detokenize{pydash_app.dashboard.aggregator.statistics:pydash_app.dashboard.aggregator.statistics.UniqueVisitorsAllTime.field_name}}\pysiglinewithargsret{\sphinxbfcode{\sphinxupquote{field\_name}}}{}{}
\end{fulllineitems}

\index{perform\_append() (pydash\_app.dashboard.aggregator.statistics.UniqueVisitorsAllTime method)}

\begin{fulllineitems}
\phantomsection\label{\detokenize{pydash_app.dashboard.aggregator.statistics:pydash_app.dashboard.aggregator.statistics.UniqueVisitorsAllTime.perform_append}}\pysiglinewithargsret{\sphinxbfcode{\sphinxupquote{perform\_append}}}{\emph{endpoint\_call}, \emph{dependencies}}{}
\end{fulllineitems}

\index{rendered\_value() (pydash\_app.dashboard.aggregator.statistics.UniqueVisitorsAllTime method)}

\begin{fulllineitems}
\phantomsection\label{\detokenize{pydash_app.dashboard.aggregator.statistics:pydash_app.dashboard.aggregator.statistics.UniqueVisitorsAllTime.rendered_value}}\pysiglinewithargsret{\sphinxbfcode{\sphinxupquote{rendered\_value}}}{}{}
\end{fulllineitems}

\index{should\_be\_rendered() (pydash\_app.dashboard.aggregator.statistics.UniqueVisitorsAllTime method)}

\begin{fulllineitems}
\phantomsection\label{\detokenize{pydash_app.dashboard.aggregator.statistics:pydash_app.dashboard.aggregator.statistics.UniqueVisitorsAllTime.should_be_rendered}}\pysiglinewithargsret{\sphinxbfcode{\sphinxupquote{should\_be\_rendered}}}{}{}
Note: implementing subclasses should add the @property decorator.
There was some strange behaviour where without adding the decorator,
subclasses implementing it as \sphinxtitleref{return True} behaved normally, but those implementing it as \sphinxtitleref{return False} still
were treated as if it returned True. Adding the @property decorator fixed it.

\end{fulllineitems}


\end{fulllineitems}

\index{Versions (class in pydash\_app.dashboard.aggregator.statistics)}

\begin{fulllineitems}
\phantomsection\label{\detokenize{pydash_app.dashboard.aggregator.statistics:pydash_app.dashboard.aggregator.statistics.Versions}}\pysigline{\sphinxbfcode{\sphinxupquote{class }}\sphinxcode{\sphinxupquote{pydash\_app.dashboard.aggregator.statistics.}}\sphinxbfcode{\sphinxupquote{Versions}}}
Bases: {\hyperref[\detokenize{pydash_app.dashboard.aggregator.statistics:pydash_app.dashboard.aggregator.statistics.Statistic}]{\sphinxcrossref{\sphinxcode{\sphinxupquote{pydash\_app.dashboard.aggregator.statistics.Statistic}}}}}
\index{add\_together() (pydash\_app.dashboard.aggregator.statistics.Versions method)}

\begin{fulllineitems}
\phantomsection\label{\detokenize{pydash_app.dashboard.aggregator.statistics:pydash_app.dashboard.aggregator.statistics.Versions.add_together}}\pysiglinewithargsret{\sphinxbfcode{\sphinxupquote{add\_together}}}{\emph{other}, \emph{dependencies\_self}, \emph{dependencies\_other}}{}
Should return a new statistic where the internals of self and other are added together.

\end{fulllineitems}

\index{empty() (pydash\_app.dashboard.aggregator.statistics.Versions method)}

\begin{fulllineitems}
\phantomsection\label{\detokenize{pydash_app.dashboard.aggregator.statistics:pydash_app.dashboard.aggregator.statistics.Versions.empty}}\pysiglinewithargsret{\sphinxbfcode{\sphinxupquote{empty}}}{}{}
\end{fulllineitems}

\index{field\_name() (pydash\_app.dashboard.aggregator.statistics.Versions method)}

\begin{fulllineitems}
\phantomsection\label{\detokenize{pydash_app.dashboard.aggregator.statistics:pydash_app.dashboard.aggregator.statistics.Versions.field_name}}\pysiglinewithargsret{\sphinxbfcode{\sphinxupquote{field\_name}}}{}{}
\end{fulllineitems}

\index{perform\_append() (pydash\_app.dashboard.aggregator.statistics.Versions method)}

\begin{fulllineitems}
\phantomsection\label{\detokenize{pydash_app.dashboard.aggregator.statistics:pydash_app.dashboard.aggregator.statistics.Versions.perform_append}}\pysiglinewithargsret{\sphinxbfcode{\sphinxupquote{perform\_append}}}{\emph{endpoint\_call}, \emph{dependencies}}{}
\end{fulllineitems}

\index{rendered\_value() (pydash\_app.dashboard.aggregator.statistics.Versions method)}

\begin{fulllineitems}
\phantomsection\label{\detokenize{pydash_app.dashboard.aggregator.statistics:pydash_app.dashboard.aggregator.statistics.Versions.rendered_value}}\pysiglinewithargsret{\sphinxbfcode{\sphinxupquote{rendered\_value}}}{}{}
\end{fulllineitems}

\index{should\_be\_rendered() (pydash\_app.dashboard.aggregator.statistics.Versions method)}

\begin{fulllineitems}
\phantomsection\label{\detokenize{pydash_app.dashboard.aggregator.statistics:pydash_app.dashboard.aggregator.statistics.Versions.should_be_rendered}}\pysiglinewithargsret{\sphinxbfcode{\sphinxupquote{should\_be\_rendered}}}{}{}
Note: implementing subclasses should add the @property decorator.
There was some strange behaviour where without adding the decorator,
subclasses implementing it as \sphinxtitleref{return True} behaved normally, but those implementing it as \sphinxtitleref{return False} still
were treated as if it returned True. Adding the @property decorator fixed it.

\end{fulllineitems}


\end{fulllineitems}

\index{VisitsPerIP (class in pydash\_app.dashboard.aggregator.statistics)}

\begin{fulllineitems}
\phantomsection\label{\detokenize{pydash_app.dashboard.aggregator.statistics:pydash_app.dashboard.aggregator.statistics.VisitsPerIP}}\pysigline{\sphinxbfcode{\sphinxupquote{class }}\sphinxcode{\sphinxupquote{pydash\_app.dashboard.aggregator.statistics.}}\sphinxbfcode{\sphinxupquote{VisitsPerIP}}}
Bases: {\hyperref[\detokenize{pydash_app.dashboard.aggregator.statistics:pydash_app.dashboard.aggregator.statistics.Statistic}]{\sphinxcrossref{\sphinxcode{\sphinxupquote{pydash\_app.dashboard.aggregator.statistics.Statistic}}}}}
\index{add\_together() (pydash\_app.dashboard.aggregator.statistics.VisitsPerIP method)}

\begin{fulllineitems}
\phantomsection\label{\detokenize{pydash_app.dashboard.aggregator.statistics:pydash_app.dashboard.aggregator.statistics.VisitsPerIP.add_together}}\pysiglinewithargsret{\sphinxbfcode{\sphinxupquote{add\_together}}}{\emph{other}, \emph{dependencies\_self}, \emph{dependencies\_other}}{}
Should return a new statistic where the internals of self and other are added together.

\end{fulllineitems}

\index{empty() (pydash\_app.dashboard.aggregator.statistics.VisitsPerIP method)}

\begin{fulllineitems}
\phantomsection\label{\detokenize{pydash_app.dashboard.aggregator.statistics:pydash_app.dashboard.aggregator.statistics.VisitsPerIP.empty}}\pysiglinewithargsret{\sphinxbfcode{\sphinxupquote{empty}}}{}{}
\end{fulllineitems}

\index{field\_name() (pydash\_app.dashboard.aggregator.statistics.VisitsPerIP method)}

\begin{fulllineitems}
\phantomsection\label{\detokenize{pydash_app.dashboard.aggregator.statistics:pydash_app.dashboard.aggregator.statistics.VisitsPerIP.field_name}}\pysiglinewithargsret{\sphinxbfcode{\sphinxupquote{field\_name}}}{}{}
\end{fulllineitems}

\index{perform\_append() (pydash\_app.dashboard.aggregator.statistics.VisitsPerIP method)}

\begin{fulllineitems}
\phantomsection\label{\detokenize{pydash_app.dashboard.aggregator.statistics:pydash_app.dashboard.aggregator.statistics.VisitsPerIP.perform_append}}\pysiglinewithargsret{\sphinxbfcode{\sphinxupquote{perform\_append}}}{\emph{endpoint\_call}, \emph{dependencies}}{}
\end{fulllineitems}

\index{rendered\_value() (pydash\_app.dashboard.aggregator.statistics.VisitsPerIP method)}

\begin{fulllineitems}
\phantomsection\label{\detokenize{pydash_app.dashboard.aggregator.statistics:pydash_app.dashboard.aggregator.statistics.VisitsPerIP.rendered_value}}\pysiglinewithargsret{\sphinxbfcode{\sphinxupquote{rendered\_value}}}{}{}
\end{fulllineitems}

\index{should\_be\_rendered() (pydash\_app.dashboard.aggregator.statistics.VisitsPerIP method)}

\begin{fulllineitems}
\phantomsection\label{\detokenize{pydash_app.dashboard.aggregator.statistics:pydash_app.dashboard.aggregator.statistics.VisitsPerIP.should_be_rendered}}\pysiglinewithargsret{\sphinxbfcode{\sphinxupquote{should\_be\_rendered}}}{}{}
Note: implementing subclasses should add the @property decorator.
There was some strange behaviour where without adding the decorator,
subclasses implementing it as \sphinxtitleref{return True} behaved normally, but those implementing it as \sphinxtitleref{return False} still
were treated as if it returned True. Adding the @property decorator fixed it.

\end{fulllineitems}


\end{fulllineitems}

\index{date\_dict() (in module pydash\_app.dashboard.aggregator.statistics)}

\begin{fulllineitems}
\phantomsection\label{\detokenize{pydash_app.dashboard.aggregator.statistics:pydash_app.dashboard.aggregator.statistics.date_dict}}\pysiglinewithargsret{\sphinxcode{\sphinxupquote{pydash\_app.dashboard.aggregator.statistics.}}\sphinxbfcode{\sphinxupquote{date\_dict}}}{\emph{dict}}{}
\end{fulllineitems}

\index{reduce\_precision() (in module pydash\_app.dashboard.aggregator.statistics)}

\begin{fulllineitems}
\phantomsection\label{\detokenize{pydash_app.dashboard.aggregator.statistics:pydash_app.dashboard.aggregator.statistics.reduce_precision}}\pysiglinewithargsret{\sphinxcode{\sphinxupquote{pydash\_app.dashboard.aggregator.statistics.}}\sphinxbfcode{\sphinxupquote{reduce\_precision}}}{\emph{value}, \emph{nr\_of\_digits}}{}
Reduces the precision of \sphinxtitleref{value} based on the amount of non-zero digits before the decimal point
and \sphinxtitleref{nr\_of\_digits}.

Examples:
\textgreater{}\textgreater{}\textgreater{} x = 2/3
\textgreater{}\textgreater{}\textgreater{} reduce\_precision(x, 3)
0.67
\textgreater{}\textgreater{}\textgreater{} x = 1234.5678
\textgreater{}\textgreater{}\textgreater{} reduce\_precision(x, 3)
1235

\end{fulllineitems}



\subparagraph{pydash\_app.dashboard.services package}
\label{\detokenize{pydash_app.dashboard.services::doc}}\label{\detokenize{pydash_app.dashboard.services:pydash-app-dashboard-services-package}}\label{\detokenize{pydash_app.dashboard.services:module-pydash_app.dashboard.services}}\index{pydash\_app.dashboard.services (module)}
Contains services for the ‘Dashboard’ concern.

These are things that use or manipulate ‘Dashboard’ entities to perform tasks,
where these tasks are either too complex to put in the Dashboard Entity,
or where these are heavily interacting with outside logic that the business domain entity should not concern itself with directly.
\index{is\_valid\_dashboard() (in module pydash\_app.dashboard.services)}

\begin{fulllineitems}
\phantomsection\label{\detokenize{pydash_app.dashboard.services:pydash_app.dashboard.services.is_valid_dashboard}}\pysiglinewithargsret{\sphinxcode{\sphinxupquote{pydash\_app.dashboard.services.}}\sphinxbfcode{\sphinxupquote{is\_valid\_dashboard}}}{\emph{url}}{}
\end{fulllineitems}



\subparagraph{Submodules}
\label{\detokenize{pydash_app.dashboard.services:submodules}}

\subparagraph{pydash\_app.dashboard.services.fetching module}
\label{\detokenize{pydash_app.dashboard.services.fetching::doc}}\label{\detokenize{pydash_app.dashboard.services.fetching:module-pydash_app.dashboard.services.fetching}}\label{\detokenize{pydash_app.dashboard.services.fetching:pydash-app-dashboard-services-fetching-module}}\index{pydash\_app.dashboard.services.fetching (module)}\index{fetch\_and\_add\_endpoint\_calls() (in module pydash\_app.dashboard.services.fetching)}

\begin{fulllineitems}
\phantomsection\label{\detokenize{pydash_app.dashboard.services.fetching:pydash_app.dashboard.services.fetching.fetch_and_add_endpoint_calls}}\pysiglinewithargsret{\sphinxcode{\sphinxupquote{pydash\_app.dashboard.services.fetching.}}\sphinxbfcode{\sphinxupquote{fetch\_and\_add\_endpoint\_calls}}}{\emph{dashboard}}{}
Retrieve the latest endpoint calls of the given dashboard and add them to it.
:param dashboard: The dashboard for which to update endpoint calls.

\end{fulllineitems}

\index{fetch\_and\_add\_endpoints() (in module pydash\_app.dashboard.services.fetching)}

\begin{fulllineitems}
\phantomsection\label{\detokenize{pydash_app.dashboard.services.fetching:pydash_app.dashboard.services.fetching.fetch_and_add_endpoints}}\pysiglinewithargsret{\sphinxcode{\sphinxupquote{pydash\_app.dashboard.services.fetching.}}\sphinxbfcode{\sphinxupquote{fetch\_and\_add\_endpoints}}}{\emph{dashboard}}{}
For a given dashboard, initialize it with the endpoints it has registered.
Note that this will not add endpoint call data.
:param dashboard: The dashboard to initialize with endpoints.

\end{fulllineitems}

\index{fetch\_and\_add\_historic\_endpoint\_calls() (in module pydash\_app.dashboard.services.fetching)}

\begin{fulllineitems}
\phantomsection\label{\detokenize{pydash_app.dashboard.services.fetching:pydash_app.dashboard.services.fetching.fetch_and_add_historic_endpoint_calls}}\pysiglinewithargsret{\sphinxcode{\sphinxupquote{pydash\_app.dashboard.services.fetching.}}\sphinxbfcode{\sphinxupquote{fetch\_and\_add\_historic\_endpoint\_calls}}}{\emph{dashboard}}{}
For a given dashboard, retrieve all historical endpoint calls and add them to it.
:param dashboard: The dashboard to initialize with historical data.

\end{fulllineitems}

\index{fetch\_and\_update\_historic\_dashboard\_info() (in module pydash\_app.dashboard.services.fetching)}

\begin{fulllineitems}
\phantomsection\label{\detokenize{pydash_app.dashboard.services.fetching:pydash_app.dashboard.services.fetching.fetch_and_update_historic_dashboard_info}}\pysiglinewithargsret{\sphinxcode{\sphinxupquote{pydash\_app.dashboard.services.fetching.}}\sphinxbfcode{\sphinxupquote{fetch\_and\_update\_historic\_dashboard\_info}}}{\emph{dashboard\_id}}{}
Updates the dashboard with the historic EndpointCall information that is fetched from the Dashboard’s remote location.

\end{fulllineitems}

\index{fetch\_and\_update\_new\_dashboard\_info() (in module pydash\_app.dashboard.services.fetching)}

\begin{fulllineitems}
\phantomsection\label{\detokenize{pydash_app.dashboard.services.fetching:pydash_app.dashboard.services.fetching.fetch_and_update_new_dashboard_info}}\pysiglinewithargsret{\sphinxcode{\sphinxupquote{pydash\_app.dashboard.services.fetching.}}\sphinxbfcode{\sphinxupquote{fetch\_and\_update\_new\_dashboard\_info}}}{\emph{dashboard\_id}}{}
Updates the dashboard with the new EndpointCall information that is fetched from the Dashboard’s remote location.

\end{fulllineitems}

\index{schedule\_all\_periodic\_dashboards\_tasks() (in module pydash\_app.dashboard.services.fetching)}

\begin{fulllineitems}
\phantomsection\label{\detokenize{pydash_app.dashboard.services.fetching:pydash_app.dashboard.services.fetching.schedule_all_periodic_dashboards_tasks}}\pysiglinewithargsret{\sphinxcode{\sphinxupquote{pydash\_app.dashboard.services.fetching.}}\sphinxbfcode{\sphinxupquote{schedule\_all\_periodic\_dashboards\_tasks}}}{\emph{interval=datetime.timedelta(0}, \emph{3600)}, \emph{scheduler=\textless{}periodic\_tasks.task\_scheduler.TaskScheduler object\textgreater{}}}{}
Sets up all tasks that should be run periodically for each of the dashboards.
(For now, that is only the EndpointCall fetching task.)

\end{fulllineitems}

\index{schedule\_historic\_dashboard\_fetching() (in module pydash\_app.dashboard.services.fetching)}

\begin{fulllineitems}
\phantomsection\label{\detokenize{pydash_app.dashboard.services.fetching:pydash_app.dashboard.services.fetching.schedule_historic_dashboard_fetching}}\pysiglinewithargsret{\sphinxcode{\sphinxupquote{pydash\_app.dashboard.services.fetching.}}\sphinxbfcode{\sphinxupquote{schedule\_historic\_dashboard\_fetching}}}{\emph{dashboard}, \emph{scheduler=\textless{}periodic\_tasks.task\_scheduler.TaskScheduler object\textgreater{}}}{}
Schedules the fetching of historic EndpointCall information as a background task.
The periodic fetching of new EndpointCall information is scheduled as soon as this task completes.

\end{fulllineitems}

\index{schedule\_periodic\_dashboard\_fetching() (in module pydash\_app.dashboard.services.fetching)}

\begin{fulllineitems}
\phantomsection\label{\detokenize{pydash_app.dashboard.services.fetching:pydash_app.dashboard.services.fetching.schedule_periodic_dashboard_fetching}}\pysiglinewithargsret{\sphinxcode{\sphinxupquote{pydash\_app.dashboard.services.fetching.}}\sphinxbfcode{\sphinxupquote{schedule\_periodic\_dashboard\_fetching}}}{\emph{dashboard}, \emph{interval=datetime.timedelta(0}, \emph{3600)}, \emph{scheduler=\textless{}periodic\_tasks.task\_scheduler.TaskScheduler object\textgreater{}}}{}
Schedules the periodic EndpointCall fetching task for this dashboard.

\end{fulllineitems}



\subparagraph{pydash\_app.dashboard.services.seeding module}
\label{\detokenize{pydash_app.dashboard.services.seeding::doc}}\label{\detokenize{pydash_app.dashboard.services.seeding:pydash-app-dashboard-services-seeding-module}}\label{\detokenize{pydash_app.dashboard.services.seeding:module-pydash_app.dashboard.services.seeding}}\index{pydash\_app.dashboard.services.seeding (module)}
Fills the application with some preliminary dashboards
to make it easier to test code in development and staging environments.
\index{seed() (in module pydash\_app.dashboard.services.seeding)}

\begin{fulllineitems}
\phantomsection\label{\detokenize{pydash_app.dashboard.services.seeding:pydash_app.dashboard.services.seeding.seed}}\pysiglinewithargsret{\sphinxcode{\sphinxupquote{pydash\_app.dashboard.services.seeding.}}\sphinxbfcode{\sphinxupquote{seed}}}{}{}
For each user, stores some preliminary debug dashboards in the datastore,
to be used during development.

\end{fulllineitems}



\paragraph{Submodules}
\label{\detokenize{pydash_app.dashboard:submodules}}

\subparagraph{pydash\_app.dashboard.endpoint module}
\label{\detokenize{pydash_app.dashboard.endpoint::doc}}\label{\detokenize{pydash_app.dashboard.endpoint:module-pydash_app.dashboard.endpoint}}\label{\detokenize{pydash_app.dashboard.endpoint:pydash-app-dashboard-endpoint-module}}\index{pydash\_app.dashboard.endpoint (module)}\index{Endpoint (class in pydash\_app.dashboard.endpoint)}

\begin{fulllineitems}
\phantomsection\label{\detokenize{pydash_app.dashboard.endpoint:pydash_app.dashboard.endpoint.Endpoint}}\pysiglinewithargsret{\sphinxbfcode{\sphinxupquote{class }}\sphinxcode{\sphinxupquote{pydash\_app.dashboard.endpoint.}}\sphinxbfcode{\sphinxupquote{Endpoint}}}{\emph{name}, \emph{is\_monitored}}{}
Bases: \sphinxcode{\sphinxupquote{persistent.Persistent}}

The Endpoint entity knows about:
- Its own properties
- The functionalities for Endpoint interactions with information from elsewhere.

It does not contain information on how to persistently store/load an endpoint,
as currently endpoints only exist in combination with dashboard objects.
If endpoints were to exist on their own, the \sphinxtitleref{endpoint\_repository} would handle their persistence.
\index{add\_endpoint\_call() (pydash\_app.dashboard.endpoint.Endpoint method)}

\begin{fulllineitems}
\phantomsection\label{\detokenize{pydash_app.dashboard.endpoint:pydash_app.dashboard.endpoint.Endpoint.add_endpoint_call}}\pysiglinewithargsret{\sphinxbfcode{\sphinxupquote{add\_endpoint\_call}}}{\emph{call}}{}
Adds an EndpointCall to its internal collection of endpoint calls.
:param call: The endpoint call to add.

\end{fulllineitems}

\index{aggregated\_data() (pydash\_app.dashboard.endpoint.Endpoint method)}

\begin{fulllineitems}
\phantomsection\label{\detokenize{pydash_app.dashboard.endpoint:pydash_app.dashboard.endpoint.Endpoint.aggregated_data}}\pysiglinewithargsret{\sphinxbfcode{\sphinxupquote{aggregated\_data}}}{\emph{filters=\{\}}}{}
Returns aggregated data on this endpoint.
:param filters: A dictionary containing property\_name-value pairs to filter on. The keys are assumed to be strings.
\begin{quote}

This is in the gist of \sphinxtitleref{\{‘day’:‘2018-05-20’, ‘ip’:‘127.0.0.1’\}}
Defaults to an empty dictionary.
\begin{description}
\item[{The currently allowed filter\_names are:}] \leavevmode\begin{itemize}
\item {} 
Time:
* ‘year’   - e.g. ‘2018’
* ‘month’  - e.g. ‘2018-05’
* ‘week’   - e.g. ‘2018-W17’
* ‘day’    - e.g. ‘2018-05-20’
* ‘hour’   - e.g. ‘2018-05-20T20’
* ‘minute’ - e.g. ‘2018-05-20T20-10’

\end{itemize}

Note that for Time filter-values, the formatting is crucial.
\begin{itemize}
\item {} 
Version:
* ‘version’ - e.g. ‘1.0.1’

\item {} 
IP:
* ‘ip’ - e.g. ‘127.0.0.1’

\item {} 
Group-by:
* ‘group\_by’ - e.g. ‘None’

\end{itemize}

\end{description}
\end{quote}
\begin{quote}\begin{description}
\item[{Returns}] \leavevmode
A dict containing aggregated data points.

\end{description}\end{quote}

\end{fulllineitems}

\index{aggregated\_data\_daterange() (pydash\_app.dashboard.endpoint.Endpoint method)}

\begin{fulllineitems}
\phantomsection\label{\detokenize{pydash_app.dashboard.endpoint:pydash_app.dashboard.endpoint.Endpoint.aggregated_data_daterange}}\pysiglinewithargsret{\sphinxbfcode{\sphinxupquote{aggregated\_data\_daterange}}}{\emph{start\_date}, \emph{end\_date}, \emph{granularity}, \emph{filters=\{\}}}{}
Returns the aggregated data on this endpoint over the specified daterange.
:param start\_date: A datetime object that is treated as the inclusive lower bound of the daterange.
:param end\_date: A datetime object that is treated as the inclusive upper bound of the daterange.
:param granularity: A string denoting the granularity of the daterange.
:param filters: A dictionary containing property\_name-value pairs to filter on. The keys are assumed to be strings.
\begin{quote}

This is in the gist of \sphinxtitleref{\{‘day’:‘2018-05-20’, ‘ip’:‘127.0.0.1’\}}
Defaults to an empty dictionary.
\begin{description}
\item[{The currently allowed filter\_names are:}] \leavevmode\begin{itemize}
\item {} 
Version:
* ‘version’ - e.g. ‘1.0.1’

\item {} 
IP:
* ‘ip’ - e.g. ‘127.0.0.1’

\item {} 
Group-by:
* ‘group\_by’ - e.g. ‘None’

\end{itemize}

Note that, contrary to \sphinxtitleref{aggregated\_data} method, Time based filters are not allowed.

\end{description}
\end{quote}
\begin{quote}\begin{description}
\item[{Returns}] \leavevmode
A dictionary with all aggregated statistics and their values.

\end{description}\end{quote}

\end{fulllineitems}

\index{get\_id() (pydash\_app.dashboard.endpoint.Endpoint method)}

\begin{fulllineitems}
\phantomsection\label{\detokenize{pydash_app.dashboard.endpoint:pydash_app.dashboard.endpoint.Endpoint.get_id}}\pysiglinewithargsret{\sphinxbfcode{\sphinxupquote{get\_id}}}{}{}
\end{fulllineitems}

\index{remove\_endpoint\_call() (pydash\_app.dashboard.endpoint.Endpoint method)}

\begin{fulllineitems}
\phantomsection\label{\detokenize{pydash_app.dashboard.endpoint:pydash_app.dashboard.endpoint.Endpoint.remove_endpoint_call}}\pysiglinewithargsret{\sphinxbfcode{\sphinxupquote{remove\_endpoint\_call}}}{\emph{call}}{}
Removes an EndpointCall from this endpoint’s internal collection of endpoint calls.
Raises a ValueError if no such call exists.
Note: does not remove it from its aggregated dataset yet.
:param call: The endpoint call to remove.

\end{fulllineitems}

\index{set\_monitored() (pydash\_app.dashboard.endpoint.Endpoint method)}

\begin{fulllineitems}
\phantomsection\label{\detokenize{pydash_app.dashboard.endpoint:pydash_app.dashboard.endpoint.Endpoint.set_monitored}}\pysiglinewithargsret{\sphinxbfcode{\sphinxupquote{set\_monitored}}}{\emph{is\_monitored}}{}
\end{fulllineitems}

\index{statistic() (pydash\_app.dashboard.endpoint.Endpoint method)}

\begin{fulllineitems}
\phantomsection\label{\detokenize{pydash_app.dashboard.endpoint:pydash_app.dashboard.endpoint.Endpoint.statistic}}\pysiglinewithargsret{\sphinxbfcode{\sphinxupquote{statistic}}}{\emph{statistic}, \emph{filters=\{\}}}{}
\end{fulllineitems}

\index{statistic\_per\_timeslice() (pydash\_app.dashboard.endpoint.Endpoint method)}

\begin{fulllineitems}
\phantomsection\label{\detokenize{pydash_app.dashboard.endpoint:pydash_app.dashboard.endpoint.Endpoint.statistic_per_timeslice}}\pysiglinewithargsret{\sphinxbfcode{\sphinxupquote{statistic\_per\_timeslice}}}{\emph{statistic}, \emph{timeslice}, \emph{start\_datetime}, \emph{end\_datetime}, \emph{filters=\{\}}}{}
\end{fulllineitems}


\end{fulllineitems}



\subparagraph{pydash\_app.dashboard.endpoint\_call module}
\label{\detokenize{pydash_app.dashboard.endpoint_call::doc}}\label{\detokenize{pydash_app.dashboard.endpoint_call:module-pydash_app.dashboard.endpoint_call}}\label{\detokenize{pydash_app.dashboard.endpoint_call:pydash-app-dashboard-endpoint-call-module}}\index{pydash\_app.dashboard.endpoint\_call (module)}\index{EndpointCall (class in pydash\_app.dashboard.endpoint\_call)}

\begin{fulllineitems}
\phantomsection\label{\detokenize{pydash_app.dashboard.endpoint_call:pydash_app.dashboard.endpoint_call.EndpointCall}}\pysiglinewithargsret{\sphinxbfcode{\sphinxupquote{class }}\sphinxcode{\sphinxupquote{pydash\_app.dashboard.endpoint\_call.}}\sphinxbfcode{\sphinxupquote{EndpointCall}}}{\emph{endpoint}, \emph{execution\_time}, \emph{time}, \emph{version}, \emph{group\_by}, \emph{ip}}{}
Bases: \sphinxcode{\sphinxupquote{persistent.Persistent}}

An EndpointCall entity only serves to store JSON data pulled from the external dashboards.

As with the other entity classes, it does not concern itself with the implementation of its persistence,
as it doesn’t exist on its own.
If this were the case, the \sphinxtitleref{endpointcall\_repository} would handle this concern.

\fvset{hllines={, ,}}%
\begin{sphinxVerbatim}[commandchars=\\\{\}]
\PYG{g+gp}{\PYGZgt{}\PYGZgt{}\PYGZgt{} }\PYG{n}{endpoint\PYGZus{}call} \PYG{o}{=} \PYG{n}{EndpointCall}\PYG{p}{(}\PYG{l+s+s2}{\PYGZdq{}}\PYG{l+s+s2}{foo}\PYG{l+s+s2}{\PYGZdq{}}\PYG{p}{,} \PYG{l+m+mf}{0.5}\PYG{p}{,} \PYG{n}{datetime}\PYG{o}{.}\PYG{n}{strptime}\PYG{p}{(}\PYG{l+s+s2}{\PYGZdq{}}\PYG{l+s+s2}{2018\PYGZhy{}04\PYGZhy{}25 15:29:23}\PYG{l+s+s2}{\PYGZdq{}}\PYG{p}{,} \PYG{l+s+s2}{\PYGZdq{}}\PYG{l+s+s2}{\PYGZpc{}}\PYG{l+s+s2}{Y\PYGZhy{}}\PYG{l+s+s2}{\PYGZpc{}}\PYG{l+s+s2}{m\PYGZhy{}}\PYG{l+s+si}{\PYGZpc{}d}\PYG{l+s+s2}{ }\PYG{l+s+s2}{\PYGZpc{}}\PYG{l+s+s2}{H:}\PYG{l+s+s2}{\PYGZpc{}}\PYG{l+s+s2}{M:}\PYG{l+s+s2}{\PYGZpc{}}\PYG{l+s+s2}{S}\PYG{l+s+s2}{\PYGZdq{}}\PYG{p}{)}\PYG{p}{,} \PYG{l+s+s2}{\PYGZdq{}}\PYG{l+s+s2}{0.1}\PYG{l+s+s2}{\PYGZdq{}}\PYG{p}{,} \PYG{l+s+s2}{\PYGZdq{}}\PYG{l+s+s2}{None}\PYG{l+s+s2}{\PYGZdq{}}\PYG{p}{,} \PYG{l+s+s2}{\PYGZdq{}}\PYG{l+s+s2}{127.0.0.1}\PYG{l+s+s2}{\PYGZdq{}}\PYG{p}{)}
\PYG{g+gp}{\PYGZgt{}\PYGZgt{}\PYGZgt{} }\PYG{n}{endpoint\PYGZus{}call}\PYG{o}{.}\PYG{n}{as\PYGZus{}dict}\PYG{p}{(}\PYG{p}{)}
\PYG{g+go}{\PYGZob{}\PYGZsq{}endpoint\PYGZsq{}: \PYGZsq{}foo\PYGZsq{}, \PYGZsq{}execution\PYGZus{}time\PYGZsq{}: 0.5, \PYGZsq{}time\PYGZsq{}: datetime.datetime(2018, 4, 25, 15, 29, 23), \PYGZsq{}version\PYGZsq{}: \PYGZsq{}0.1\PYGZsq{}, \PYGZsq{}group\PYGZus{}by\PYGZsq{}: \PYGZsq{}None\PYGZsq{}, \PYGZsq{}ip\PYGZsq{}: \PYGZsq{}127.0.0.1\PYGZsq{}\PYGZcb{}}
\end{sphinxVerbatim}
\index{as\_dict() (pydash\_app.dashboard.endpoint\_call.EndpointCall method)}

\begin{fulllineitems}
\phantomsection\label{\detokenize{pydash_app.dashboard.endpoint_call:pydash_app.dashboard.endpoint_call.EndpointCall.as_dict}}\pysiglinewithargsret{\sphinxbfcode{\sphinxupquote{as\_dict}}}{}{}
returns a dict containing the data of the EndpointCall

\end{fulllineitems}


\end{fulllineitems}



\subparagraph{pydash\_app.dashboard.entity module}
\label{\detokenize{pydash_app.dashboard.entity::doc}}\label{\detokenize{pydash_app.dashboard.entity:module-pydash_app.dashboard.entity}}\label{\detokenize{pydash_app.dashboard.entity:pydash-app-dashboard-entity-module}}\index{pydash\_app.dashboard.entity (module)}
Involved usage example:

\fvset{hllines={, ,}}%
\begin{sphinxVerbatim}[commandchars=\\\{\}]
\PYG{g+gp}{\PYGZgt{}\PYGZgt{}\PYGZgt{} }\PYG{k+kn}{from} \PYG{n+nn}{pydash\PYGZus{}app}\PYG{n+nn}{.}\PYG{n+nn}{dashboard}\PYG{n+nn}{.}\PYG{n+nn}{entity} \PYG{k}{import} \PYG{n}{Dashboard}
\PYG{g+gp}{\PYGZgt{}\PYGZgt{}\PYGZgt{} }\PYG{k+kn}{from} \PYG{n+nn}{pydash\PYGZus{}app}\PYG{n+nn}{.}\PYG{n+nn}{user}\PYG{n+nn}{.}\PYG{n+nn}{entity} \PYG{k}{import} \PYG{n}{User}
\PYG{g+gp}{\PYGZgt{}\PYGZgt{}\PYGZgt{} }\PYG{k+kn}{from} \PYG{n+nn}{pydash\PYGZus{}app}\PYG{n+nn}{.}\PYG{n+nn}{dashboard}\PYG{n+nn}{.}\PYG{n+nn}{endpoint} \PYG{k}{import} \PYG{n}{Endpoint}
\PYG{g+gp}{\PYGZgt{}\PYGZgt{}\PYGZgt{} }\PYG{k+kn}{from} \PYG{n+nn}{pydash\PYGZus{}app}\PYG{n+nn}{.}\PYG{n+nn}{dashboard}\PYG{n+nn}{.}\PYG{n+nn}{endpoint\PYGZus{}call} \PYG{k}{import} \PYG{n}{EndpointCall}
\PYG{g+gp}{\PYGZgt{}\PYGZgt{}\PYGZgt{} }\PYG{k+kn}{import} \PYG{n+nn}{uuid}
\PYG{g+gp}{\PYGZgt{}\PYGZgt{}\PYGZgt{} }\PYG{k+kn}{from} \PYG{n+nn}{datetime} \PYG{k}{import} \PYG{n}{datetime}\PYG{p}{,} \PYG{n}{timedelta}
\PYG{g+gp}{\PYGZgt{}\PYGZgt{}\PYGZgt{} }\PYG{n}{user} \PYG{o}{=} \PYG{n}{User}\PYG{p}{(}\PYG{l+s+s2}{\PYGZdq{}}\PYG{l+s+s2}{Gandalf}\PYG{l+s+s2}{\PYGZdq{}}\PYG{p}{,} \PYG{l+s+s2}{\PYGZdq{}}\PYG{l+s+s2}{pass}\PYG{l+s+s2}{\PYGZdq{}}\PYG{p}{,} \PYG{l+s+s1}{\PYGZsq{}}\PYG{l+s+s1}{some@email.com}\PYG{l+s+s1}{\PYGZsq{}}\PYG{p}{)}
\PYG{g+gp}{\PYGZgt{}\PYGZgt{}\PYGZgt{} }\PYG{n}{d} \PYG{o}{=} \PYG{n}{Dashboard}\PYG{p}{(}\PYG{l+s+s2}{\PYGZdq{}}\PYG{l+s+s2}{http://foo.io}\PYG{l+s+s2}{\PYGZdq{}}\PYG{p}{,} \PYG{n+nb}{str}\PYG{p}{(}\PYG{n}{uuid}\PYG{o}{.}\PYG{n}{uuid4}\PYG{p}{(}\PYG{p}{)}\PYG{p}{)}\PYG{p}{,} \PYG{n+nb}{str}\PYG{p}{(}\PYG{n}{user}\PYG{o}{.}\PYG{n}{id}\PYG{p}{)}\PYG{p}{)}
\PYG{g+gp}{\PYGZgt{}\PYGZgt{}\PYGZgt{} }\PYG{n}{e1} \PYG{o}{=} \PYG{n}{Endpoint}\PYG{p}{(}\PYG{l+s+s2}{\PYGZdq{}}\PYG{l+s+s2}{foo}\PYG{l+s+s2}{\PYGZdq{}}\PYG{p}{,} \PYG{k+kc}{True}\PYG{p}{)}
\PYG{g+gp}{\PYGZgt{}\PYGZgt{}\PYGZgt{} }\PYG{n}{e2} \PYG{o}{=} \PYG{n}{Endpoint}\PYG{p}{(}\PYG{l+s+s2}{\PYGZdq{}}\PYG{l+s+s2}{bar}\PYG{l+s+s2}{\PYGZdq{}}\PYG{p}{,} \PYG{k+kc}{True}\PYG{p}{)}
\PYG{g+gp}{\PYGZgt{}\PYGZgt{}\PYGZgt{} }\PYG{n}{d}\PYG{o}{.}\PYG{n}{add\PYGZus{}endpoint}\PYG{p}{(}\PYG{n}{e1}\PYG{p}{)}
\PYG{g+gp}{\PYGZgt{}\PYGZgt{}\PYGZgt{} }\PYG{n}{d}\PYG{o}{.}\PYG{n}{add\PYGZus{}endpoint}\PYG{p}{(}\PYG{n}{e2}\PYG{p}{)}
\PYG{g+gp}{\PYGZgt{}\PYGZgt{}\PYGZgt{} }\PYG{n}{ec1} \PYG{o}{=} \PYG{n}{EndpointCall}\PYG{p}{(}\PYG{l+s+s2}{\PYGZdq{}}\PYG{l+s+s2}{foo}\PYG{l+s+s2}{\PYGZdq{}}\PYG{p}{,} \PYG{l+m+mf}{0.5}\PYG{p}{,} \PYG{n}{datetime}\PYG{o}{.}\PYG{n}{strptime}\PYG{p}{(}\PYG{l+s+s2}{\PYGZdq{}}\PYG{l+s+s2}{2018\PYGZhy{}04\PYGZhy{}25 15:29:23}\PYG{l+s+s2}{\PYGZdq{}}\PYG{p}{,} \PYG{l+s+s2}{\PYGZdq{}}\PYG{l+s+s2}{\PYGZpc{}}\PYG{l+s+s2}{Y\PYGZhy{}}\PYG{l+s+s2}{\PYGZpc{}}\PYG{l+s+s2}{m\PYGZhy{}}\PYG{l+s+si}{\PYGZpc{}d}\PYG{l+s+s2}{ }\PYG{l+s+s2}{\PYGZpc{}}\PYG{l+s+s2}{H:}\PYG{l+s+s2}{\PYGZpc{}}\PYG{l+s+s2}{M:}\PYG{l+s+s2}{\PYGZpc{}}\PYG{l+s+s2}{S}\PYG{l+s+s2}{\PYGZdq{}}\PYG{p}{)}\PYG{p}{,} \PYG{l+s+s2}{\PYGZdq{}}\PYG{l+s+s2}{0.1}\PYG{l+s+s2}{\PYGZdq{}}\PYG{p}{,} \PYG{l+s+s2}{\PYGZdq{}}\PYG{l+s+s2}{None}\PYG{l+s+s2}{\PYGZdq{}}\PYG{p}{,} \PYG{l+s+s2}{\PYGZdq{}}\PYG{l+s+s2}{127.0.0.1}\PYG{l+s+s2}{\PYGZdq{}}\PYG{p}{)}
\PYG{g+gp}{\PYGZgt{}\PYGZgt{}\PYGZgt{} }\PYG{n}{ec2} \PYG{o}{=} \PYG{n}{EndpointCall}\PYG{p}{(}\PYG{l+s+s2}{\PYGZdq{}}\PYG{l+s+s2}{foo}\PYG{l+s+s2}{\PYGZdq{}}\PYG{p}{,} \PYG{l+m+mf}{0.1}\PYG{p}{,} \PYG{n}{datetime}\PYG{o}{.}\PYG{n}{strptime}\PYG{p}{(}\PYG{l+s+s2}{\PYGZdq{}}\PYG{l+s+s2}{2018\PYGZhy{}04\PYGZhy{}25 15:29:23}\PYG{l+s+s2}{\PYGZdq{}}\PYG{p}{,} \PYG{l+s+s2}{\PYGZdq{}}\PYG{l+s+s2}{\PYGZpc{}}\PYG{l+s+s2}{Y\PYGZhy{}}\PYG{l+s+s2}{\PYGZpc{}}\PYG{l+s+s2}{m\PYGZhy{}}\PYG{l+s+si}{\PYGZpc{}d}\PYG{l+s+s2}{ }\PYG{l+s+s2}{\PYGZpc{}}\PYG{l+s+s2}{H:}\PYG{l+s+s2}{\PYGZpc{}}\PYG{l+s+s2}{M:}\PYG{l+s+s2}{\PYGZpc{}}\PYG{l+s+s2}{S}\PYG{l+s+s2}{\PYGZdq{}}\PYG{p}{)}\PYG{p}{,} \PYG{l+s+s2}{\PYGZdq{}}\PYG{l+s+s2}{0.1}\PYG{l+s+s2}{\PYGZdq{}}\PYG{p}{,} \PYG{l+s+s2}{\PYGZdq{}}\PYG{l+s+s2}{None}\PYG{l+s+s2}{\PYGZdq{}}\PYG{p}{,} \PYG{l+s+s2}{\PYGZdq{}}\PYG{l+s+s2}{127.0.0.2}\PYG{l+s+s2}{\PYGZdq{}}\PYG{p}{)}
\PYG{g+gp}{\PYGZgt{}\PYGZgt{}\PYGZgt{} }\PYG{n}{ec3} \PYG{o}{=} \PYG{n}{EndpointCall}\PYG{p}{(}\PYG{l+s+s2}{\PYGZdq{}}\PYG{l+s+s2}{bar}\PYG{l+s+s2}{\PYGZdq{}}\PYG{p}{,} \PYG{l+m+mf}{0.2}\PYG{p}{,} \PYG{n}{datetime}\PYG{o}{.}\PYG{n}{strptime}\PYG{p}{(}\PYG{l+s+s2}{\PYGZdq{}}\PYG{l+s+s2}{2018\PYGZhy{}04\PYGZhy{}25 15:29:23}\PYG{l+s+s2}{\PYGZdq{}}\PYG{p}{,} \PYG{l+s+s2}{\PYGZdq{}}\PYG{l+s+s2}{\PYGZpc{}}\PYG{l+s+s2}{Y\PYGZhy{}}\PYG{l+s+s2}{\PYGZpc{}}\PYG{l+s+s2}{m\PYGZhy{}}\PYG{l+s+si}{\PYGZpc{}d}\PYG{l+s+s2}{ }\PYG{l+s+s2}{\PYGZpc{}}\PYG{l+s+s2}{H:}\PYG{l+s+s2}{\PYGZpc{}}\PYG{l+s+s2}{M:}\PYG{l+s+s2}{\PYGZpc{}}\PYG{l+s+s2}{S}\PYG{l+s+s2}{\PYGZdq{}}\PYG{p}{)}\PYG{p}{,} \PYG{l+s+s2}{\PYGZdq{}}\PYG{l+s+s2}{0.1}\PYG{l+s+s2}{\PYGZdq{}}\PYG{p}{,} \PYG{l+s+s2}{\PYGZdq{}}\PYG{l+s+s2}{None}\PYG{l+s+s2}{\PYGZdq{}}\PYG{p}{,} \PYG{l+s+s2}{\PYGZdq{}}\PYG{l+s+s2}{127.0.0.1}\PYG{l+s+s2}{\PYGZdq{}}\PYG{p}{)}
\PYG{g+gp}{\PYGZgt{}\PYGZgt{}\PYGZgt{} }\PYG{n}{ec4} \PYG{o}{=} \PYG{n}{EndpointCall}\PYG{p}{(}\PYG{l+s+s2}{\PYGZdq{}}\PYG{l+s+s2}{bar}\PYG{l+s+s2}{\PYGZdq{}}\PYG{p}{,} \PYG{l+m+mf}{0.2}\PYG{p}{,} \PYG{n}{datetime}\PYG{o}{.}\PYG{n}{strptime}\PYG{p}{(}\PYG{l+s+s2}{\PYGZdq{}}\PYG{l+s+s2}{2018\PYGZhy{}04\PYGZhy{}25 15:29:23}\PYG{l+s+s2}{\PYGZdq{}}\PYG{p}{,} \PYG{l+s+s2}{\PYGZdq{}}\PYG{l+s+s2}{\PYGZpc{}}\PYG{l+s+s2}{Y\PYGZhy{}}\PYG{l+s+s2}{\PYGZpc{}}\PYG{l+s+s2}{m\PYGZhy{}}\PYG{l+s+si}{\PYGZpc{}d}\PYG{l+s+s2}{ }\PYG{l+s+s2}{\PYGZpc{}}\PYG{l+s+s2}{H:}\PYG{l+s+s2}{\PYGZpc{}}\PYG{l+s+s2}{M:}\PYG{l+s+s2}{\PYGZpc{}}\PYG{l+s+s2}{S}\PYG{l+s+s2}{\PYGZdq{}}\PYG{p}{)} \PYG{o}{\PYGZhy{}} \PYG{n}{timedelta}\PYG{p}{(}\PYG{n}{days}\PYG{o}{=}\PYG{l+m+mi}{1}\PYG{p}{)}\PYG{p}{,} \PYG{l+s+s2}{\PYGZdq{}}\PYG{l+s+s2}{0.1}\PYG{l+s+s2}{\PYGZdq{}}\PYG{p}{,} \PYG{l+s+s2}{\PYGZdq{}}\PYG{l+s+s2}{None}\PYG{l+s+s2}{\PYGZdq{}}\PYG{p}{,} \PYG{l+s+s2}{\PYGZdq{}}\PYG{l+s+s2}{127.0.0.1}\PYG{l+s+s2}{\PYGZdq{}}\PYG{p}{)}
\PYG{g+gp}{\PYGZgt{}\PYGZgt{}\PYGZgt{} }\PYG{n}{ec5} \PYG{o}{=} \PYG{n}{EndpointCall}\PYG{p}{(}\PYG{l+s+s2}{\PYGZdq{}}\PYG{l+s+s2}{bar}\PYG{l+s+s2}{\PYGZdq{}}\PYG{p}{,} \PYG{l+m+mf}{0.2}\PYG{p}{,} \PYG{n}{datetime}\PYG{o}{.}\PYG{n}{strptime}\PYG{p}{(}\PYG{l+s+s2}{\PYGZdq{}}\PYG{l+s+s2}{2018\PYGZhy{}04\PYGZhy{}25 15:29:23}\PYG{l+s+s2}{\PYGZdq{}}\PYG{p}{,} \PYG{l+s+s2}{\PYGZdq{}}\PYG{l+s+s2}{\PYGZpc{}}\PYG{l+s+s2}{Y\PYGZhy{}}\PYG{l+s+s2}{\PYGZpc{}}\PYG{l+s+s2}{m\PYGZhy{}}\PYG{l+s+si}{\PYGZpc{}d}\PYG{l+s+s2}{ }\PYG{l+s+s2}{\PYGZpc{}}\PYG{l+s+s2}{H:}\PYG{l+s+s2}{\PYGZpc{}}\PYG{l+s+s2}{M:}\PYG{l+s+s2}{\PYGZpc{}}\PYG{l+s+s2}{S}\PYG{l+s+s2}{\PYGZdq{}}\PYG{p}{)} \PYG{o}{\PYGZhy{}} \PYG{n}{timedelta}\PYG{p}{(}\PYG{n}{days}\PYG{o}{=}\PYG{l+m+mi}{2}\PYG{p}{)}\PYG{p}{,} \PYG{l+s+s2}{\PYGZdq{}}\PYG{l+s+s2}{0.1}\PYG{l+s+s2}{\PYGZdq{}}\PYG{p}{,} \PYG{l+s+s2}{\PYGZdq{}}\PYG{l+s+s2}{None}\PYG{l+s+s2}{\PYGZdq{}}\PYG{p}{,} \PYG{l+s+s2}{\PYGZdq{}}\PYG{l+s+s2}{127.0.0.1}\PYG{l+s+s2}{\PYGZdq{}}\PYG{p}{)}
\PYG{g+gp}{\PYGZgt{}\PYGZgt{}\PYGZgt{} }\PYG{n}{d}\PYG{o}{.}\PYG{n}{add\PYGZus{}endpoint\PYGZus{}call}\PYG{p}{(}\PYG{n}{ec1}\PYG{p}{)}
\PYG{g+gp}{\PYGZgt{}\PYGZgt{}\PYGZgt{} }\PYG{n}{d}\PYG{o}{.}\PYG{n}{add\PYGZus{}endpoint\PYGZus{}call}\PYG{p}{(}\PYG{n}{ec2}\PYG{p}{)}
\PYG{g+gp}{\PYGZgt{}\PYGZgt{}\PYGZgt{} }\PYG{n}{d}\PYG{o}{.}\PYG{n}{add\PYGZus{}endpoint\PYGZus{}call}\PYG{p}{(}\PYG{n}{ec3}\PYG{p}{)}
\PYG{g+gp}{\PYGZgt{}\PYGZgt{}\PYGZgt{} }\PYG{n}{d}\PYG{o}{.}\PYG{n}{add\PYGZus{}endpoint\PYGZus{}call}\PYG{p}{(}\PYG{n}{ec4}\PYG{p}{)}
\PYG{g+gp}{\PYGZgt{}\PYGZgt{}\PYGZgt{} }\PYG{n}{d}\PYG{o}{.}\PYG{n}{add\PYGZus{}endpoint\PYGZus{}call}\PYG{p}{(}\PYG{n}{ec5}\PYG{p}{)}
\PYG{g+gp}{\PYGZgt{}\PYGZgt{}\PYGZgt{} }\PYG{n}{d}\PYG{o}{.}\PYG{n}{aggregated\PYGZus{}data}\PYG{p}{(}\PYG{p}{)}
\PYG{g+go}{\PYGZob{}\PYGZsq{}total\PYGZus{}visits\PYGZsq{}: 5, \PYGZsq{}total\PYGZus{}execution\PYGZus{}time\PYGZsq{}: 1.2, \PYGZsq{}average\PYGZus{}execution\PYGZus{}time\PYGZsq{}: 0.24, \PYGZsq{}visits\PYGZus{}per\PYGZus{}ip\PYGZsq{}: \PYGZob{}\PYGZsq{}127.0.0.1\PYGZsq{}: 4, \PYGZsq{}127.0.0.2\PYGZsq{}: 1\PYGZcb{}, \PYGZsq{}unique\PYGZus{}visitors\PYGZsq{}: 2, \PYGZsq{}fastest\PYGZus{}measured\PYGZus{}execution\PYGZus{}time\PYGZsq{}: 0.1, \PYGZsq{}fastest\PYGZus{}quartile\PYGZus{}execution\PYGZus{}time\PYGZsq{}: 0.14, \PYGZsq{}median\PYGZus{}execution\PYGZus{}time\PYGZsq{}: 0.2, \PYGZsq{}slowest\PYGZus{}quartile\PYGZus{}execution\PYGZus{}time\PYGZsq{}: 0.39, \PYGZsq{}ninetieth\PYGZus{}percentile\PYGZus{}execution\PYGZus{}time\PYGZsq{}: 0.5, \PYGZsq{}ninety\PYGZhy{}ninth\PYGZus{}percentile\PYGZus{}execution\PYGZus{}time\PYGZsq{}: 0.5, \PYGZsq{}slowest\PYGZus{}measured\PYGZus{}execution\PYGZus{}time\PYGZsq{}: 0.5, \PYGZsq{}versions\PYGZsq{}: [\PYGZsq{}0.1\PYGZsq{}]\PYGZcb{}}
\PYG{g+gp}{\PYGZgt{}\PYGZgt{}\PYGZgt{} }\PYG{n}{d}\PYG{o}{.}\PYG{n}{endpoints}\PYG{p}{[}\PYG{l+s+s1}{\PYGZsq{}}\PYG{l+s+s1}{foo}\PYG{l+s+s1}{\PYGZsq{}}\PYG{p}{]}\PYG{o}{.}\PYG{n}{aggregated\PYGZus{}data}\PYG{p}{(}\PYG{p}{)}
\PYG{g+go}{\PYGZob{}\PYGZsq{}total\PYGZus{}visits\PYGZsq{}: 2, \PYGZsq{}total\PYGZus{}execution\PYGZus{}time\PYGZsq{}: 0.6, \PYGZsq{}average\PYGZus{}execution\PYGZus{}time\PYGZsq{}: 0.3, \PYGZsq{}visits\PYGZus{}per\PYGZus{}ip\PYGZsq{}: \PYGZob{}\PYGZsq{}127.0.0.1\PYGZsq{}: 1, \PYGZsq{}127.0.0.2\PYGZsq{}: 1\PYGZcb{}, \PYGZsq{}unique\PYGZus{}visitors\PYGZsq{}: 2, \PYGZsq{}fastest\PYGZus{}measured\PYGZus{}execution\PYGZus{}time\PYGZsq{}: 0.1, \PYGZsq{}fastest\PYGZus{}quartile\PYGZus{}execution\PYGZus{}time\PYGZsq{}: 0.1, \PYGZsq{}median\PYGZus{}execution\PYGZus{}time\PYGZsq{}: 0.3, \PYGZsq{}slowest\PYGZus{}quartile\PYGZus{}execution\PYGZus{}time\PYGZsq{}: 0.5, \PYGZsq{}ninetieth\PYGZus{}percentile\PYGZus{}execution\PYGZus{}time\PYGZsq{}: 0.5, \PYGZsq{}ninety\PYGZhy{}ninth\PYGZus{}percentile\PYGZus{}execution\PYGZus{}time\PYGZsq{}: 0.5, \PYGZsq{}slowest\PYGZus{}measured\PYGZus{}execution\PYGZus{}time\PYGZsq{}: 0.5, \PYGZsq{}versions\PYGZsq{}: [\PYGZsq{}0.1\PYGZsq{}]\PYGZcb{}}
\PYG{g+gp}{\PYGZgt{}\PYGZgt{}\PYGZgt{} }\PYG{n}{d}\PYG{o}{.}\PYG{n}{endpoints}\PYG{p}{[}\PYG{l+s+s1}{\PYGZsq{}}\PYG{l+s+s1}{bar}\PYG{l+s+s1}{\PYGZsq{}}\PYG{p}{]}\PYG{o}{.}\PYG{n}{aggregated\PYGZus{}data}\PYG{p}{(}\PYG{p}{)}
\PYG{g+go}{\PYGZob{}\PYGZsq{}total\PYGZus{}visits\PYGZsq{}: 3, \PYGZsq{}total\PYGZus{}execution\PYGZus{}time\PYGZsq{}: 0.6, \PYGZsq{}average\PYGZus{}execution\PYGZus{}time\PYGZsq{}: 0.2, \PYGZsq{}visits\PYGZus{}per\PYGZus{}ip\PYGZsq{}: \PYGZob{}\PYGZsq{}127.0.0.1\PYGZsq{}: 3\PYGZcb{}, \PYGZsq{}unique\PYGZus{}visitors\PYGZsq{}: 1, \PYGZsq{}fastest\PYGZus{}measured\PYGZus{}execution\PYGZus{}time\PYGZsq{}: 0.2, \PYGZsq{}fastest\PYGZus{}quartile\PYGZus{}execution\PYGZus{}time\PYGZsq{}: 0.2, \PYGZsq{}median\PYGZus{}execution\PYGZus{}time\PYGZsq{}: 0.2, \PYGZsq{}slowest\PYGZus{}quartile\PYGZus{}execution\PYGZus{}time\PYGZsq{}: 0.2, \PYGZsq{}ninetieth\PYGZus{}percentile\PYGZus{}execution\PYGZus{}time\PYGZsq{}: 0.2, \PYGZsq{}ninety\PYGZhy{}ninth\PYGZus{}percentile\PYGZus{}execution\PYGZus{}time\PYGZsq{}: 0.2, \PYGZsq{}slowest\PYGZus{}measured\PYGZus{}execution\PYGZus{}time\PYGZsq{}: 0.2, \PYGZsq{}versions\PYGZsq{}: [\PYGZsq{}0.1\PYGZsq{}]\PYGZcb{}}
\end{sphinxVerbatim}
\index{Dashboard (class in pydash\_app.dashboard.entity)}

\begin{fulllineitems}
\phantomsection\label{\detokenize{pydash_app.dashboard.entity:pydash_app.dashboard.entity.Dashboard}}\pysiglinewithargsret{\sphinxbfcode{\sphinxupquote{class }}\sphinxcode{\sphinxupquote{pydash\_app.dashboard.entity.}}\sphinxbfcode{\sphinxupquote{Dashboard}}}{\emph{url}, \emph{token}, \emph{user\_id}, \emph{name=None}}{}
Bases: \sphinxcode{\sphinxupquote{persistent.Persistent}}

The Dashboard entity knows about:
- Its own properties (id, url, user\_id, endpoints, endpoint\_calls and last\_fetch\_time)
- The functionalities for Dashboard interactions with information from elsewhere.

It does not contain information on how to persistently store/load a dashboard.
This task is handled by the \sphinxtitleref{dashboard\_repository}.
\index{add\_endpoint() (pydash\_app.dashboard.entity.Dashboard method)}

\begin{fulllineitems}
\phantomsection\label{\detokenize{pydash_app.dashboard.entity:pydash_app.dashboard.entity.Dashboard.add_endpoint}}\pysiglinewithargsret{\sphinxbfcode{\sphinxupquote{add\_endpoint}}}{\emph{endpoint}}{}
Adds an endpoint to this dashboard’s internal collection of endpoints.
:param endpoint:  The endpoint to add, expects an Endpoint object.

\end{fulllineitems}

\index{add\_endpoint\_call() (pydash\_app.dashboard.entity.Dashboard method)}

\begin{fulllineitems}
\phantomsection\label{\detokenize{pydash_app.dashboard.entity:pydash_app.dashboard.entity.Dashboard.add_endpoint_call}}\pysiglinewithargsret{\sphinxbfcode{\sphinxupquote{add\_endpoint\_call}}}{\emph{endpoint\_call}}{}~\begin{description}
\item[{Adds an endpoint call to the dashboard. Will register the corresponding endpoint to the dashboard if this has}] \leavevmode
not been done yet.

\end{description}
\begin{quote}\begin{description}
\item[{Parameters}] \leavevmode
\sphinxstyleliteralstrong{\sphinxupquote{endpoint\_call}} \textendash{} The endpoint call to add

\end{description}\end{quote}

\end{fulllineitems}

\index{aggregated\_data() (pydash\_app.dashboard.entity.Dashboard method)}

\begin{fulllineitems}
\phantomsection\label{\detokenize{pydash_app.dashboard.entity:pydash_app.dashboard.entity.Dashboard.aggregated_data}}\pysiglinewithargsret{\sphinxbfcode{\sphinxupquote{aggregated\_data}}}{\emph{filters=\{\}}}{}
Returns aggregated data on this dashboard.
:param filters: A dictionary containing property\_name-value pairs to filter on. The keys are assumed to be strings.
\begin{quote}

This is in the gist of \sphinxtitleref{\{‘day’:‘2018-05-20’, ‘ip’:‘127.0.0.1’\}}
Defaults to an empty dictionary.
\begin{description}
\item[{The currently allowed filter\_names are:}] \leavevmode\begin{itemize}
\item {} 
Time:
* ‘year’   - e.g. ‘2018’
* ‘month’  - e.g. ‘2018-05’
* ‘week’   - e.g. ‘2018-W17’
* ‘day’    - e.g. ‘2018-05-20’
* ‘hour’   - e.g. ‘2018-05-20T20’
* ‘minute’ - e.g. ‘2018-05-20T20-10’

\end{itemize}

Note that for Time filter-values, the formatting is crucial.
\begin{itemize}
\item {} 
Version:
* ‘version’ - e.g. ‘1.0.1’

\item {} 
IP:
* ‘ip’ - e.g. ‘127.0.0.1’

\item {} 
Group-by:
* ‘group\_by’ - e.g. ‘None’

\end{itemize}

\end{description}
\end{quote}
\begin{quote}\begin{description}
\item[{Returns}] \leavevmode
A dict containing aggregated data points.

\end{description}\end{quote}

\end{fulllineitems}

\index{aggregated\_data\_daterange() (pydash\_app.dashboard.entity.Dashboard method)}

\begin{fulllineitems}
\phantomsection\label{\detokenize{pydash_app.dashboard.entity:pydash_app.dashboard.entity.Dashboard.aggregated_data_daterange}}\pysiglinewithargsret{\sphinxbfcode{\sphinxupquote{aggregated\_data\_daterange}}}{\emph{start\_date}, \emph{end\_date}, \emph{granularity}, \emph{filters=\{\}}}{}
Returns the aggregated data on this dashboard over the specified daterange.
:param start\_date: A datetime object that is treated as the inclusive lower bound of the daterange.
:param end\_date: A datetime object that is treated as the inclusive upper bound of the daterange.
:param granularity: A string denoting the granularity of the daterange.
:param filters: A dictionary containing property\_name-value pairs to filter on. The keys are assumed to be strings.
\begin{quote}

This is in the gist of \sphinxtitleref{\{‘day’:‘2018-05-20’, ‘ip’:‘127.0.0.1’\}}
Defaults to an empty dictionary.
\begin{description}
\item[{The currently allowed filter\_names are:}] \leavevmode\begin{itemize}
\item {} 
Version:
* ‘version’ - e.g. ‘1.0.1’

\item {} 
IP:
* ‘ip’ - e.g. ‘127.0.0.1’

\item {} 
Group-by:
* ‘group\_by’ - e.g. ‘None’

\end{itemize}

Note that, contrary to \sphinxtitleref{aggregated\_data} method, Time based filters are not allowed.

\end{description}
\end{quote}
\begin{quote}\begin{description}
\item[{Returns}] \leavevmode
A dictionary with all aggregated statistics and their values.

\end{description}\end{quote}

\end{fulllineitems}

\index{first\_endpoint\_call\_time() (pydash\_app.dashboard.entity.Dashboard method)}

\begin{fulllineitems}
\phantomsection\label{\detokenize{pydash_app.dashboard.entity:pydash_app.dashboard.entity.Dashboard.first_endpoint_call_time}}\pysiglinewithargsret{\sphinxbfcode{\sphinxupquote{first\_endpoint\_call\_time}}}{}{}
\end{fulllineitems}

\index{get\_id() (pydash\_app.dashboard.entity.Dashboard method)}

\begin{fulllineitems}
\phantomsection\label{\detokenize{pydash_app.dashboard.entity:pydash_app.dashboard.entity.Dashboard.get_id}}\pysiglinewithargsret{\sphinxbfcode{\sphinxupquote{get\_id}}}{}{}
\end{fulllineitems}

\index{remove\_endpoint() (pydash\_app.dashboard.entity.Dashboard method)}

\begin{fulllineitems}
\phantomsection\label{\detokenize{pydash_app.dashboard.entity:pydash_app.dashboard.entity.Dashboard.remove_endpoint}}\pysiglinewithargsret{\sphinxbfcode{\sphinxupquote{remove\_endpoint}}}{\emph{endpoint}}{}
Removes an endpoint from this dashboard’s internal collection of endpoints.

Raises a ValueError if no such endpoint exists.
:param endpoint: The endpoint to remove.

\end{fulllineitems}

\index{statistic() (pydash\_app.dashboard.entity.Dashboard method)}

\begin{fulllineitems}
\phantomsection\label{\detokenize{pydash_app.dashboard.entity:pydash_app.dashboard.entity.Dashboard.statistic}}\pysiglinewithargsret{\sphinxbfcode{\sphinxupquote{statistic}}}{\emph{statistic}, \emph{filters=\{\}}}{}
\end{fulllineitems}

\index{statistic\_per\_timeslice() (pydash\_app.dashboard.entity.Dashboard method)}

\begin{fulllineitems}
\phantomsection\label{\detokenize{pydash_app.dashboard.entity:pydash_app.dashboard.entity.Dashboard.statistic_per_timeslice}}\pysiglinewithargsret{\sphinxbfcode{\sphinxupquote{statistic\_per\_timeslice}}}{\emph{statistic}, \emph{timeslice}, \emph{start\_datetime}, \emph{end\_datetime}, \emph{filters=\{\}}}{}
\end{fulllineitems}


\end{fulllineitems}

\index{DashboardState (class in pydash\_app.dashboard.entity)}

\begin{fulllineitems}
\phantomsection\label{\detokenize{pydash_app.dashboard.entity:pydash_app.dashboard.entity.DashboardState}}\pysigline{\sphinxbfcode{\sphinxupquote{class }}\sphinxcode{\sphinxupquote{pydash\_app.dashboard.entity.}}\sphinxbfcode{\sphinxupquote{DashboardState}}}
Bases: \sphinxhref{https://docs.python.org/3/library/enum.html\#enum.Enum}{\sphinxcode{\sphinxupquote{enum.Enum}}}

The DashboardState enum indicates the state in which a Dashboard can remain, regarding remote fetching:
\begin{itemize}
\item {} 
not\_initialized indicates the dashboard is newly created and not initialized with Endpoints and
historic EndpointCalls;

\item {} 
initialized\_endpoints indicates the dashboard has successfully initialized Endpoints,
but not yet historical EndpointCalls;

\item {} 
initialize\_endpoints\_failure indicates something went wrong while initializing Endpoints, which means
initialization of Endpoints needs to be retried;

\item {} 
initialized\_endpoint\_calls indicates the dashboard has successfully initialized historical EndpointCalls,
and can start fetching new EndpointCalls in a periodic task;

\item {} 
initialize\_endpoint\_calls\_failure indicates something went wrong while initializing historical EndpointCalls,
which means this needs to be retried;

\item {} 
fetched\_endpoint\_calls indicates last time new EndpointCalls were fetched, it was done successfully;

\item {} 
fetch\_endpoint\_calls\_failure indicates something went wrong while fetching new EndpointCalls,
which means this needs to be retried.

\end{itemize}
\index{fetch\_endpoint\_calls\_failure (pydash\_app.dashboard.entity.DashboardState attribute)}

\begin{fulllineitems}
\phantomsection\label{\detokenize{pydash_app.dashboard.entity:pydash_app.dashboard.entity.DashboardState.fetch_endpoint_calls_failure}}\pysigline{\sphinxbfcode{\sphinxupquote{fetch\_endpoint\_calls\_failure}}\sphinxbfcode{\sphinxupquote{ = 31}}}
\end{fulllineitems}

\index{fetched\_endpoint\_calls (pydash\_app.dashboard.entity.DashboardState attribute)}

\begin{fulllineitems}
\phantomsection\label{\detokenize{pydash_app.dashboard.entity:pydash_app.dashboard.entity.DashboardState.fetched_endpoint_calls}}\pysigline{\sphinxbfcode{\sphinxupquote{fetched\_endpoint\_calls}}\sphinxbfcode{\sphinxupquote{ = 30}}}
\end{fulllineitems}

\index{initialize\_endpoint\_calls\_failure (pydash\_app.dashboard.entity.DashboardState attribute)}

\begin{fulllineitems}
\phantomsection\label{\detokenize{pydash_app.dashboard.entity:pydash_app.dashboard.entity.DashboardState.initialize_endpoint_calls_failure}}\pysigline{\sphinxbfcode{\sphinxupquote{initialize\_endpoint\_calls\_failure}}\sphinxbfcode{\sphinxupquote{ = 21}}}
\end{fulllineitems}

\index{initialize\_endpoints\_failure (pydash\_app.dashboard.entity.DashboardState attribute)}

\begin{fulllineitems}
\phantomsection\label{\detokenize{pydash_app.dashboard.entity:pydash_app.dashboard.entity.DashboardState.initialize_endpoints_failure}}\pysigline{\sphinxbfcode{\sphinxupquote{initialize\_endpoints\_failure}}\sphinxbfcode{\sphinxupquote{ = 11}}}
\end{fulllineitems}

\index{initialized\_endpoint\_calls (pydash\_app.dashboard.entity.DashboardState attribute)}

\begin{fulllineitems}
\phantomsection\label{\detokenize{pydash_app.dashboard.entity:pydash_app.dashboard.entity.DashboardState.initialized_endpoint_calls}}\pysigline{\sphinxbfcode{\sphinxupquote{initialized\_endpoint\_calls}}\sphinxbfcode{\sphinxupquote{ = 20}}}
\end{fulllineitems}

\index{initialized\_endpoints (pydash\_app.dashboard.entity.DashboardState attribute)}

\begin{fulllineitems}
\phantomsection\label{\detokenize{pydash_app.dashboard.entity:pydash_app.dashboard.entity.DashboardState.initialized_endpoints}}\pysigline{\sphinxbfcode{\sphinxupquote{initialized\_endpoints}}\sphinxbfcode{\sphinxupquote{ = 10}}}
\end{fulllineitems}

\index{not\_initialized (pydash\_app.dashboard.entity.DashboardState attribute)}

\begin{fulllineitems}
\phantomsection\label{\detokenize{pydash_app.dashboard.entity:pydash_app.dashboard.entity.DashboardState.not_initialized}}\pysigline{\sphinxbfcode{\sphinxupquote{not\_initialized}}\sphinxbfcode{\sphinxupquote{ = 0}}}
\end{fulllineitems}


\end{fulllineitems}



\subparagraph{pydash\_app.dashboard.repository module}
\label{\detokenize{pydash_app.dashboard.repository::doc}}\label{\detokenize{pydash_app.dashboard.repository:module-pydash_app.dashboard.repository}}\label{\detokenize{pydash_app.dashboard.repository:pydash-app-dashboard-repository-module}}\index{pydash\_app.dashboard.repository (module)}
This module handles the persistence of \sphinxtitleref{Dashboard} entities:

It is an adapter of the actual persistence layer, to insulate the application
from datastore-specific details.

It handles a subset of the following tasks
(specifically, it only actually contains functions for the tasks the application needs in its current state!):
\begin{itemize}
\item {} 
Creating new entities of the specified type and finding them based on id.

\end{itemize}

\fvset{hllines={, ,}}%
\begin{sphinxVerbatim}[commandchars=\\\{\}]
\PYG{g+gp}{\PYGZgt{}\PYGZgt{}\PYGZgt{} }\PYG{k+kn}{import} \PYG{n+nn}{pydash\PYGZus{}app}\PYG{n+nn}{.}\PYG{n+nn}{dashboard}\PYG{n+nn}{.}\PYG{n+nn}{entity} \PYG{k}{as} \PYG{n+nn}{dashboard}
\PYG{g+gp}{\PYGZgt{}\PYGZgt{}\PYGZgt{} }\PYG{k+kn}{import} \PYG{n+nn}{uuid}
\PYG{g+gp}{\PYGZgt{}\PYGZgt{}\PYGZgt{} }\PYG{n}{dashboard} \PYG{o}{=} \PYG{n}{dashboard}\PYG{o}{.}\PYG{n}{Dashboard}\PYG{p}{(}\PYG{l+s+s2}{\PYGZdq{}}\PYG{l+s+s2}{\PYGZdq{}}\PYG{p}{,} \PYG{l+s+s2}{\PYGZdq{}}\PYG{l+s+s2}{\PYGZdq{}}\PYG{p}{,} \PYG{n+nb}{str}\PYG{p}{(}\PYG{n}{uuid}\PYG{o}{.}\PYG{n}{uuid4}\PYG{p}{(}\PYG{p}{)}\PYG{p}{)}\PYG{p}{)}
\PYG{g+gp}{\PYGZgt{}\PYGZgt{}\PYGZgt{} }\PYG{n}{add}\PYG{p}{(}\PYG{n}{dashboard}\PYG{p}{)}
\PYG{g+gp}{\PYGZgt{}\PYGZgt{}\PYGZgt{} }\PYG{n}{found\PYGZus{}dashboard} \PYG{o}{=} \PYG{n}{find}\PYG{p}{(}\PYG{n}{dashboard}\PYG{o}{.}\PYG{n}{get\PYGZus{}id}\PYG{p}{(}\PYG{p}{)}\PYG{p}{)}
\PYG{g+gp}{\PYGZgt{}\PYGZgt{}\PYGZgt{} }\PYG{n}{found\PYGZus{}dashboard}\PYG{o}{.}\PYG{n}{get\PYGZus{}id}\PYG{p}{(}\PYG{p}{)} \PYG{o}{==} \PYG{n}{dashboard}\PYG{o}{.}\PYG{n}{get\PYGZus{}id}\PYG{p}{(}\PYG{p}{)}
\PYG{g+go}{True}
\end{sphinxVerbatim}
\begin{itemize}
\item {} 
Asking for all dashboards is also possible!

\end{itemize}

\fvset{hllines={, ,}}%
\begin{sphinxVerbatim}[commandchars=\\\{\}]
\PYG{g+gp}{\PYGZgt{}\PYGZgt{}\PYGZgt{} }\PYG{n+nb}{all}\PYG{p}{(}\PYG{p}{)} 
\PYG{g+go}{\PYGZlt{}OOBTreeItems object at 0x...\PYGZgt{}}
\end{sphinxVerbatim}
\begin{itemize}
\item {} 
Adding multiple instances of the same dashboard will return a KeyError or a DuplicateIndexError

\end{itemize}

TODO fix it so that it actually errors??
\textgreater{}\textgreater{}\textgreater{} import pydash\_app.dashboard.entity as dashboard
\textgreater{}\textgreater{}\textgreater{} import uuid
\textgreater{}\textgreater{}\textgreater{} dashboard = dashboard.Dashboard(“”, “”, str(uuid.uuid4()))
\textgreater{}\textgreater{}\textgreater{} add(dashboard)
\textgreater{}\textgreater{}\textgreater{} add(dashboard)
\begin{itemize}
\item {} 
Persisting updated versions of existing entities.

\end{itemize}

\fvset{hllines={, ,}}%
\begin{sphinxVerbatim}[commandchars=\\\{\}]
\PYG{g+gp}{\PYGZgt{}\PYGZgt{}\PYGZgt{} }\PYG{k+kn}{import} \PYG{n+nn}{pydash\PYGZus{}app}\PYG{n+nn}{.}\PYG{n+nn}{dashboard}\PYG{n+nn}{.}\PYG{n+nn}{entity} \PYG{k}{as} \PYG{n+nn}{dashboard}
\PYG{g+gp}{\PYGZgt{}\PYGZgt{}\PYGZgt{} }\PYG{k+kn}{import} \PYG{n+nn}{uuid}
\PYG{g+gp}{\PYGZgt{}\PYGZgt{}\PYGZgt{} }\PYG{n}{dashboard} \PYG{o}{=} \PYG{n}{dashboard}\PYG{o}{.}\PYG{n}{Dashboard}\PYG{p}{(}\PYG{l+s+s2}{\PYGZdq{}}\PYG{l+s+s2}{\PYGZdq{}}\PYG{p}{,} \PYG{l+s+s2}{\PYGZdq{}}\PYG{l+s+s2}{\PYGZdq{}}\PYG{p}{,} \PYG{n+nb}{str}\PYG{p}{(}\PYG{n}{uuid}\PYG{o}{.}\PYG{n}{uuid4}\PYG{p}{(}\PYG{p}{)}\PYG{p}{)}\PYG{p}{)}
\PYG{g+gp}{\PYGZgt{}\PYGZgt{}\PYGZgt{} }\PYG{n}{add}\PYG{p}{(}\PYG{n}{dashboard}\PYG{p}{)}
\PYG{g+gp}{\PYGZgt{}\PYGZgt{}\PYGZgt{} }\PYG{n}{dashboard}\PYG{o}{.}\PYG{n}{token} \PYG{o}{=} \PYG{l+s+s2}{\PYGZdq{}}\PYG{l+s+s2}{newToken}\PYG{l+s+s2}{\PYGZdq{}}
\PYG{g+gp}{\PYGZgt{}\PYGZgt{}\PYGZgt{} }\PYG{n}{update}\PYG{p}{(}\PYG{n}{dashboard}\PYG{p}{)}
\PYG{g+gp}{\PYGZgt{}\PYGZgt{}\PYGZgt{} }\PYG{n}{found\PYGZus{}dashboard} \PYG{o}{=} \PYG{n}{find}\PYG{p}{(}\PYG{n}{dashboard}\PYG{o}{.}\PYG{n}{get\PYGZus{}id}\PYG{p}{(}\PYG{p}{)}\PYG{p}{)}
\PYG{g+gp}{\PYGZgt{}\PYGZgt{}\PYGZgt{} }\PYG{n}{found\PYGZus{}dashboard}\PYG{o}{.}\PYG{n}{token} \PYG{o}{==} \PYG{n}{dashboard}\PYG{o}{.}\PYG{n}{token}
\PYG{g+go}{True}
\end{sphinxVerbatim}
\begin{itemize}
\item {} 
Deleting entities from the persistence layer, note that find() will return a KeyError if no dashboard was found.

\end{itemize}

\fvset{hllines={, ,}}%
\begin{sphinxVerbatim}[commandchars=\\\{\}]
\PYG{g+gp}{\PYGZgt{}\PYGZgt{}\PYGZgt{} }\PYG{n}{delete}\PYG{p}{(}\PYG{n}{dashboard}\PYG{p}{)}
\PYG{g+gp}{\PYGZgt{}\PYGZgt{}\PYGZgt{} }\PYG{n}{found\PYGZus{}dashboard} \PYG{o}{=} \PYG{n}{find}\PYG{p}{(}\PYG{n}{dashboard}\PYG{o}{.}\PYG{n}{get\PYGZus{}id}\PYG{p}{(}\PYG{p}{)}\PYG{p}{)}
\PYG{g+gt}{Traceback (most recent call last):}
    \PYG{o}{.}\PYG{o}{.}\PYG{o}{.}
\PYG{g+gr}{KeyError}
\end{sphinxVerbatim}
\begin{itemize}
\item {} 
Deleting non-existent dashboards will result in a KeyError.

\end{itemize}

\fvset{hllines={, ,}}%
\begin{sphinxVerbatim}[commandchars=\\\{\}]
\PYG{g+gp}{\PYGZgt{}\PYGZgt{}\PYGZgt{} }\PYG{k+kn}{import} \PYG{n+nn}{pydash\PYGZus{}app}\PYG{n+nn}{.}\PYG{n+nn}{dashboard}\PYG{n+nn}{.}\PYG{n+nn}{entity} \PYG{k}{as} \PYG{n+nn}{dashboard}
\PYG{g+gp}{\PYGZgt{}\PYGZgt{}\PYGZgt{} }\PYG{k+kn}{import} \PYG{n+nn}{uuid}
\PYG{g+gp}{\PYGZgt{}\PYGZgt{}\PYGZgt{} }\PYG{n}{dashboard} \PYG{o}{=} \PYG{n}{dashboard}\PYG{o}{.}\PYG{n}{Dashboard}\PYG{p}{(}\PYG{l+s+s2}{\PYGZdq{}}\PYG{l+s+s2}{\PYGZdq{}}\PYG{p}{,} \PYG{l+s+s2}{\PYGZdq{}}\PYG{l+s+s2}{\PYGZdq{}}\PYG{p}{,} \PYG{n+nb}{str}\PYG{p}{(}\PYG{n}{uuid}\PYG{o}{.}\PYG{n}{uuid4}\PYG{p}{(}\PYG{p}{)}\PYG{p}{)}\PYG{p}{)}
\PYG{g+gp}{\PYGZgt{}\PYGZgt{}\PYGZgt{} }\PYG{n}{add}\PYG{p}{(}\PYG{n}{dashboard}\PYG{p}{)}
\PYG{g+gp}{\PYGZgt{}\PYGZgt{}\PYGZgt{} }\PYG{n}{delete}\PYG{p}{(}\PYG{n}{dashboard}\PYG{p}{)}
\PYG{g+gp}{\PYGZgt{}\PYGZgt{}\PYGZgt{} }\PYG{n}{delete}\PYG{p}{(}\PYG{n}{dashboard}\PYG{p}{)}
\PYG{g+gt}{Traceback (most recent call last):}
    \PYG{o}{.}\PYG{o}{.}\PYG{o}{.}
\PYG{g+gr}{KeyError}
\end{sphinxVerbatim}
\index{add() (in module pydash\_app.dashboard.repository)}

\begin{fulllineitems}
\phantomsection\label{\detokenize{pydash_app.dashboard.repository:pydash_app.dashboard.repository.add}}\pysiglinewithargsret{\sphinxcode{\sphinxupquote{pydash\_app.dashboard.repository.}}\sphinxbfcode{\sphinxupquote{add}}}{\emph{dashboard}}{}
\end{fulllineitems}

\index{all() (in module pydash\_app.dashboard.repository)}

\begin{fulllineitems}
\phantomsection\label{\detokenize{pydash_app.dashboard.repository:pydash_app.dashboard.repository.all}}\pysiglinewithargsret{\sphinxcode{\sphinxupquote{pydash\_app.dashboard.repository.}}\sphinxbfcode{\sphinxupquote{all}}}{}{}
\end{fulllineitems}

\index{clear\_all() (in module pydash\_app.dashboard.repository)}

\begin{fulllineitems}
\phantomsection\label{\detokenize{pydash_app.dashboard.repository:pydash_app.dashboard.repository.clear_all}}\pysiglinewithargsret{\sphinxcode{\sphinxupquote{pydash\_app.dashboard.repository.}}\sphinxbfcode{\sphinxupquote{clear\_all}}}{}{}
\end{fulllineitems}

\index{delete() (in module pydash\_app.dashboard.repository)}

\begin{fulllineitems}
\phantomsection\label{\detokenize{pydash_app.dashboard.repository:pydash_app.dashboard.repository.delete}}\pysiglinewithargsret{\sphinxcode{\sphinxupquote{pydash\_app.dashboard.repository.}}\sphinxbfcode{\sphinxupquote{delete}}}{\emph{dashboard}}{}
\end{fulllineitems}

\index{find() (in module pydash\_app.dashboard.repository)}

\begin{fulllineitems}
\phantomsection\label{\detokenize{pydash_app.dashboard.repository:pydash_app.dashboard.repository.find}}\pysiglinewithargsret{\sphinxcode{\sphinxupquote{pydash\_app.dashboard.repository.}}\sphinxbfcode{\sphinxupquote{find}}}{\emph{dashboard\_id}}{}
\end{fulllineitems}

\index{update() (in module pydash\_app.dashboard.repository)}

\begin{fulllineitems}
\phantomsection\label{\detokenize{pydash_app.dashboard.repository:pydash_app.dashboard.repository.update}}\pysiglinewithargsret{\sphinxcode{\sphinxupquote{pydash\_app.dashboard.repository.}}\sphinxbfcode{\sphinxupquote{update}}}{\emph{dashboard}}{}
\end{fulllineitems}



\subsubsection{pydash\_app.user package}
\label{\detokenize{pydash_app.user::doc}}\label{\detokenize{pydash_app.user:pydash-app-user-package}}\label{\detokenize{pydash_app.user:module-pydash_app.user}}\index{pydash\_app.user (module)}
This module is the public interface (available to the web-application pydash\_web)
for interacting with Users.

Example Usage:

\fvset{hllines={, ,}}%
\begin{sphinxVerbatim}[commandchars=\\\{\}]
\PYG{g+gp}{\PYGZgt{}\PYGZgt{}\PYGZgt{} }\PYG{n}{gandalf} \PYG{o}{=} \PYG{n}{User}\PYG{p}{(}\PYG{l+s+s2}{\PYGZdq{}}\PYG{l+s+s2}{Gandalf}\PYG{l+s+s2}{\PYGZdq{}}\PYG{p}{,} \PYG{l+s+s2}{\PYGZdq{}}\PYG{l+s+s2}{pass}\PYG{l+s+s2}{\PYGZdq{}}\PYG{p}{,} \PYG{l+s+s1}{\PYGZsq{}}\PYG{l+s+s1}{some@email.com}\PYG{l+s+s1}{\PYGZsq{}}\PYG{p}{)}
\PYG{g+gp}{\PYGZgt{}\PYGZgt{}\PYGZgt{} }\PYG{n}{add\PYGZus{}to\PYGZus{}repository}\PYG{p}{(}\PYG{n}{gandalf}\PYG{p}{)}
\PYG{g+gp}{...}
\PYG{g+gp}{\PYGZgt{}\PYGZgt{}\PYGZgt{} }\PYG{n}{found\PYGZus{}user} \PYG{o}{=} \PYG{n}{find}\PYG{p}{(}\PYG{n}{gandalf}\PYG{o}{.}\PYG{n}{id}\PYG{p}{)}
\PYG{g+gp}{\PYGZgt{}\PYGZgt{}\PYGZgt{} }\PYG{n}{found\PYGZus{}user}\PYG{o}{.}\PYG{n}{name} \PYG{o}{==} \PYG{l+s+s2}{\PYGZdq{}}\PYG{l+s+s2}{Gandalf}\PYG{l+s+s2}{\PYGZdq{}}
\PYG{g+go}{True}
\end{sphinxVerbatim}

You can also use a string-version of the ID to find the user again:

\fvset{hllines={, ,}}%
\begin{sphinxVerbatim}[commandchars=\\\{\}]
\PYG{g+gp}{\PYGZgt{}\PYGZgt{}\PYGZgt{} }\PYG{n}{found\PYGZus{}user} \PYG{o}{=} \PYG{n}{find}\PYG{p}{(}\PYG{n+nb}{str}\PYG{p}{(}\PYG{n}{gandalf}\PYG{o}{.}\PYG{n}{id}\PYG{p}{)}\PYG{p}{)}
\PYG{g+gp}{\PYGZgt{}\PYGZgt{}\PYGZgt{} }\PYG{n}{found\PYGZus{}user}\PYG{o}{.}\PYG{n}{name} \PYG{o}{==} \PYG{l+s+s2}{\PYGZdq{}}\PYG{l+s+s2}{Gandalf}\PYG{l+s+s2}{\PYGZdq{}}
\PYG{g+go}{True}
\end{sphinxVerbatim}

\fvset{hllines={, ,}}%
\begin{sphinxVerbatim}[commandchars=\\\{\}]
\PYG{g+gp}{\PYGZgt{}\PYGZgt{}\PYGZgt{} }\PYG{n}{found\PYGZus{}user2} \PYG{o}{=} \PYG{n}{find\PYGZus{}by\PYGZus{}name}\PYG{p}{(}\PYG{l+s+s2}{\PYGZdq{}}\PYG{l+s+s2}{Gandalf}\PYG{l+s+s2}{\PYGZdq{}}\PYG{p}{)}
\PYG{g+gp}{\PYGZgt{}\PYGZgt{}\PYGZgt{} }\PYG{n}{found\PYGZus{}user2} \PYG{o}{==} \PYG{n}{found\PYGZus{}user}
\PYG{g+go}{True}
\PYG{g+gp}{\PYGZgt{}\PYGZgt{}\PYGZgt{} }\PYG{n}{find\PYGZus{}by\PYGZus{}name}\PYG{p}{(}\PYG{l+s+s2}{\PYGZdq{}}\PYG{l+s+s2}{Dumbledore}\PYG{l+s+s2}{\PYGZdq{}}\PYG{p}{)}
\PYG{g+gp}{\PYGZgt{}\PYGZgt{}\PYGZgt{} }\PYG{c+c1}{\PYGZsh{} \PYGZca{}Returns nothing}
\PYG{g+gp}{\PYGZgt{}\PYGZgt{}\PYGZgt{} }\PYG{n}{res\PYGZus{}user} \PYG{o}{=} \PYG{n}{authenticate}\PYG{p}{(}\PYG{l+s+s2}{\PYGZdq{}}\PYG{l+s+s2}{Gandalf}\PYG{l+s+s2}{\PYGZdq{}}\PYG{p}{,} \PYG{l+s+s2}{\PYGZdq{}}\PYG{l+s+s2}{pass}\PYG{l+s+s2}{\PYGZdq{}}\PYG{p}{)}
\PYG{g+gp}{\PYGZgt{}\PYGZgt{}\PYGZgt{} }\PYG{n}{res\PYGZus{}user}\PYG{o}{.}\PYG{n}{name} \PYG{o}{==} \PYG{l+s+s2}{\PYGZdq{}}\PYG{l+s+s2}{Gandalf}\PYG{l+s+s2}{\PYGZdq{}}
\PYG{g+go}{True}
\PYG{g+gp}{\PYGZgt{}\PYGZgt{}\PYGZgt{} }\PYG{n}{authenticate}\PYG{p}{(}\PYG{l+s+s2}{\PYGZdq{}}\PYG{l+s+s2}{Gandalf}\PYG{l+s+s2}{\PYGZdq{}}\PYG{p}{,} \PYG{l+s+s2}{\PYGZdq{}}\PYG{l+s+s2}{youshallnot}\PYG{l+s+s2}{\PYGZdq{}}\PYG{p}{)}
\PYG{g+gp}{\PYGZgt{}\PYGZgt{}\PYGZgt{} }\PYG{c+c1}{\PYGZsh{} \PYGZca{}Returns nothing}
\PYG{g+gp}{\PYGZgt{}\PYGZgt{}\PYGZgt{} }\PYG{n}{authenticate}\PYG{p}{(}\PYG{l+s+s2}{\PYGZdq{}}\PYG{l+s+s2}{Dumbledore}\PYG{l+s+s2}{\PYGZdq{}}\PYG{p}{,} \PYG{l+s+s2}{\PYGZdq{}}\PYG{l+s+s2}{secrets}\PYG{l+s+s2}{\PYGZdq{}}\PYG{p}{)}
\PYG{g+gp}{\PYGZgt{}\PYGZgt{}\PYGZgt{} }\PYG{c+c1}{\PYGZsh{} \PYGZca{}Returns nothing}
\end{sphinxVerbatim}
\index{add\_to\_repository() (in module pydash\_app.user)}

\begin{fulllineitems}
\phantomsection\label{\detokenize{pydash_app.user:pydash_app.user.add_to_repository}}\pysiglinewithargsret{\sphinxcode{\sphinxupquote{pydash\_app.user.}}\sphinxbfcode{\sphinxupquote{add\_to\_repository}}}{\emph{user}}{}
Adds the given User-entity to the user\_repository. Raises a KeyError if the user is already in the repository.
:param user: The User-entity in question.

Adding the same user twice with the same name is not allowed:

\fvset{hllines={, ,}}%
\begin{sphinxVerbatim}[commandchars=\\\{\}]
\PYGZgt{}\PYGZgt{}\PYGZgt{} gandalf1 = User(\PYGZdq{}Gandalf\PYGZdq{}, \PYGZdq{}pass\PYGZdq{}, \PYGZsq{}some@email.com\PYGZsq{})
\PYGZgt{}\PYGZgt{}\PYGZgt{} add\PYGZus{}to\PYGZus{}repository(gandalf1)
\PYGZgt{}\PYGZgt{}\PYGZgt{} gandalf2 = User(\PYGZdq{}Gandalf\PYGZdq{}, \PYGZdq{}balrog\PYGZdq{}, \PYGZsq{}some@email.com\PYGZsq{})
\PYGZgt{}\PYGZgt{}\PYGZgt{} add\PYGZus{}to\PYGZus{}repository(gandalf2)
Traceback (most recent call last):
  ...
multi\PYGZus{}indexed\PYGZus{}collection.DuplicateIndexError
\end{sphinxVerbatim}

\end{fulllineitems}

\index{authenticate() (in module pydash\_app.user)}

\begin{fulllineitems}
\phantomsection\label{\detokenize{pydash_app.user:pydash_app.user.authenticate}}\pysiglinewithargsret{\sphinxcode{\sphinxupquote{pydash\_app.user.}}\sphinxbfcode{\sphinxupquote{authenticate}}}{\emph{name}, \emph{password}}{}
Attempts to authenticate the user with name \sphinxtitleref{name}
and password \sphinxtitleref{password}.

If authentication fails (unknown user or incorrect password), returns None.
Otherwise, returns the user object.

\end{fulllineitems}

\index{check\_password\_requirements() (in module pydash\_app.user)}

\begin{fulllineitems}
\phantomsection\label{\detokenize{pydash_app.user:pydash_app.user.check_password_requirements}}\pysiglinewithargsret{\sphinxcode{\sphinxupquote{pydash\_app.user.}}\sphinxbfcode{\sphinxupquote{check\_password\_requirements}}}{\emph{password}}{}
\end{fulllineitems}

\index{find() (in module pydash\_app.user)}

\begin{fulllineitems}
\phantomsection\label{\detokenize{pydash_app.user:pydash_app.user.find}}\pysiglinewithargsret{\sphinxcode{\sphinxupquote{pydash\_app.user.}}\sphinxbfcode{\sphinxupquote{find}}}{\emph{user\_id}}{}
Returns a single User-entity with the given UUID or None if it could not be found.

user\_id- UUID of the user we hope to find.

\end{fulllineitems}

\index{find\_by\_name() (in module pydash\_app.user)}

\begin{fulllineitems}
\phantomsection\label{\detokenize{pydash_app.user:pydash_app.user.find_by_name}}\pysiglinewithargsret{\sphinxcode{\sphinxupquote{pydash\_app.user.}}\sphinxbfcode{\sphinxupquote{find\_by\_name}}}{\emph{name}}{}
Returns a single User-entity with the given \sphinxtitleref{name}, or None if it could not be found.

name \textendash{} Name of the user we hope to find.

\end{fulllineitems}

\index{find\_by\_verification\_code() (in module pydash\_app.user)}

\begin{fulllineitems}
\phantomsection\label{\detokenize{pydash_app.user:pydash_app.user.find_by_verification_code}}\pysiglinewithargsret{\sphinxcode{\sphinxupquote{pydash\_app.user.}}\sphinxbfcode{\sphinxupquote{find\_by\_verification\_code}}}{\emph{verification\_code}}{}
Returns a single User-entity with the given \sphinxtitleref{verification\_code}, or None if it could not be found.
:param verification\_code: The verification code of the user we hope to find.

\end{fulllineitems}

\index{maybe\_find\_user() (in module pydash\_app.user)}

\begin{fulllineitems}
\phantomsection\label{\detokenize{pydash_app.user:pydash_app.user.maybe_find_user}}\pysiglinewithargsret{\sphinxcode{\sphinxupquote{pydash\_app.user.}}\sphinxbfcode{\sphinxupquote{maybe\_find\_user}}}{\emph{user\_id}}{}
Returns the User entity, or \sphinxtitleref{None} if it does not exist.

\fvset{hllines={, ,}}%
\begin{sphinxVerbatim}[commandchars=\\\{\}]
\PYG{g+gp}{\PYGZgt{}\PYGZgt{}\PYGZgt{} }\PYG{n}{user} \PYG{o}{=} \PYG{n}{User}\PYG{p}{(}\PYG{l+s+s2}{\PYGZdq{}}\PYG{l+s+s2}{Gandalf}\PYG{l+s+s2}{\PYGZdq{}}\PYG{p}{,} \PYG{l+s+s2}{\PYGZdq{}}\PYG{l+s+s2}{pass}\PYG{l+s+s2}{\PYGZdq{}}\PYG{p}{,} \PYG{l+s+s1}{\PYGZsq{}}\PYG{l+s+s1}{some@email.com}\PYG{l+s+s1}{\PYGZsq{}}\PYG{p}{)}
\PYG{g+gp}{\PYGZgt{}\PYGZgt{}\PYGZgt{} }\PYG{n}{add\PYGZus{}to\PYGZus{}repository}\PYG{p}{(}\PYG{n}{user}\PYG{p}{)}
\PYG{g+gp}{...}
\PYG{g+gp}{\PYGZgt{}\PYGZgt{}\PYGZgt{} }\PYG{n}{found\PYGZus{}user} \PYG{o}{=} \PYG{n}{maybe\PYGZus{}find\PYGZus{}user}\PYG{p}{(}\PYG{n}{user}\PYG{o}{.}\PYG{n}{id}\PYG{p}{)}
\PYG{g+gp}{\PYGZgt{}\PYGZgt{}\PYGZgt{} }\PYG{n}{found\PYGZus{}user}\PYG{o}{.}\PYG{n}{name} \PYG{o}{==} \PYG{l+s+s2}{\PYGZdq{}}\PYG{l+s+s2}{Gandalf}\PYG{l+s+s2}{\PYGZdq{}}
\PYG{g+go}{True}
\PYG{g+gp}{\PYGZgt{}\PYGZgt{}\PYGZgt{} }\PYG{k+kn}{import} \PYG{n+nn}{uuid}
\PYG{g+gp}{\PYGZgt{}\PYGZgt{}\PYGZgt{} }\PYG{n}{unexistent\PYGZus{}uuid} \PYG{o}{=} \PYG{n}{uuid}\PYG{o}{.}\PYG{n}{UUID}\PYG{p}{(}\PYG{l+s+s1}{\PYGZsq{}}\PYG{l+s+s1}{ced84534\PYGZhy{}7a55\PYGZhy{}440f\PYGZhy{}ad77\PYGZhy{}9912466fe022}\PYG{l+s+s1}{\PYGZsq{}}\PYG{p}{)}
\PYG{g+gp}{\PYGZgt{}\PYGZgt{}\PYGZgt{} }\PYG{n}{unexistent\PYGZus{}user} \PYG{o}{=} \PYG{n}{maybe\PYGZus{}find\PYGZus{}user}\PYG{p}{(}\PYG{n}{unexistent\PYGZus{}uuid}\PYG{p}{)}
\PYG{g+gp}{\PYGZgt{}\PYGZgt{}\PYGZgt{} }\PYG{n}{unexistent\PYGZus{}user} \PYG{o}{==} \PYG{k+kc}{None}
\PYG{g+go}{True}
\end{sphinxVerbatim}

\end{fulllineitems}

\index{remove\_from\_repository() (in module pydash\_app.user)}

\begin{fulllineitems}
\phantomsection\label{\detokenize{pydash_app.user:pydash_app.user.remove_from_repository}}\pysiglinewithargsret{\sphinxcode{\sphinxupquote{pydash\_app.user.}}\sphinxbfcode{\sphinxupquote{remove\_from\_repository}}}{\emph{user\_id}}{}
Removes the User-entity whose user\_id is \sphinxtitleref{user\_id} from the repository.

\fvset{hllines={, ,}}%
\begin{sphinxVerbatim}[commandchars=\\\{\}]
\PYG{g+gp}{\PYGZgt{}\PYGZgt{}\PYGZgt{} }\PYG{n}{gandalf1} \PYG{o}{=} \PYG{n}{User}\PYG{p}{(}\PYG{l+s+s2}{\PYGZdq{}}\PYG{l+s+s2}{Gandalf}\PYG{l+s+s2}{\PYGZdq{}}\PYG{p}{,} \PYG{l+s+s2}{\PYGZdq{}}\PYG{l+s+s2}{pass}\PYG{l+s+s2}{\PYGZdq{}}\PYG{p}{,} \PYG{l+s+s1}{\PYGZsq{}}\PYG{l+s+s1}{some@email.com}\PYG{l+s+s1}{\PYGZsq{}}\PYG{p}{)}
\PYG{g+gp}{\PYGZgt{}\PYGZgt{}\PYGZgt{} }\PYG{n}{add\PYGZus{}to\PYGZus{}repository}\PYG{p}{(}\PYG{n}{gandalf1}\PYG{p}{)}
\PYG{g+gp}{\PYGZgt{}\PYGZgt{}\PYGZgt{} }\PYG{n}{remove\PYGZus{}from\PYGZus{}repository}\PYG{p}{(}\PYG{n}{gandalf1}\PYG{o}{.}\PYG{n}{get\PYGZus{}id}\PYG{p}{(}\PYG{p}{)}\PYG{p}{)}
\PYG{g+gp}{\PYGZgt{}\PYGZgt{}\PYGZgt{} }\PYG{n}{found\PYGZus{}user} \PYG{o}{=} \PYG{n}{find\PYGZus{}by\PYGZus{}name}\PYG{p}{(}\PYG{l+s+s2}{\PYGZdq{}}\PYG{l+s+s2}{Gandalf}\PYG{l+s+s2}{\PYGZdq{}}\PYG{p}{)}
\PYG{g+gp}{\PYGZgt{}\PYGZgt{}\PYGZgt{} }\PYG{n}{found\PYGZus{}user} \PYG{o}{==} \PYG{k+kc}{None}
\PYG{g+go}{True}
\end{sphinxVerbatim}

Will raise a KeyError if said user is not in the repository.

\fvset{hllines={, ,}}%
\begin{sphinxVerbatim}[commandchars=\\\{\}]
\PYG{g+gp}{\PYGZgt{}\PYGZgt{}\PYGZgt{} }\PYG{n}{gandalf1} \PYG{o}{=} \PYG{n}{User}\PYG{p}{(}\PYG{l+s+s2}{\PYGZdq{}}\PYG{l+s+s2}{Gandalf}\PYG{l+s+s2}{\PYGZdq{}}\PYG{p}{,} \PYG{l+s+s2}{\PYGZdq{}}\PYG{l+s+s2}{pass}\PYG{l+s+s2}{\PYGZdq{}}\PYG{p}{,} \PYG{l+s+s1}{\PYGZsq{}}\PYG{l+s+s1}{some@email.com}\PYG{l+s+s1}{\PYGZsq{}}\PYG{p}{)}
\PYG{g+gp}{\PYGZgt{}\PYGZgt{}\PYGZgt{} }\PYG{n}{add\PYGZus{}to\PYGZus{}repository}\PYG{p}{(}\PYG{n}{gandalf1}\PYG{p}{)}
\PYG{g+gp}{\PYGZgt{}\PYGZgt{}\PYGZgt{} }\PYG{n}{remove\PYGZus{}from\PYGZus{}repository}\PYG{p}{(}\PYG{n}{gandalf1}\PYG{o}{.}\PYG{n}{get\PYGZus{}id}\PYG{p}{(}\PYG{p}{)}\PYG{p}{)}
\PYG{g+gp}{\PYGZgt{}\PYGZgt{}\PYGZgt{} }\PYG{n}{remove\PYGZus{}from\PYGZus{}repository}\PYG{p}{(}\PYG{n}{gandalf1}\PYG{o}{.}\PYG{n}{get\PYGZus{}id}\PYG{p}{(}\PYG{p}{)}\PYG{p}{)}
\PYG{g+gt}{Traceback (most recent call last):}
  \PYG{c}{...}
\PYG{g+gr}{KeyError}
\end{sphinxVerbatim}
\begin{quote}\begin{description}
\item[{Parameters}] \leavevmode
\sphinxstyleliteralstrong{\sphinxupquote{user\_id}} \textendash{} The ID of the User-entity to be removed. This can be either a UUID-entity or the corresponding
string representation.

\end{description}\end{quote}

\end{fulllineitems}

\index{verify() (in module pydash\_app.user)}

\begin{fulllineitems}
\phantomsection\label{\detokenize{pydash_app.user:pydash_app.user.verify}}\pysiglinewithargsret{\sphinxcode{\sphinxupquote{pydash\_app.user.}}\sphinxbfcode{\sphinxupquote{verify}}}{\emph{verification\_code}}{}
Attempts to verify a user with the provided verification code.
This is intended as a one-time action per user after registration.
:param verification\_code: The verification code that should match the User-entity’s verification code.
\begin{quote}

Can be a string or UUID object.
\end{quote}
\begin{quote}\begin{description}
\item[{Returns}] \leavevmode
Returns True if both verification codes are equal, returns False otherwise.
Raises an InvalidVerificationCodeError when the provided verification code is invalid.
Raises an VerificationCodeExpiredError when the provided verification code has expired.

\end{description}\end{quote}

\end{fulllineitems}



\paragraph{Subpackages}
\label{\detokenize{pydash_app.user:subpackages}}

\subparagraph{pydash\_app.user.services package}
\label{\detokenize{pydash_app.user.services::doc}}\label{\detokenize{pydash_app.user.services:module-pydash_app.user.services}}\label{\detokenize{pydash_app.user.services:pydash-app-user-services-package}}\index{pydash\_app.user.services (module)}
Contains services for the ‘User’ concern.

These are things that use or manipulate ‘User’ entities to perform tasks,
where these tasks are either too complex to put in the User Entity,
or where these are heavily interacting with outside logic that the business domain entity should not concern itself with directly.


\subparagraph{Submodules}
\label{\detokenize{pydash_app.user.services:submodules}}

\subparagraph{pydash\_app.user.services.pruning module}
\label{\detokenize{pydash_app.user.services.pruning::doc}}\label{\detokenize{pydash_app.user.services.pruning:pydash-app-user-services-pruning-module}}\label{\detokenize{pydash_app.user.services.pruning:module-pydash_app.user.services.pruning}}\index{pydash\_app.user.services.pruning (module)}
Provides functionality to periodically remove all users that have not verified their account.
\index{schedule\_periodic\_pruning\_task() (in module pydash\_app.user.services.pruning)}

\begin{fulllineitems}
\phantomsection\label{\detokenize{pydash_app.user.services.pruning:pydash_app.user.services.pruning.schedule_periodic_pruning_task}}\pysiglinewithargsret{\sphinxcode{\sphinxupquote{pydash\_app.user.services.pruning.}}\sphinxbfcode{\sphinxupquote{schedule\_periodic\_pruning\_task}}}{\emph{interval=datetime.timedelta(1)}, \emph{scheduler=\textless{}periodic\_tasks.task\_scheduler.TaskScheduler object\textgreater{}}}{}
\end{fulllineitems}



\subparagraph{pydash\_app.user.services.seeding module}
\label{\detokenize{pydash_app.user.services.seeding::doc}}\label{\detokenize{pydash_app.user.services.seeding:module-pydash_app.user.services.seeding}}\label{\detokenize{pydash_app.user.services.seeding:pydash-app-user-services-seeding-module}}\index{pydash\_app.user.services.seeding (module)}
Fills the application with some preliminary users
to make it easier to test code in development and staging environments.
\index{seed() (in module pydash\_app.user.services.seeding)}

\begin{fulllineitems}
\phantomsection\label{\detokenize{pydash_app.user.services.seeding:pydash_app.user.services.seeding.seed}}\pysiglinewithargsret{\sphinxcode{\sphinxupquote{pydash\_app.user.services.seeding.}}\sphinxbfcode{\sphinxupquote{seed}}}{}{}
Stores some preliminary debug users in the datastore,
to be used during development.

\fvset{hllines={, ,}}%
\begin{sphinxVerbatim}[commandchars=\\\{\}]
\PYG{g+gp}{\PYGZgt{}\PYGZgt{}\PYGZgt{} }\PYG{n}{seed}\PYG{p}{(}\PYG{p}{)} 
\PYG{g+go}{Adding user \PYGZlt{}User id=... name=Alberto\PYGZgt{}}
\PYG{g+go}{Adding user \PYGZlt{}User id=... name=Arjan\PYGZgt{}}
\PYG{g+go}{Adding user \PYGZlt{}User id=... name=JeroenO\PYGZgt{}}
\PYG{g+go}{Adding user \PYGZlt{}User id=... name=JeroenL\PYGZgt{}}
\PYG{g+go}{Adding user \PYGZlt{}User id=... name=Koen\PYGZgt{}}
\PYG{g+go}{Adding user \PYGZlt{}User id=... name=Lars\PYGZgt{}}
\PYG{g+go}{Adding user \PYGZlt{}User id=... name=Patrick\PYGZgt{}}
\PYG{g+go}{Adding user \PYGZlt{}User id=... name=Tom\PYGZgt{}}
\PYG{g+go}{Adding user \PYGZlt{}User id=... name=W\PYGZhy{}M\PYGZgt{}}
\PYG{g+go}{Seeding of users is done!}
\PYG{g+gp}{\PYGZgt{}\PYGZgt{}\PYGZgt{} }\PYG{n}{found\PYGZus{}user} \PYG{o}{=} \PYG{n}{repository}\PYG{o}{.}\PYG{n}{find\PYGZus{}by\PYGZus{}name}\PYG{p}{(}\PYG{l+s+s2}{\PYGZdq{}}\PYG{l+s+s2}{Alberto}\PYG{l+s+s2}{\PYGZdq{}}\PYG{p}{)}
\PYG{g+gp}{\PYGZgt{}\PYGZgt{}\PYGZgt{} }\PYG{n}{found\PYGZus{}user}\PYG{o}{.}\PYG{n}{name} \PYG{o}{==} \PYG{l+s+s2}{\PYGZdq{}}\PYG{l+s+s2}{Alberto}\PYG{l+s+s2}{\PYGZdq{}}
\PYG{g+go}{True}
\end{sphinxVerbatim}

\end{fulllineitems}



\paragraph{Submodules}
\label{\detokenize{pydash_app.user:submodules}}

\subparagraph{pydash\_app.user.entity module}
\label{\detokenize{pydash_app.user.entity::doc}}\label{\detokenize{pydash_app.user.entity:module-pydash_app.user.entity}}\label{\detokenize{pydash_app.user.entity:pydash-app-user-entity-module}}\index{pydash\_app.user.entity (module)}\index{User (class in pydash\_app.user.entity)}

\begin{fulllineitems}
\phantomsection\label{\detokenize{pydash_app.user.entity:pydash_app.user.entity.User}}\pysiglinewithargsret{\sphinxbfcode{\sphinxupquote{class }}\sphinxcode{\sphinxupquote{pydash\_app.user.entity.}}\sphinxbfcode{\sphinxupquote{User}}}{\emph{name}, \emph{password}, \emph{mail}}{}
Bases: \sphinxcode{\sphinxupquote{persistent.Persistent}}, \sphinxcode{\sphinxupquote{flask\_login.mixins.UserMixin}}

The User entity knows about:
\begin{itemize}
\item {} 
What properties a User has

\item {} 
What functionality makes sense to have this User interact with information from elsewhere.

\end{itemize}

Per Domain Driven Design, it does \_not\_ contain information on how to persistently store/load a user!
(That is instead handled by the \sphinxtitleref{user\_repository}).

The User entity checks its parameters on creation:

\fvset{hllines={, ,}}%
\begin{sphinxVerbatim}[commandchars=\\\{\}]
\PYG{g+gp}{\PYGZgt{}\PYGZgt{}\PYGZgt{} }\PYG{n}{User}\PYG{p}{(}\PYG{l+m+mi}{42}\PYG{p}{,} \PYG{l+m+mi}{32}\PYG{p}{,} \PYG{l+m+mi}{11}\PYG{p}{)}
\PYG{g+gt}{Traceback (most recent call last):}
  \PYG{c}{...}
\PYG{g+gr}{TypeError}
\end{sphinxVerbatim}
\index{check\_password() (pydash\_app.user.entity.User method)}

\begin{fulllineitems}
\phantomsection\label{\detokenize{pydash_app.user.entity:pydash_app.user.entity.User.check_password}}\pysiglinewithargsret{\sphinxbfcode{\sphinxupquote{check\_password}}}{\emph{password}}{}
\end{fulllineitems}

\index{generate\_new\_verification\_code() (pydash\_app.user.entity.User method)}

\begin{fulllineitems}
\phantomsection\label{\detokenize{pydash_app.user.entity:pydash_app.user.entity.User.generate_new_verification_code}}\pysiglinewithargsret{\sphinxbfcode{\sphinxupquote{generate\_new\_verification\_code}}}{}{}
\end{fulllineitems}

\index{get\_id() (pydash\_app.user.entity.User method)}

\begin{fulllineitems}
\phantomsection\label{\detokenize{pydash_app.user.entity:pydash_app.user.entity.User.get_id}}\pysiglinewithargsret{\sphinxbfcode{\sphinxupquote{get\_id}}}{}{}
\end{fulllineitems}

\index{get\_verification\_code() (pydash\_app.user.entity.User method)}

\begin{fulllineitems}
\phantomsection\label{\detokenize{pydash_app.user.entity:pydash_app.user.entity.User.get_verification_code}}\pysiglinewithargsret{\sphinxbfcode{\sphinxupquote{get\_verification\_code}}}{}{}
Returns this User’s verification code or None if it has expired or this User has already been verified

\end{fulllineitems}

\index{get\_verification\_code\_expiration\_date() (pydash\_app.user.entity.User method)}

\begin{fulllineitems}
\phantomsection\label{\detokenize{pydash_app.user.entity:pydash_app.user.entity.User.get_verification_code_expiration_date}}\pysiglinewithargsret{\sphinxbfcode{\sphinxupquote{get\_verification\_code\_expiration\_date}}}{}{}
Returns a datetime object of when this User’s verification code is about to expire,
or None if it has already expired or this User has already been verified

\end{fulllineitems}

\index{has\_verification\_code\_expired() (pydash\_app.user.entity.User method)}

\begin{fulllineitems}
\phantomsection\label{\detokenize{pydash_app.user.entity:pydash_app.user.entity.User.has_verification_code_expired}}\pysiglinewithargsret{\sphinxbfcode{\sphinxupquote{has\_verification\_code\_expired}}}{}{}
Returns a boolean whether this User’s verification code has expired, if it has one.

\end{fulllineitems}

\index{is\_verified() (pydash\_app.user.entity.User method)}

\begin{fulllineitems}
\phantomsection\label{\detokenize{pydash_app.user.entity:pydash_app.user.entity.User.is_verified}}\pysiglinewithargsret{\sphinxbfcode{\sphinxupquote{is\_verified}}}{}{}
\end{fulllineitems}

\index{set\_password() (pydash\_app.user.entity.User method)}

\begin{fulllineitems}
\phantomsection\label{\detokenize{pydash_app.user.entity:pydash_app.user.entity.User.set_password}}\pysiglinewithargsret{\sphinxbfcode{\sphinxupquote{set\_password}}}{\emph{password}}{}
\end{fulllineitems}


\end{fulllineitems}



\subparagraph{pydash\_app.user.repository module}
\label{\detokenize{pydash_app.user.repository::doc}}\label{\detokenize{pydash_app.user.repository:pydash-app-user-repository-module}}\label{\detokenize{pydash_app.user.repository:module-pydash_app.user.repository}}\index{pydash\_app.user.repository (module)}
This module handles the persistence of \sphinxtitleref{User} entities:

It is an adapter of the actual persistence layer, to insulate the application
from datastore-specific details.

It handles a subset of the following tasks
(specifically, it only actually contains functions for the tasks the application needs in its current state!):
- Creating new entities of the specified type
- Finding them based on certain attributes
- Persisting updated versions of existing entities.
- Deleting entities from the persistence layer.
\index{add() (in module pydash\_app.user.repository)}

\begin{fulllineitems}
\phantomsection\label{\detokenize{pydash_app.user.repository:pydash_app.user.repository.add}}\pysiglinewithargsret{\sphinxcode{\sphinxupquote{pydash\_app.user.repository.}}\sphinxbfcode{\sphinxupquote{add}}}{\emph{user}}{}
Adds the User-entity to the repository. Will raise a (KeyError, DuplicateIndexError) tuple on failure.
:param user: The User-entity to add.

\fvset{hllines={, ,}}%
\begin{sphinxVerbatim}[commandchars=\\\{\}]
\PYG{g+gp}{\PYGZgt{}\PYGZgt{}\PYGZgt{} }\PYG{n+nb}{list}\PYG{p}{(}\PYG{n+nb}{all}\PYG{p}{(}\PYG{p}{)}\PYG{p}{)}
\PYG{g+go}{[]}
\PYG{g+gp}{\PYGZgt{}\PYGZgt{}\PYGZgt{} }\PYG{n}{gandalf} \PYG{o}{=} \PYG{n}{User}\PYG{p}{(}\PYG{l+s+s2}{\PYGZdq{}}\PYG{l+s+s2}{Gandalf}\PYG{l+s+s2}{\PYGZdq{}}\PYG{p}{,} \PYG{l+s+s2}{\PYGZdq{}}\PYG{l+s+s2}{pass}\PYG{l+s+s2}{\PYGZdq{}}\PYG{p}{,} \PYG{l+s+s1}{\PYGZsq{}}\PYG{l+s+s1}{some@email.com}\PYG{l+s+s1}{\PYGZsq{}}\PYG{p}{)}
\PYG{g+gp}{\PYGZgt{}\PYGZgt{}\PYGZgt{} }\PYG{n}{dumbledore} \PYG{o}{=} \PYG{n}{User}\PYG{p}{(}\PYG{l+s+s2}{\PYGZdq{}}\PYG{l+s+s2}{Dumbledore}\PYG{l+s+s2}{\PYGZdq{}}\PYG{p}{,} \PYG{l+s+s2}{\PYGZdq{}}\PYG{l+s+s2}{secret}\PYG{l+s+s2}{\PYGZdq{}}\PYG{p}{,} \PYG{l+s+s1}{\PYGZsq{}}\PYG{l+s+s1}{some@email.com}\PYG{l+s+s1}{\PYGZsq{}}\PYG{p}{)}
\PYG{g+gp}{\PYGZgt{}\PYGZgt{}\PYGZgt{} }\PYG{n}{add}\PYG{p}{(}\PYG{n}{gandalf}\PYG{p}{)}
\PYG{g+gp}{\PYGZgt{}\PYGZgt{}\PYGZgt{} }\PYG{n}{add}\PYG{p}{(}\PYG{n}{dumbledore}\PYG{p}{)}
\PYG{g+gp}{\PYGZgt{}\PYGZgt{}\PYGZgt{} }\PYG{n+nb}{sorted}\PYG{p}{(}\PYG{p}{[}\PYG{n}{user}\PYG{o}{.}\PYG{n}{name} \PYG{k}{for} \PYG{n}{user} \PYG{o+ow}{in} \PYG{n+nb}{all}\PYG{p}{(}\PYG{p}{)}\PYG{p}{]}\PYG{p}{)}
\PYG{g+go}{[\PYGZsq{}Dumbledore\PYGZsq{}, \PYGZsq{}Gandalf\PYGZsq{}]}
\end{sphinxVerbatim}

\end{fulllineitems}

\index{all() (in module pydash\_app.user.repository)}

\begin{fulllineitems}
\phantomsection\label{\detokenize{pydash_app.user.repository:pydash_app.user.repository.all}}\pysiglinewithargsret{\sphinxcode{\sphinxupquote{pydash\_app.user.repository.}}\sphinxbfcode{\sphinxupquote{all}}}{}{}
Returns a (lazy) collection of all users (in no guaranteed order).

\fvset{hllines={, ,}}%
\begin{sphinxVerbatim}[commandchars=\\\{\}]
\PYG{g+gp}{\PYGZgt{}\PYGZgt{}\PYGZgt{} }\PYG{n+nb}{list}\PYG{p}{(}\PYG{n+nb}{all}\PYG{p}{(}\PYG{p}{)}\PYG{p}{)}
\PYG{g+go}{[]}
\PYG{g+gp}{\PYGZgt{}\PYGZgt{}\PYGZgt{} }\PYG{n}{gandalf} \PYG{o}{=} \PYG{n}{User}\PYG{p}{(}\PYG{l+s+s2}{\PYGZdq{}}\PYG{l+s+s2}{Gandalf}\PYG{l+s+s2}{\PYGZdq{}}\PYG{p}{,} \PYG{l+s+s2}{\PYGZdq{}}\PYG{l+s+s2}{pass}\PYG{l+s+s2}{\PYGZdq{}}\PYG{p}{,} \PYG{l+s+s1}{\PYGZsq{}}\PYG{l+s+s1}{some@email.com}\PYG{l+s+s1}{\PYGZsq{}}\PYG{p}{)}
\PYG{g+gp}{\PYGZgt{}\PYGZgt{}\PYGZgt{} }\PYG{n}{dumbledore} \PYG{o}{=} \PYG{n}{User}\PYG{p}{(}\PYG{l+s+s2}{\PYGZdq{}}\PYG{l+s+s2}{Dumbledore}\PYG{l+s+s2}{\PYGZdq{}}\PYG{p}{,} \PYG{l+s+s2}{\PYGZdq{}}\PYG{l+s+s2}{secret}\PYG{l+s+s2}{\PYGZdq{}}\PYG{p}{,} \PYG{l+s+s1}{\PYGZsq{}}\PYG{l+s+s1}{some@email.com}\PYG{l+s+s1}{\PYGZsq{}}\PYG{p}{)}
\PYG{g+gp}{\PYGZgt{}\PYGZgt{}\PYGZgt{} }\PYG{n}{add}\PYG{p}{(}\PYG{n}{gandalf}\PYG{p}{)}
\PYG{g+gp}{\PYGZgt{}\PYGZgt{}\PYGZgt{} }\PYG{n}{add}\PYG{p}{(}\PYG{n}{dumbledore}\PYG{p}{)}
\PYG{g+gp}{\PYGZgt{}\PYGZgt{}\PYGZgt{} }\PYG{n+nb}{sorted}\PYG{p}{(}\PYG{p}{[}\PYG{n}{user}\PYG{o}{.}\PYG{n}{name} \PYG{k}{for} \PYG{n}{user} \PYG{o+ow}{in} \PYG{n+nb}{all}\PYG{p}{(}\PYG{p}{)}\PYG{p}{]}\PYG{p}{)}
\PYG{g+go}{[\PYGZsq{}Dumbledore\PYGZsq{}, \PYGZsq{}Gandalf\PYGZsq{}]}
\PYG{g+gp}{\PYGZgt{}\PYGZgt{}\PYGZgt{} }\PYG{n}{clear\PYGZus{}all}\PYG{p}{(}\PYG{p}{)}
\PYG{g+gp}{\PYGZgt{}\PYGZgt{}\PYGZgt{} }\PYG{n+nb}{sorted}\PYG{p}{(}\PYG{p}{[}\PYG{n}{user}\PYG{o}{.}\PYG{n}{name} \PYG{k}{for} \PYG{n}{user} \PYG{o+ow}{in} \PYG{n+nb}{all}\PYG{p}{(}\PYG{p}{)}\PYG{p}{]}\PYG{p}{)}
\PYG{g+go}{[]}
\end{sphinxVerbatim}

\end{fulllineitems}

\index{all\_unverified() (in module pydash\_app.user.repository)}

\begin{fulllineitems}
\phantomsection\label{\detokenize{pydash_app.user.repository:pydash_app.user.repository.all_unverified}}\pysiglinewithargsret{\sphinxcode{\sphinxupquote{pydash\_app.user.repository.}}\sphinxbfcode{\sphinxupquote{all\_unverified}}}{}{}
Returns a collection of all unverified users (in no guaranteed order).

\end{fulllineitems}

\index{clear\_all() (in module pydash\_app.user.repository)}

\begin{fulllineitems}
\phantomsection\label{\detokenize{pydash_app.user.repository:pydash_app.user.repository.clear_all}}\pysiglinewithargsret{\sphinxcode{\sphinxupquote{pydash\_app.user.repository.}}\sphinxbfcode{\sphinxupquote{clear\_all}}}{}{}
Flushes the database.

\fvset{hllines={, ,}}%
\begin{sphinxVerbatim}[commandchars=\\\{\}]
\PYG{g+gp}{\PYGZgt{}\PYGZgt{}\PYGZgt{} }\PYG{n}{gandalf} \PYG{o}{=} \PYG{n}{User}\PYG{p}{(}\PYG{l+s+s2}{\PYGZdq{}}\PYG{l+s+s2}{Gandalf}\PYG{l+s+s2}{\PYGZdq{}}\PYG{p}{,} \PYG{l+s+s2}{\PYGZdq{}}\PYG{l+s+s2}{pass}\PYG{l+s+s2}{\PYGZdq{}}\PYG{p}{,} \PYG{l+s+s1}{\PYGZsq{}}\PYG{l+s+s1}{some@email.com}\PYG{l+s+s1}{\PYGZsq{}}\PYG{p}{)}
\PYG{g+gp}{\PYGZgt{}\PYGZgt{}\PYGZgt{} }\PYG{n}{dumbledore} \PYG{o}{=} \PYG{n}{User}\PYG{p}{(}\PYG{l+s+s2}{\PYGZdq{}}\PYG{l+s+s2}{Dumbledore}\PYG{l+s+s2}{\PYGZdq{}}\PYG{p}{,} \PYG{l+s+s2}{\PYGZdq{}}\PYG{l+s+s2}{secret}\PYG{l+s+s2}{\PYGZdq{}}\PYG{p}{,} \PYG{l+s+s1}{\PYGZsq{}}\PYG{l+s+s1}{some@email.com}\PYG{l+s+s1}{\PYGZsq{}}\PYG{p}{)}
\PYG{g+gp}{\PYGZgt{}\PYGZgt{}\PYGZgt{} }\PYG{n}{add}\PYG{p}{(}\PYG{n}{gandalf}\PYG{p}{)}
\PYG{g+gp}{\PYGZgt{}\PYGZgt{}\PYGZgt{} }\PYG{n}{add}\PYG{p}{(}\PYG{n}{dumbledore}\PYG{p}{)}
\PYG{g+gp}{\PYGZgt{}\PYGZgt{}\PYGZgt{} }\PYG{n+nb}{sorted}\PYG{p}{(}\PYG{p}{[}\PYG{n}{user}\PYG{o}{.}\PYG{n}{name} \PYG{k}{for} \PYG{n}{user} \PYG{o+ow}{in} \PYG{n+nb}{all}\PYG{p}{(}\PYG{p}{)}\PYG{p}{]}\PYG{p}{)}
\PYG{g+go}{[\PYGZsq{}Dumbledore\PYGZsq{}, \PYGZsq{}Gandalf\PYGZsq{}]}
\PYG{g+gp}{\PYGZgt{}\PYGZgt{}\PYGZgt{} }\PYG{n}{clear\PYGZus{}all}\PYG{p}{(}\PYG{p}{)}
\PYG{g+gp}{\PYGZgt{}\PYGZgt{}\PYGZgt{} }\PYG{n+nb}{list}\PYG{p}{(}\PYG{n+nb}{all}\PYG{p}{(}\PYG{p}{)}\PYG{p}{)}
\PYG{g+go}{[]}
\end{sphinxVerbatim}

\end{fulllineitems}

\index{delete\_by\_id() (in module pydash\_app.user.repository)}

\begin{fulllineitems}
\phantomsection\label{\detokenize{pydash_app.user.repository:pydash_app.user.repository.delete_by_id}}\pysiglinewithargsret{\sphinxcode{\sphinxupquote{pydash\_app.user.repository.}}\sphinxbfcode{\sphinxupquote{delete\_by\_id}}}{\emph{user\_id}}{}
Removes the User-entity whose user\_id is \sphinxtitleref{user\_id} from the repository.
Will raise a KeyError if said user is not in the repository.
Note that this might also occur when delete\_by\_id(user\_id) is called in the middle of the deletion,
\begin{quote}

in a multiprocessing environment.
\end{quote}
\begin{quote}\begin{description}
\item[{Parameters}] \leavevmode
\sphinxstyleliteralstrong{\sphinxupquote{user\_id}} \textendash{} The ID of the User-entity to be removed. This can be either a UUID-entity or the corresponding
string representation.

\end{description}\end{quote}

\fvset{hllines={, ,}}%
\begin{sphinxVerbatim}[commandchars=\\\{\}]
\PYG{g+gp}{\PYGZgt{}\PYGZgt{}\PYGZgt{} }\PYG{n}{gandalf} \PYG{o}{=} \PYG{n}{User}\PYG{p}{(}\PYG{l+s+s2}{\PYGZdq{}}\PYG{l+s+s2}{Gandalf}\PYG{l+s+s2}{\PYGZdq{}}\PYG{p}{,} \PYG{l+s+s2}{\PYGZdq{}}\PYG{l+s+s2}{pass}\PYG{l+s+s2}{\PYGZdq{}}\PYG{p}{,} \PYG{l+s+s1}{\PYGZsq{}}\PYG{l+s+s1}{some@email.com}\PYG{l+s+s1}{\PYGZsq{}}\PYG{p}{)}
\PYG{g+gp}{\PYGZgt{}\PYGZgt{}\PYGZgt{} }\PYG{n}{add}\PYG{p}{(}\PYG{n}{gandalf}\PYG{p}{)}
\PYG{g+gp}{\PYGZgt{}\PYGZgt{}\PYGZgt{} }\PYG{n}{find\PYGZus{}by\PYGZus{}name}\PYG{p}{(}\PYG{l+s+s2}{\PYGZdq{}}\PYG{l+s+s2}{Gandalf}\PYG{l+s+s2}{\PYGZdq{}}\PYG{p}{)} \PYG{o}{==} \PYG{n}{gandalf}
\PYG{g+go}{True}
\PYG{g+gp}{\PYGZgt{}\PYGZgt{}\PYGZgt{} }\PYG{n}{delete\PYGZus{}by\PYGZus{}id}\PYG{p}{(}\PYG{n}{gandalf}\PYG{o}{.}\PYG{n}{get\PYGZus{}id}\PYG{p}{(}\PYG{p}{)}\PYG{p}{)}
\PYG{g+gp}{\PYGZgt{}\PYGZgt{}\PYGZgt{} }\PYG{n}{find\PYGZus{}by\PYGZus{}name}\PYG{p}{(}\PYG{l+s+s2}{\PYGZdq{}}\PYG{l+s+s2}{Gandalf}\PYG{l+s+s2}{\PYGZdq{}}\PYG{p}{)} \PYG{o}{==} \PYG{n}{gandalf}
\PYG{g+go}{False}
\end{sphinxVerbatim}

\end{fulllineitems}

\index{find() (in module pydash\_app.user.repository)}

\begin{fulllineitems}
\phantomsection\label{\detokenize{pydash_app.user.repository:pydash_app.user.repository.find}}\pysiglinewithargsret{\sphinxcode{\sphinxupquote{pydash\_app.user.repository.}}\sphinxbfcode{\sphinxupquote{find}}}{\emph{user\_id}}{}
Finds a user in the database.
:param user\_id: UUID for the user to be retrieved.
:return: User object or None if no user could be found.

\end{fulllineitems}

\index{find\_by\_name() (in module pydash\_app.user.repository)}

\begin{fulllineitems}
\phantomsection\label{\detokenize{pydash_app.user.repository:pydash_app.user.repository.find_by_name}}\pysiglinewithargsret{\sphinxcode{\sphinxupquote{pydash\_app.user.repository.}}\sphinxbfcode{\sphinxupquote{find\_by\_name}}}{\emph{name}}{}
Returns a single User-entity with the given \sphinxtitleref{name}, or None if it could not be found.

name \textendash{} Name of the user we hope to find.

\end{fulllineitems}

\index{find\_by\_verification\_code() (in module pydash\_app.user.repository)}

\begin{fulllineitems}
\phantomsection\label{\detokenize{pydash_app.user.repository:pydash_app.user.repository.find_by_verification_code}}\pysiglinewithargsret{\sphinxcode{\sphinxupquote{pydash\_app.user.repository.}}\sphinxbfcode{\sphinxupquote{find\_by\_verification\_code}}}{\emph{verification\_code}}{}
Returns a single User-entity with the given \sphinxtitleref{verification\_code}, or None if it could not be found.
The latter case might indicate that the user does not exist, or that the verification code has expired.
:param verification\_code: The verification code of the user we hope to find.
Should be a pydash\_app.user.verification\_code.VerificationCode object.

\end{fulllineitems}

\index{update() (in module pydash\_app.user.repository)}

\begin{fulllineitems}
\phantomsection\label{\detokenize{pydash_app.user.repository:pydash_app.user.repository.update}}\pysiglinewithargsret{\sphinxcode{\sphinxupquote{pydash\_app.user.repository.}}\sphinxbfcode{\sphinxupquote{update}}}{\emph{user}}{}
Changes the user’s information

\fvset{hllines={, ,}}%
\begin{sphinxVerbatim}[commandchars=\\\{\}]
\PYG{g+gp}{\PYGZgt{}\PYGZgt{}\PYGZgt{} }\PYG{n}{gandalf} \PYG{o}{=} \PYG{n}{User}\PYG{p}{(}\PYG{l+s+s2}{\PYGZdq{}}\PYG{l+s+s2}{GandalfTheGrey}\PYG{l+s+s2}{\PYGZdq{}}\PYG{p}{,} \PYG{l+s+s2}{\PYGZdq{}}\PYG{l+s+s2}{pass}\PYG{l+s+s2}{\PYGZdq{}}\PYG{p}{,} \PYG{l+s+s1}{\PYGZsq{}}\PYG{l+s+s1}{some@email.com}\PYG{l+s+s1}{\PYGZsq{}}\PYG{p}{)}
\PYG{g+gp}{\PYGZgt{}\PYGZgt{}\PYGZgt{} }\PYG{n}{add}\PYG{p}{(}\PYG{n}{gandalf}\PYG{p}{)}
\PYG{g+gp}{\PYGZgt{}\PYGZgt{}\PYGZgt{} }\PYG{n}{gandalf}\PYG{o}{.}\PYG{n}{name} \PYG{o}{=} \PYG{l+s+s2}{\PYGZdq{}}\PYG{l+s+s2}{GandalfTheWhite}\PYG{l+s+s2}{\PYGZdq{}}
\PYG{g+gp}{\PYGZgt{}\PYGZgt{}\PYGZgt{} }\PYG{n}{update}\PYG{p}{(}\PYG{n}{gandalf}\PYG{p}{)}
\PYG{g+gp}{\PYGZgt{}\PYGZgt{}\PYGZgt{} }\PYG{n}{find\PYGZus{}by\PYGZus{}name}\PYG{p}{(}\PYG{l+s+s2}{\PYGZdq{}}\PYG{l+s+s2}{GandalfTheGrey}\PYG{l+s+s2}{\PYGZdq{}}\PYG{p}{)} \PYG{o}{==} \PYG{n}{gandalf}
\PYG{g+go}{False}
\PYG{g+gp}{\PYGZgt{}\PYGZgt{}\PYGZgt{} }\PYG{n}{find\PYGZus{}by\PYGZus{}name}\PYG{p}{(}\PYG{l+s+s2}{\PYGZdq{}}\PYG{l+s+s2}{GandalfTheWhite}\PYG{l+s+s2}{\PYGZdq{}}\PYG{p}{)} \PYG{o}{==} \PYG{n}{gandalf}
\PYG{g+go}{True}
\end{sphinxVerbatim}

\end{fulllineitems}



\subparagraph{pydash\_app.user.verification module}
\label{\detokenize{pydash_app.user.verification::doc}}\label{\detokenize{pydash_app.user.verification:module-pydash_app.user.verification}}\label{\detokenize{pydash_app.user.verification:pydash-app-user-verification-module}}\index{pydash\_app.user.verification (module)}\index{InvalidVerificationCodeError}

\begin{fulllineitems}
\phantomsection\label{\detokenize{pydash_app.user.verification:pydash_app.user.verification.InvalidVerificationCodeError}}\pysigline{\sphinxbfcode{\sphinxupquote{exception }}\sphinxcode{\sphinxupquote{pydash\_app.user.verification.}}\sphinxbfcode{\sphinxupquote{InvalidVerificationCodeError}}}
Bases: \sphinxhref{https://docs.python.org/3/library/exceptions.html\#Exception}{\sphinxcode{\sphinxupquote{Exception}}}

\end{fulllineitems}

\index{VerificationCodeExpiredError}

\begin{fulllineitems}
\phantomsection\label{\detokenize{pydash_app.user.verification:pydash_app.user.verification.VerificationCodeExpiredError}}\pysigline{\sphinxbfcode{\sphinxupquote{exception }}\sphinxcode{\sphinxupquote{pydash\_app.user.verification.}}\sphinxbfcode{\sphinxupquote{VerificationCodeExpiredError}}}
Bases: \sphinxhref{https://docs.python.org/3/library/exceptions.html\#Exception}{\sphinxcode{\sphinxupquote{Exception}}}

\end{fulllineitems}

\index{verify() (in module pydash\_app.user.verification)}

\begin{fulllineitems}
\phantomsection\label{\detokenize{pydash_app.user.verification:pydash_app.user.verification.verify}}\pysiglinewithargsret{\sphinxcode{\sphinxupquote{pydash\_app.user.verification.}}\sphinxbfcode{\sphinxupquote{verify}}}{\emph{verification\_code}}{}
Attempts to verify a user with the provided verification code.
This is intended as a one-time action per user after registration.
:param verification\_code: The verification code that should match the User-entity’s verification code.
\begin{quote}

Can be a string or UUID object.
\end{quote}
\begin{quote}\begin{description}
\item[{Returns}] \leavevmode
Returns True if both verification codes are equal, returns False otherwise.
Raises an InvalidVerificationCodeError when the provided verification code is invalid.
Raises an VerificationCodeExpiredError when the provided verification code has expired.

\end{description}\end{quote}

\end{fulllineitems}



\subparagraph{pydash\_app.user.verification\_code module}
\label{\detokenize{pydash_app.user.verification_code::doc}}\label{\detokenize{pydash_app.user.verification_code:module-pydash_app.user.verification_code}}\label{\detokenize{pydash_app.user.verification_code:pydash-app-user-verification-code-module}}\index{pydash\_app.user.verification\_code (module)}\index{VerificationCode (class in pydash\_app.user.verification\_code)}

\begin{fulllineitems}
\phantomsection\label{\detokenize{pydash_app.user.verification_code:pydash_app.user.verification_code.VerificationCode}}\pysiglinewithargsret{\sphinxbfcode{\sphinxupquote{class }}\sphinxcode{\sphinxupquote{pydash\_app.user.verification\_code.}}\sphinxbfcode{\sphinxupquote{VerificationCode}}}{\emph{expiration\_time=datetime.timedelta(1)}}{}
Bases: \sphinxhref{https://docs.python.org/3/library/functions.html\#object}{\sphinxcode{\sphinxupquote{object}}}

A ‘smart’ randomly generated verification code that keeps track of whether it has expired.
Default expiration time is 7 days.
\index{is\_expired() (pydash\_app.user.verification\_code.VerificationCode method)}

\begin{fulllineitems}
\phantomsection\label{\detokenize{pydash_app.user.verification_code:pydash_app.user.verification_code.VerificationCode.is_expired}}\pysiglinewithargsret{\sphinxbfcode{\sphinxupquote{is\_expired}}}{}{}
\end{fulllineitems}


\end{fulllineitems}



\section{pydash\_database package}
\label{\detokenize{pydash_database::doc}}\label{\detokenize{pydash_database:pydash-database-package}}\label{\detokenize{pydash_database:module-pydash_database}}\index{pydash\_database (module)}\index{MultiIndexedPersistentCollection (class in pydash\_database)}

\begin{fulllineitems}
\phantomsection\label{\detokenize{pydash_database:pydash_database.MultiIndexedPersistentCollection}}\pysiglinewithargsret{\sphinxbfcode{\sphinxupquote{class }}\sphinxcode{\sphinxupquote{pydash\_database.}}\sphinxbfcode{\sphinxupquote{MultiIndexedPersistentCollection}}}{\emph{properties}}{}
Bases: \sphinxcode{\sphinxupquote{multi\_indexed\_collection.MultiIndexedCollection}}, \sphinxcode{\sphinxupquote{persistent.Persistent}}

\end{fulllineitems}

\index{database\_connection() (in module pydash\_database)}

\begin{fulllineitems}
\phantomsection\label{\detokenize{pydash_database:pydash_database.database_connection}}\pysiglinewithargsret{\sphinxcode{\sphinxupquote{pydash\_database.}}\sphinxbfcode{\sphinxupquote{database\_connection}}}{}{}
\end{fulllineitems}

\index{database\_root() (in module pydash\_database)}

\begin{fulllineitems}
\phantomsection\label{\detokenize{pydash_database:pydash_database.database_root}}\pysiglinewithargsret{\sphinxcode{\sphinxupquote{pydash\_database.}}\sphinxbfcode{\sphinxupquote{database\_root}}}{}{}
Returns the ZEO database root object.
Wraps a database connection; a new connection is initialized once
on each multiprocessing.Process.
(on all subsequent calls on this process, the connection is re-used.)

\end{fulllineitems}



\section{pydash\_logger package}
\label{\detokenize{pydash_logger::doc}}\label{\detokenize{pydash_logger:pydash-logger-package}}\label{\detokenize{pydash_logger:module-pydash_logger}}\index{pydash\_logger (module)}

\subsection{Submodules}
\label{\detokenize{pydash_logger:submodules}}

\subsubsection{pydash\_logger.logger module}
\label{\detokenize{pydash_logger.logger::doc}}\label{\detokenize{pydash_logger.logger:module-pydash_logger.logger}}\label{\detokenize{pydash_logger.logger:pydash-logger-logger-module}}\index{pydash\_logger.logger (module)}
Logger object will log messages and errors to date-stamped ‘.log’ files in the /logs directory of the project. Simply
import the class and use it to log messages.
\index{Logger (class in pydash\_logger.logger)}

\begin{fulllineitems}
\phantomsection\label{\detokenize{pydash_logger.logger:pydash_logger.logger.Logger}}\pysiglinewithargsret{\sphinxbfcode{\sphinxupquote{class }}\sphinxcode{\sphinxupquote{pydash\_logger.logger.}}\sphinxbfcode{\sphinxupquote{Logger}}}{\emph{name='pydash\_logger.logger'}}{}
Bases: \sphinxhref{https://docs.python.org/3/library/functions.html\#object}{\sphinxcode{\sphinxupquote{object}}}
\index{debug() (pydash\_logger.logger.Logger method)}

\begin{fulllineitems}
\phantomsection\label{\detokenize{pydash_logger.logger:pydash_logger.logger.Logger.debug}}\pysiglinewithargsret{\sphinxbfcode{\sphinxupquote{debug}}}{\emph{msg}}{}
Takes a message and logs it at the logging.DEBUG level
:param: msg: the message to be logged

\end{fulllineitems}

\index{error() (pydash\_logger.logger.Logger method)}

\begin{fulllineitems}
\phantomsection\label{\detokenize{pydash_logger.logger:pydash_logger.logger.Logger.error}}\pysiglinewithargsret{\sphinxbfcode{\sphinxupquote{error}}}{\emph{msg}}{}
Takes a message and logs it at the logging.ERROR level
:param: msg: the message to be logged

\end{fulllineitems}

\index{info() (pydash\_logger.logger.Logger method)}

\begin{fulllineitems}
\phantomsection\label{\detokenize{pydash_logger.logger:pydash_logger.logger.Logger.info}}\pysiglinewithargsret{\sphinxbfcode{\sphinxupquote{info}}}{\emph{msg}}{}
Takes a message and logs it at the logging.INFO level
:param: msg: the message to be logged

\end{fulllineitems}

\index{warning() (pydash\_logger.logger.Logger method)}

\begin{fulllineitems}
\phantomsection\label{\detokenize{pydash_logger.logger:pydash_logger.logger.Logger.warning}}\pysiglinewithargsret{\sphinxbfcode{\sphinxupquote{warning}}}{\emph{msg}}{}
Takes a message and logs it at the logging.WARN level
:param: msg: the message to be logged

\end{fulllineitems}


\end{fulllineitems}



\section{pydash\_mail package}
\label{\detokenize{pydash_mail::doc}}\label{\detokenize{pydash_mail:module-pydash_mail}}\label{\detokenize{pydash_mail:pydash-mail-package}}\index{pydash\_mail (module)}

\subsection{Submodules}
\label{\detokenize{pydash_mail:submodules}}

\subsubsection{pydash\_mail.templates module}
\label{\detokenize{pydash_mail.templates::doc}}\label{\detokenize{pydash_mail.templates:module-pydash_mail.templates}}\label{\detokenize{pydash_mail.templates:pydash-mail-templates-module}}\index{pydash\_mail.templates (module)}
Reads mail templates into memory and provides functions to format them.
\index{format\_verification\_mail\_html() (in module pydash\_mail.templates)}

\begin{fulllineitems}
\phantomsection\label{\detokenize{pydash_mail.templates:pydash_mail.templates.format_verification_mail_html}}\pysiglinewithargsret{\sphinxcode{\sphinxupquote{pydash\_mail.templates.}}\sphinxbfcode{\sphinxupquote{format\_verification\_mail\_html}}}{\emph{username}, \emph{verification\_url}, \emph{expiration\_date}}{}
Format an HTML verification mail.
:param username: Username to use in the mail.
:param verification\_url: Verification link to use in the mail.
:param expiration\_date: Expiration date of the verification code.
:return: The formatted HTML verification mail.

\end{fulllineitems}

\index{format\_verification\_mail\_plain() (in module pydash\_mail.templates)}

\begin{fulllineitems}
\phantomsection\label{\detokenize{pydash_mail.templates:pydash_mail.templates.format_verification_mail_plain}}\pysiglinewithargsret{\sphinxcode{\sphinxupquote{pydash\_mail.templates.}}\sphinxbfcode{\sphinxupquote{format\_verification\_mail\_plain}}}{\emph{username}, \emph{verification\_url}, \emph{expiration\_date}}{}
Format a plaintext verification mail.
:param username: Username to use in the mail.
:param verification\_url: Verification link to use in the mail.
:param expiration\_date: Expiration date of the verification code.
:return: The formatted plaintext verification mail.

\end{fulllineitems}



\section{pydash\_web package}
\label{\detokenize{pydash_web::doc}}\label{\detokenize{pydash_web:pydash-web-package}}\label{\detokenize{pydash_web:module-pydash_web}}\index{pydash\_web (module)}
Entrypoint of \sphinxtitleref{pydash\_web}

Initializes a Flask web application, and loads the relevant configuration settings.
\index{load\_user() (in module pydash\_web)}

\begin{fulllineitems}
\phantomsection\label{\detokenize{pydash_web:pydash_web.load_user}}\pysiglinewithargsret{\sphinxcode{\sphinxupquote{pydash\_web.}}\sphinxbfcode{\sphinxupquote{load\_user}}}{\emph{user\_id}}{}
\end{fulllineitems}

\index{unauthorized() (in module pydash\_web)}

\begin{fulllineitems}
\phantomsection\label{\detokenize{pydash_web:pydash_web.unauthorized}}\pysiglinewithargsret{\sphinxcode{\sphinxupquote{pydash\_web.}}\sphinxbfcode{\sphinxupquote{unauthorized}}}{}{}
\end{fulllineitems}



\subsection{Subpackages}
\label{\detokenize{pydash_web:subpackages}}

\subsubsection{pydash\_web.controller package}
\label{\detokenize{pydash_web.controller::doc}}\label{\detokenize{pydash_web.controller:module-pydash_web.controller}}\label{\detokenize{pydash_web.controller:pydash-web-controller-package}}\index{pydash\_web.controller (module)}
The controller contains one dispatching function per flask\_webapp endpoint action.


\paragraph{Submodules}
\label{\detokenize{pydash_web.controller:submodules}}

\subparagraph{pydash\_web.controller.change\_dashboard\_settings module}
\label{\detokenize{pydash_web.controller.change_dashboard_settings::doc}}\label{\detokenize{pydash_web.controller.change_dashboard_settings:pydash-web-controller-change-dashboard-settings-module}}\label{\detokenize{pydash_web.controller.change_dashboard_settings:module-pydash_web.controller.change_dashboard_settings}}\index{pydash\_web.controller.change\_dashboard\_settings (module)}
Handles changing dashboard settings.
\index{change\_dashboard\_settings() (in module pydash\_web.controller.change\_dashboard\_settings)}

\begin{fulllineitems}
\phantomsection\label{\detokenize{pydash_web.controller.change_dashboard_settings:pydash_web.controller.change_dashboard_settings.change_dashboard_settings}}\pysiglinewithargsret{\sphinxcode{\sphinxupquote{pydash\_web.controller.change\_dashboard\_settings.}}\sphinxbfcode{\sphinxupquote{change\_dashboard\_settings}}}{\emph{dashboard\_id}}{}
\end{fulllineitems}



\subparagraph{pydash\_web.controller.change\_password module}
\label{\detokenize{pydash_web.controller.change_password::doc}}\label{\detokenize{pydash_web.controller.change_password:module-pydash_web.controller.change_password}}\label{\detokenize{pydash_web.controller.change_password:pydash-web-controller-change-password-module}}\index{pydash\_web.controller.change\_password (module)}
Manages changing of the user’s password.
\index{change\_password() (in module pydash\_web.controller.change\_password)}

\begin{fulllineitems}
\phantomsection\label{\detokenize{pydash_web.controller.change_password:pydash_web.controller.change_password.change_password}}\pysiglinewithargsret{\sphinxcode{\sphinxupquote{pydash\_web.controller.change\_password.}}\sphinxbfcode{\sphinxupquote{change\_password}}}{}{}
\end{fulllineitems}



\subparagraph{pydash\_web.controller.change\_settings module}
\label{\detokenize{pydash_web.controller.change_settings::doc}}\label{\detokenize{pydash_web.controller.change_settings:pydash-web-controller-change-settings-module}}\label{\detokenize{pydash_web.controller.change_settings:module-pydash_web.controller.change_settings}}\index{pydash\_web.controller.change\_settings (module)}
Manages changing of user settings.
\index{change\_settings() (in module pydash\_web.controller.change\_settings)}

\begin{fulllineitems}
\phantomsection\label{\detokenize{pydash_web.controller.change_settings:pydash_web.controller.change_settings.change_settings}}\pysiglinewithargsret{\sphinxcode{\sphinxupquote{pydash\_web.controller.change\_settings.}}\sphinxbfcode{\sphinxupquote{change\_settings}}}{}{}
\end{fulllineitems}



\subparagraph{pydash\_web.controller.dashboards module}
\label{\detokenize{pydash_web.controller.dashboards::doc}}\label{\detokenize{pydash_web.controller.dashboards:pydash-web-controller-dashboards-module}}\label{\detokenize{pydash_web.controller.dashboards:module-pydash_web.controller.dashboards}}\index{pydash\_web.controller.dashboards (module)}
Manages the lookup and returning of dashboard information for a certain user.

Currently only returns static mock data.
\index{check\_allowed\_statistics() (in module pydash\_web.controller.dashboards)}

\begin{fulllineitems}
\phantomsection\label{\detokenize{pydash_web.controller.dashboards:pydash_web.controller.dashboards.check_allowed_statistics}}\pysiglinewithargsret{\sphinxcode{\sphinxupquote{pydash\_web.controller.dashboards.}}\sphinxbfcode{\sphinxupquote{check\_allowed\_statistics}}}{\emph{statistic}}{}
\end{fulllineitems}

\index{check\_allowed\_timeslices() (in module pydash\_web.controller.dashboards)}

\begin{fulllineitems}
\phantomsection\label{\detokenize{pydash_web.controller.dashboards:pydash_web.controller.dashboards.check_allowed_timeslices}}\pysiglinewithargsret{\sphinxcode{\sphinxupquote{pydash\_web.controller.dashboards.}}\sphinxbfcode{\sphinxupquote{check\_allowed\_timeslices}}}{\emph{timeslice}}{}
\end{fulllineitems}

\index{dashboard() (in module pydash\_web.controller.dashboards)}

\begin{fulllineitems}
\phantomsection\label{\detokenize{pydash_web.controller.dashboards:pydash_web.controller.dashboards.dashboard}}\pysiglinewithargsret{\sphinxcode{\sphinxupquote{pydash\_web.controller.dashboards.}}\sphinxbfcode{\sphinxupquote{dashboard}}}{\emph{dashboard\_id}}{}
Lists information of a single dashboard.
:param dashboard\_id: ID of the dashboard to retrieve information from.
:return: The returned value consists of a tuple of dashboard information, together with a http status code.
This route supports the following request arguments:
- statistic: The name of the statistic of which aggregated information should be returned.
\begin{quote}
\begin{description}
\item[{The currently supported statistics are:}] \leavevmode\begin{itemize}
\item {} 
total\_visits

\item {} 
total\_execution\_time

\item {} 
average\_execution\_time

\item {} 
visits\_per\_ip

\item {} 
unique\_visitors

\item {} 
fastest\_measured\_execution\_time

\item {} 
fastest\_quartile\_execution\_time

\item {} 
median\_execution\_time

\item {} 
slowest\_quartile\_execution\_time

\item {} 
ninetieth\_percentile\_execution\_time

\item {} 
ninety-ninth\_percentile\_execution\_time

\item {} 
slowest\_measured\_execution\_time

\end{itemize}

\end{description}
\end{quote}
\begin{itemize}
\item {} \begin{description}
\item[{start\_date, end\_date: The start- and end dates of the datetime range in which the desired information lies.}] \leavevmode
Both start\_date and end\_date are inclusive resp. upper- and lower bounds of this datetime range.
If start\_date is not provided, it defaults to 1970-1-1.
If end\_date is not provided, it defaults to the current utc time.

It is assumed both start\_date and end\_date are provided in utc time.

\end{description}

\item {} \begin{description}
\item[{granularity: Since end\_date is inclusive, a time granularity is required in order to determine how much time from}] \leavevmode
end\_date on should be included as well. The possibilities here are: ‘year’, ‘month’, ‘week’, ‘day’, ‘hour’ and ‘minute’.
If granularity is not privided, it defaults to ‘day’.

\end{description}

\item {} \begin{description}
\item[{timeslice: Indicates the data should be returned as a series of points in time, each ‘timeslice’ long.}] \leavevmode
‘timeslice’ overrules ‘granularity’ in terms of granularity.

\end{description}

\end{itemize}
\begin{description}
\item[{If ‘timeslice’ is absent, a the returned information is a single value. When it is not, a dictionary is returned,}] \leavevmode
containing datetime-value pairs, where ‘datetime’ is formatted to the granularity of ‘timeslice’.
(e.g. ‘timeslice=day’ will result in datetimes like ‘2018-05-29’, while ‘timeslice=minute’ will result in
datetimes like ‘2018-05-29T15:45’)

\end{description}

Note that if the dashboard has not yet received any endpoint calls, it will simply return an empty dictionary.

\end{fulllineitems}

\index{dashboards() (in module pydash\_web.controller.dashboards)}

\begin{fulllineitems}
\phantomsection\label{\detokenize{pydash_web.controller.dashboards:pydash_web.controller.dashboards.dashboards}}\pysiglinewithargsret{\sphinxcode{\sphinxupquote{pydash\_web.controller.dashboards.}}\sphinxbfcode{\sphinxupquote{dashboards}}}{}{}
Lists the dashboards of the current user.
:return: A tuple containing:
\begin{itemize}
\item {} 
A list of dicts, containing dashboard details of the current user’s dashboards.
or
A dict containing an error message describing the particular error.

\item {} 
A corresponding HTML status code.

\end{itemize}

\end{fulllineitems}

\index{handle\_statistic\_per\_timeslice() (in module pydash\_web.controller.dashboards)}

\begin{fulllineitems}
\phantomsection\label{\detokenize{pydash_web.controller.dashboards:pydash_web.controller.dashboards.handle_statistic_per_timeslice}}\pysiglinewithargsret{\sphinxcode{\sphinxupquote{pydash\_web.controller.dashboards.}}\sphinxbfcode{\sphinxupquote{handle\_statistic\_per\_timeslice}}}{\emph{dashboard}, \emph{statistic}, \emph{timeslice}, \emph{start\_datetime}, \emph{end\_datetime}}{}
These datetimes are treated as inclusive boundaries of a datetime range (e.g. {[}start\_datetime, end\_datetime{]}.
Assumes start\_timedate and end\_timedate are both timezone aware, with timezone utc.
:param dashboard:
:param statistic:
:param timeslice:
:param start\_datetime:
:param end\_datetime:
:return: A dictionary consisting of a datetime string (key)(formatted according to the ISO-8601 standard)
\begin{quote}

and the corresponding statistic, over the specified datetime range.
\end{quote}

\end{fulllineitems}

\index{handle\_statistic\_without\_timeslice() (in module pydash\_web.controller.dashboards)}

\begin{fulllineitems}
\phantomsection\label{\detokenize{pydash_web.controller.dashboards:pydash_web.controller.dashboards.handle_statistic_without_timeslice}}\pysiglinewithargsret{\sphinxcode{\sphinxupquote{pydash\_web.controller.dashboards.}}\sphinxbfcode{\sphinxupquote{handle\_statistic\_without\_timeslice}}}{\emph{dashboard}, \emph{statistic}, \emph{start\_datetime}, \emph{end\_datetime}, \emph{granularity}}{}
These datetimes are treated as inclusive boundaries of a datetime range (e.g. {[}start\_datetime, end\_datetime{]}
:param dashboard:
:param statistic:
:param start\_datetime:
:param end\_datetime:
:param granularity:
:return: The value of a single statistic over the specified datetime range.

\end{fulllineitems}

\index{match\_datetime\_string\_with\_formats() (in module pydash\_web.controller.dashboards)}

\begin{fulllineitems}
\phantomsection\label{\detokenize{pydash_web.controller.dashboards:pydash_web.controller.dashboards.match_datetime_string_with_formats}}\pysiglinewithargsret{\sphinxcode{\sphinxupquote{pydash\_web.controller.dashboards.}}\sphinxbfcode{\sphinxupquote{match\_datetime\_string\_with\_formats}}}{\emph{datetime\_string}}{}
Returns a datetime object of this datetime string if the provided string matched with
one of the allowed formats. Otherwise, returns None and None.

\end{fulllineitems}



\subparagraph{pydash\_web.controller.delete\_dashboard module}
\label{\detokenize{pydash_web.controller.delete_dashboard::doc}}\label{\detokenize{pydash_web.controller.delete_dashboard:pydash-web-controller-delete-dashboard-module}}\label{\detokenize{pydash_web.controller.delete_dashboard:module-pydash_web.controller.delete_dashboard}}\index{pydash\_web.controller.delete\_dashboard (module)}
Manages the deletion of a dashboard.
\index{delete\_dashboard() (in module pydash\_web.controller.delete\_dashboard)}

\begin{fulllineitems}
\phantomsection\label{\detokenize{pydash_web.controller.delete_dashboard:pydash_web.controller.delete_dashboard.delete_dashboard}}\pysiglinewithargsret{\sphinxcode{\sphinxupquote{pydash\_web.controller.delete\_dashboard.}}\sphinxbfcode{\sphinxupquote{delete\_dashboard}}}{\emph{dashboard\_id}}{}
\end{fulllineitems}



\subparagraph{pydash\_web.controller.delete\_user module}
\label{\detokenize{pydash_web.controller.delete_user::doc}}\label{\detokenize{pydash_web.controller.delete_user:module-pydash_web.controller.delete_user}}\label{\detokenize{pydash_web.controller.delete_user:pydash-web-controller-delete-user-module}}\index{pydash\_web.controller.delete\_user (module)}
Manages deletion of a user.
\index{delete\_user() (in module pydash\_web.controller.delete\_user)}

\begin{fulllineitems}
\phantomsection\label{\detokenize{pydash_web.controller.delete_user:pydash_web.controller.delete_user.delete_user}}\pysiglinewithargsret{\sphinxcode{\sphinxupquote{pydash\_web.controller.delete\_user.}}\sphinxbfcode{\sphinxupquote{delete\_user}}}{}{}
Deletes the currently logged in user and all dashboards they own.

\end{fulllineitems}



\subparagraph{pydash\_web.controller.execution\_times\_boxplots module}
\label{\detokenize{pydash_web.controller.execution_times_boxplots::doc}}\label{\detokenize{pydash_web.controller.execution_times_boxplots:module-pydash_web.controller.execution_times_boxplots}}\label{\detokenize{pydash_web.controller.execution_times_boxplots:pydash-web-controller-execution-times-boxplots-module}}\index{pydash\_web.controller.execution\_times\_boxplots (module)}\index{endpoint\_execution\_times\_boxplots() (in module pydash\_web.controller.execution\_times\_boxplots)}

\begin{fulllineitems}
\phantomsection\label{\detokenize{pydash_web.controller.execution_times_boxplots:pydash_web.controller.execution_times_boxplots.endpoint_execution_times_boxplots}}\pysiglinewithargsret{\sphinxcode{\sphinxupquote{pydash\_web.controller.execution\_times\_boxplots.}}\sphinxbfcode{\sphinxupquote{endpoint\_execution\_times\_boxplots}}}{\emph{dashboard\_id}, \emph{endpoint\_name=None}}{}
\end{fulllineitems}



\subparagraph{pydash\_web.controller.execution\_times\_per\_version module}
\label{\detokenize{pydash_web.controller.execution_times_per_version::doc}}\label{\detokenize{pydash_web.controller.execution_times_per_version:module-pydash_web.controller.execution_times_per_version}}\label{\detokenize{pydash_web.controller.execution_times_per_version:pydash-web-controller-execution-times-per-version-module}}\index{pydash\_web.controller.execution\_times\_per\_version (module)}
Handles requests for tdigest data of response times per version.
\index{execution\_times\_per\_version() (in module pydash\_web.controller.execution\_times\_per\_version)}

\begin{fulllineitems}
\phantomsection\label{\detokenize{pydash_web.controller.execution_times_per_version:pydash_web.controller.execution_times_per_version.execution_times_per_version}}\pysiglinewithargsret{\sphinxcode{\sphinxupquote{pydash\_web.controller.execution\_times\_per\_version.}}\sphinxbfcode{\sphinxupquote{execution\_times\_per\_version}}}{\emph{dashboard\_id}, \emph{endpoint\_name=None}}{}
\end{fulllineitems}



\subparagraph{pydash\_web.controller.login module}
\label{\detokenize{pydash_web.controller.login::doc}}\label{\detokenize{pydash_web.controller.login:module-pydash_web.controller.login}}\label{\detokenize{pydash_web.controller.login:pydash-web-controller-login-module}}\index{pydash\_web.controller.login (module)}
Manages the logging in of a user into the application,
and rejecting visitors that enter improper sign-in information or have not been verified yet.
\index{login() (in module pydash\_web.controller.login)}

\begin{fulllineitems}
\phantomsection\label{\detokenize{pydash_web.controller.login:pydash_web.controller.login.login}}\pysiglinewithargsret{\sphinxcode{\sphinxupquote{pydash\_web.controller.login.}}\sphinxbfcode{\sphinxupquote{login}}}{}{}
\end{fulllineitems}



\subparagraph{pydash\_web.controller.logout module}
\label{\detokenize{pydash_web.controller.logout::doc}}\label{\detokenize{pydash_web.controller.logout:module-pydash_web.controller.logout}}\label{\detokenize{pydash_web.controller.logout:pydash-web-controller-logout-module}}\index{pydash\_web.controller.logout (module)}
Allows a user to sign out again after finishing using the application
\index{logout() (in module pydash\_web.controller.logout)}

\begin{fulllineitems}
\phantomsection\label{\detokenize{pydash_web.controller.logout:pydash_web.controller.logout.logout}}\pysiglinewithargsret{\sphinxcode{\sphinxupquote{pydash\_web.controller.logout.}}\sphinxbfcode{\sphinxupquote{logout}}}{}{}
\end{fulllineitems}



\subparagraph{pydash\_web.controller.register\_dashboard module}
\label{\detokenize{pydash_web.controller.register_dashboard::doc}}\label{\detokenize{pydash_web.controller.register_dashboard:module-pydash_web.controller.register_dashboard}}\label{\detokenize{pydash_web.controller.register_dashboard:pydash-web-controller-register-dashboard-module}}\index{pydash\_web.controller.register\_dashboard (module)}\index{register\_dashboard() (in module pydash\_web.controller.register\_dashboard)}

\begin{fulllineitems}
\phantomsection\label{\detokenize{pydash_web.controller.register_dashboard:pydash_web.controller.register_dashboard.register_dashboard}}\pysiglinewithargsret{\sphinxcode{\sphinxupquote{pydash\_web.controller.register\_dashboard.}}\sphinxbfcode{\sphinxupquote{register\_dashboard}}}{}{}
\end{fulllineitems}



\subparagraph{pydash\_web.controller.register\_user module}
\label{\detokenize{pydash_web.controller.register_user::doc}}\label{\detokenize{pydash_web.controller.register_user:module-pydash_web.controller.register_user}}\label{\detokenize{pydash_web.controller.register_user:pydash-web-controller-register-user-module}}\index{pydash\_web.controller.register\_user (module)}
Manages the registration of a new user.
\index{register\_user() (in module pydash\_web.controller.register\_user)}

\begin{fulllineitems}
\phantomsection\label{\detokenize{pydash_web.controller.register_user:pydash_web.controller.register_user.register_user}}\pysiglinewithargsret{\sphinxcode{\sphinxupquote{pydash\_web.controller.register\_user.}}\sphinxbfcode{\sphinxupquote{register\_user}}}{}{}
\end{fulllineitems}



\subparagraph{pydash\_web.controller.user\_verification module}
\label{\detokenize{pydash_web.controller.user_verification::doc}}\label{\detokenize{pydash_web.controller.user_verification:pydash-web-controller-user-verification-module}}\label{\detokenize{pydash_web.controller.user_verification:module-pydash_web.controller.user_verification}}\index{pydash\_web.controller.user\_verification (module)}
Manages the verification of a User.
\index{verify\_user() (in module pydash\_web.controller.user\_verification)}

\begin{fulllineitems}
\phantomsection\label{\detokenize{pydash_web.controller.user_verification:pydash_web.controller.user_verification.verify_user}}\pysiglinewithargsret{\sphinxcode{\sphinxupquote{pydash\_web.controller.user\_verification.}}\sphinxbfcode{\sphinxupquote{verify\_user}}}{}{}
Verifies the currently logged in User by comparing the given verification\_code with the code assigned to the User.
This is intended to be used only once, after the user has just registered their account in order to gain access to
api-routes that have the \sphinxtitleref{verification\_required} decorator.

\end{fulllineitems}



\subparagraph{pydash\_web.controller.utils module}
\label{\detokenize{pydash_web.controller.utils::doc}}\label{\detokenize{pydash_web.controller.utils:module-pydash_web.controller.utils}}\label{\detokenize{pydash_web.controller.utils:pydash-web-controller-utils-module}}\index{pydash\_web.controller.utils (module)}
The go-to place for general methods that can be used in multiple controller methods.
\index{execution\_times() (in module pydash\_web.controller.utils)}

\begin{fulllineitems}
\phantomsection\label{\detokenize{pydash_web.controller.utils:pydash_web.controller.utils.execution_times}}\pysiglinewithargsret{\sphinxcode{\sphinxupquote{pydash\_web.controller.utils.}}\sphinxbfcode{\sphinxupquote{execution\_times}}}{\emph{aggregator\_group\_container}, \emph{filters=\{\}}}{}
\end{fulllineitems}



\subparagraph{pydash\_web.controller.visitor\_heatmap module}
\label{\detokenize{pydash_web.controller.visitor_heatmap::doc}}\label{\detokenize{pydash_web.controller.visitor_heatmap:module-pydash_web.controller.visitor_heatmap}}\label{\detokenize{pydash_web.controller.visitor_heatmap:pydash-web-controller-visitor-heatmap-module}}\index{pydash\_web.controller.visitor\_heatmap (module)}\index{daterange() (in module pydash\_web.controller.visitor\_heatmap)}

\begin{fulllineitems}
\phantomsection\label{\detokenize{pydash_web.controller.visitor_heatmap:pydash_web.controller.visitor_heatmap.daterange}}\pysiglinewithargsret{\sphinxcode{\sphinxupquote{pydash\_web.controller.visitor\_heatmap.}}\sphinxbfcode{\sphinxupquote{daterange}}}{\emph{start\_date}, \emph{end\_date}}{}
\end{fulllineitems}

\index{get\_hourly\_data() (in module pydash\_web.controller.visitor\_heatmap)}

\begin{fulllineitems}
\phantomsection\label{\detokenize{pydash_web.controller.visitor_heatmap:pydash_web.controller.visitor_heatmap.get_hourly_data}}\pysiglinewithargsret{\sphinxcode{\sphinxupquote{pydash\_web.controller.visitor\_heatmap.}}\sphinxbfcode{\sphinxupquote{get\_hourly\_data}}}{\emph{dashboard}, \emph{day}, \emph{field}}{}
\end{fulllineitems}

\index{visitor\_heatmap() (in module pydash\_web.controller.visitor\_heatmap)}

\begin{fulllineitems}
\phantomsection\label{\detokenize{pydash_web.controller.visitor_heatmap:pydash_web.controller.visitor_heatmap.visitor_heatmap}}\pysiglinewithargsret{\sphinxcode{\sphinxupquote{pydash\_web.controller.visitor\_heatmap.}}\sphinxbfcode{\sphinxupquote{visitor\_heatmap}}}{\emph{dashboard\_id}, \emph{field='total\_visits'}}{}
\end{fulllineitems}



\subsection{Submodules}
\label{\detokenize{pydash_web:submodules}}

\subsubsection{pydash\_web.api module}
\label{\detokenize{pydash_web.api::doc}}\label{\detokenize{pydash_web.api:module-pydash_web.api}}\label{\detokenize{pydash_web.api:pydash-web-api-module}}\index{pydash\_web.api (module)}
Serves as a blueprint for the entire pydash\_web package.
url\_for() calls within this package should prepend ‘pydash\_web.’ to their input argument.
\begin{quote}

{[}e.g. url\_for(login) becomes url\_for(pydash\_web.login) {]}
\end{quote}

route decorators in this package should also use this blueprint object instead of the flask application object.


\subsubsection{pydash\_web.api\_routes module}
\label{\detokenize{pydash_web.api_routes::doc}}\label{\detokenize{pydash_web.api_routes:pydash-web-api-routes-module}}\label{\detokenize{pydash_web.api_routes:module-pydash_web.api_routes}}\index{pydash\_web.api\_routes (module)}
Contains the different routes (web endpoints) that the pydash\_web flask application can respond to.

The actual implementation of each of the routes’ dispatching logic is handled by the respective ‘controller’ function.
\index{change\_dashboard\_settings() (in module pydash\_web.api\_routes)}

\begin{fulllineitems}
\phantomsection\label{\detokenize{pydash_web.api_routes:pydash_web.api_routes.change_dashboard_settings}}\pysiglinewithargsret{\sphinxcode{\sphinxupquote{pydash\_web.api\_routes.}}\sphinxbfcode{\sphinxupquote{change\_dashboard\_settings}}}{\emph{dashboard\_id}}{}
\end{fulllineitems}

\index{change\_password() (in module pydash\_web.api\_routes)}

\begin{fulllineitems}
\phantomsection\label{\detokenize{pydash_web.api_routes:pydash_web.api_routes.change_password}}\pysiglinewithargsret{\sphinxcode{\sphinxupquote{pydash\_web.api\_routes.}}\sphinxbfcode{\sphinxupquote{change\_password}}}{}{}
\end{fulllineitems}

\index{change\_settings() (in module pydash\_web.api\_routes)}

\begin{fulllineitems}
\phantomsection\label{\detokenize{pydash_web.api_routes:pydash_web.api_routes.change_settings}}\pysiglinewithargsret{\sphinxcode{\sphinxupquote{pydash\_web.api\_routes.}}\sphinxbfcode{\sphinxupquote{change\_settings}}}{}{}
\end{fulllineitems}

\index{delete\_dashboard() (in module pydash\_web.api\_routes)}

\begin{fulllineitems}
\phantomsection\label{\detokenize{pydash_web.api_routes:pydash_web.api_routes.delete_dashboard}}\pysiglinewithargsret{\sphinxcode{\sphinxupquote{pydash\_web.api\_routes.}}\sphinxbfcode{\sphinxupquote{delete\_dashboard}}}{\emph{dashboard\_id}}{}
\end{fulllineitems}

\index{delete\_user() (in module pydash\_web.api\_routes)}

\begin{fulllineitems}
\phantomsection\label{\detokenize{pydash_web.api_routes:pydash_web.api_routes.delete_user}}\pysiglinewithargsret{\sphinxcode{\sphinxupquote{pydash\_web.api\_routes.}}\sphinxbfcode{\sphinxupquote{delete\_user}}}{}{}
\end{fulllineitems}

\index{get\_dashboard() (in module pydash\_web.api\_routes)}

\begin{fulllineitems}
\phantomsection\label{\detokenize{pydash_web.api_routes:pydash_web.api_routes.get_dashboard}}\pysiglinewithargsret{\sphinxcode{\sphinxupquote{pydash\_web.api\_routes.}}\sphinxbfcode{\sphinxupquote{get\_dashboard}}}{\emph{dashboard\_id}}{}
\end{fulllineitems}

\index{get\_dashboards() (in module pydash\_web.api\_routes)}

\begin{fulllineitems}
\phantomsection\label{\detokenize{pydash_web.api_routes:pydash_web.api_routes.get_dashboards}}\pysiglinewithargsret{\sphinxcode{\sphinxupquote{pydash\_web.api\_routes.}}\sphinxbfcode{\sphinxupquote{get\_dashboards}}}{}{}
\end{fulllineitems}

\index{get\_endpoint\_execution\_times\_boxplots() (in module pydash\_web.api\_routes)}

\begin{fulllineitems}
\phantomsection\label{\detokenize{pydash_web.api_routes:pydash_web.api_routes.get_endpoint_execution_times_boxplots}}\pysiglinewithargsret{\sphinxcode{\sphinxupquote{pydash\_web.api\_routes.}}\sphinxbfcode{\sphinxupquote{get\_endpoint\_execution\_times\_boxplots}}}{\emph{dashboard\_id}}{}
\end{fulllineitems}

\index{get\_execution\_times\_boxplot() (in module pydash\_web.api\_routes)}

\begin{fulllineitems}
\phantomsection\label{\detokenize{pydash_web.api_routes:pydash_web.api_routes.get_execution_times_boxplot}}\pysiglinewithargsret{\sphinxcode{\sphinxupquote{pydash\_web.api\_routes.}}\sphinxbfcode{\sphinxupquote{get\_execution\_times\_boxplot}}}{\emph{dashboard\_id}, \emph{endpoint\_name}}{}
\end{fulllineitems}

\index{get\_execution\_times\_per\_version\_dashboard() (in module pydash\_web.api\_routes)}

\begin{fulllineitems}
\phantomsection\label{\detokenize{pydash_web.api_routes:pydash_web.api_routes.get_execution_times_per_version_dashboard}}\pysiglinewithargsret{\sphinxcode{\sphinxupquote{pydash\_web.api\_routes.}}\sphinxbfcode{\sphinxupquote{get\_execution\_times\_per\_version\_dashboard}}}{\emph{dashboard\_id}}{}
\end{fulllineitems}

\index{get\_execution\_times\_per\_version\_endpoint() (in module pydash\_web.api\_routes)}

\begin{fulllineitems}
\phantomsection\label{\detokenize{pydash_web.api_routes:pydash_web.api_routes.get_execution_times_per_version_endpoint}}\pysiglinewithargsret{\sphinxcode{\sphinxupquote{pydash\_web.api\_routes.}}\sphinxbfcode{\sphinxupquote{get\_execution\_times\_per\_version\_endpoint}}}{\emph{dashboard\_id}, \emph{endpoint\_name}}{}
\end{fulllineitems}

\index{get\_unique\_visitor\_heatmap() (in module pydash\_web.api\_routes)}

\begin{fulllineitems}
\phantomsection\label{\detokenize{pydash_web.api_routes:pydash_web.api_routes.get_unique_visitor_heatmap}}\pysiglinewithargsret{\sphinxcode{\sphinxupquote{pydash\_web.api\_routes.}}\sphinxbfcode{\sphinxupquote{get\_unique\_visitor\_heatmap}}}{\emph{dashboard\_id}}{}
\end{fulllineitems}

\index{get\_visitor\_heatmap() (in module pydash\_web.api\_routes)}

\begin{fulllineitems}
\phantomsection\label{\detokenize{pydash_web.api_routes:pydash_web.api_routes.get_visitor_heatmap}}\pysiglinewithargsret{\sphinxcode{\sphinxupquote{pydash\_web.api\_routes.}}\sphinxbfcode{\sphinxupquote{get\_visitor\_heatmap}}}{\emph{dashboard\_id}}{}
\end{fulllineitems}

\index{login() (in module pydash\_web.api\_routes)}

\begin{fulllineitems}
\phantomsection\label{\detokenize{pydash_web.api_routes:pydash_web.api_routes.login}}\pysiglinewithargsret{\sphinxcode{\sphinxupquote{pydash\_web.api\_routes.}}\sphinxbfcode{\sphinxupquote{login}}}{}{}
\end{fulllineitems}

\index{logout() (in module pydash\_web.api\_routes)}

\begin{fulllineitems}
\phantomsection\label{\detokenize{pydash_web.api_routes:pydash_web.api_routes.logout}}\pysiglinewithargsret{\sphinxcode{\sphinxupquote{pydash\_web.api\_routes.}}\sphinxbfcode{\sphinxupquote{logout}}}{}{}
\end{fulllineitems}

\index{register\_dashboard() (in module pydash\_web.api\_routes)}

\begin{fulllineitems}
\phantomsection\label{\detokenize{pydash_web.api_routes:pydash_web.api_routes.register_dashboard}}\pysiglinewithargsret{\sphinxcode{\sphinxupquote{pydash\_web.api\_routes.}}\sphinxbfcode{\sphinxupquote{register\_dashboard}}}{}{}
\end{fulllineitems}

\index{register\_user() (in module pydash\_web.api\_routes)}

\begin{fulllineitems}
\phantomsection\label{\detokenize{pydash_web.api_routes:pydash_web.api_routes.register_user}}\pysiglinewithargsret{\sphinxcode{\sphinxupquote{pydash\_web.api\_routes.}}\sphinxbfcode{\sphinxupquote{register\_user}}}{}{}
\end{fulllineitems}

\index{verify\_user() (in module pydash\_web.api\_routes)}

\begin{fulllineitems}
\phantomsection\label{\detokenize{pydash_web.api_routes:pydash_web.api_routes.verify_user}}\pysiglinewithargsret{\sphinxcode{\sphinxupquote{pydash\_web.api\_routes.}}\sphinxbfcode{\sphinxupquote{verify\_user}}}{}{}
\end{fulllineitems}



\subsubsection{pydash\_web.react\_server module}
\label{\detokenize{pydash_web.react_server::doc}}\label{\detokenize{pydash_web.react_server:module-pydash_web.react_server}}\label{\detokenize{pydash_web.react_server:pydash-web-react-server-module}}\index{pydash\_web.react\_server (module)}\index{serve() (in module pydash\_web.react\_server)}

\begin{fulllineitems}
\phantomsection\label{\detokenize{pydash_web.react_server:pydash_web.react_server.serve}}\pysiglinewithargsret{\sphinxcode{\sphinxupquote{pydash\_web.react\_server.}}\sphinxbfcode{\sphinxupquote{serve}}}{\emph{path}}{}
\end{fulllineitems}



\chapter{Indices and tables}
\label{\detokenize{index:indices-and-tables}}\begin{itemize}
\item {} 
\DUrole{xref,std,std-ref}{genindex}

\item {} 
\DUrole{xref,std,std-ref}{modindex}

\item {} 
\DUrole{xref,std,std-ref}{search}

\end{itemize}


\renewcommand{\indexname}{Python Module Index}
\begin{sphinxtheindex}
\def\bigletter#1{{\Large\sffamily#1}\nopagebreak\vspace{1mm}}
\bigletter{f}
\item {\sphinxstyleindexentry{flask\_monitoring\_dashboard\_client}}\sphinxstyleindexpageref{flask_monitoring_dashboard_client:\detokenize{module-flask_monitoring_dashboard_client}}
\indexspace
\bigletter{p}
\item {\sphinxstyleindexentry{periodic\_tasks}}\sphinxstyleindexpageref{periodic_tasks:\detokenize{module-periodic_tasks}}
\item {\sphinxstyleindexentry{periodic\_tasks.pqdict\_iter\_upto\_priority}}\sphinxstyleindexpageref{periodic_tasks.pqdict_iter_upto_priority:\detokenize{module-periodic_tasks.pqdict_iter_upto_priority}}
\item {\sphinxstyleindexentry{periodic\_tasks.queue\_nonblocking\_iter}}\sphinxstyleindexpageref{periodic_tasks.queue_nonblocking_iter:\detokenize{module-periodic_tasks.queue_nonblocking_iter}}
\item {\sphinxstyleindexentry{periodic\_tasks.task\_scheduler}}\sphinxstyleindexpageref{periodic_tasks.task_scheduler:\detokenize{module-periodic_tasks.task_scheduler}}
\item {\sphinxstyleindexentry{pydash}}\sphinxstyleindexpageref{pydash:\detokenize{module-pydash}}
\item {\sphinxstyleindexentry{pydash\_app}}\sphinxstyleindexpageref{pydash_app:\detokenize{module-pydash_app}}
\item {\sphinxstyleindexentry{pydash\_app.dashboard}}\sphinxstyleindexpageref{pydash_app.dashboard:\detokenize{module-pydash_app.dashboard}}
\item {\sphinxstyleindexentry{pydash\_app.dashboard.aggregator}}\sphinxstyleindexpageref{pydash_app.dashboard.aggregator:\detokenize{module-pydash_app.dashboard.aggregator}}
\item {\sphinxstyleindexentry{pydash\_app.dashboard.aggregator.aggregator\_group}}\sphinxstyleindexpageref{pydash_app.dashboard.aggregator.aggregator_group:\detokenize{module-pydash_app.dashboard.aggregator.aggregator_group}}
\item {\sphinxstyleindexentry{pydash\_app.dashboard.aggregator.statistics}}\sphinxstyleindexpageref{pydash_app.dashboard.aggregator.statistics:\detokenize{module-pydash_app.dashboard.aggregator.statistics}}
\item {\sphinxstyleindexentry{pydash\_app.dashboard.endpoint}}\sphinxstyleindexpageref{pydash_app.dashboard.endpoint:\detokenize{module-pydash_app.dashboard.endpoint}}
\item {\sphinxstyleindexentry{pydash\_app.dashboard.endpoint\_call}}\sphinxstyleindexpageref{pydash_app.dashboard.endpoint_call:\detokenize{module-pydash_app.dashboard.endpoint_call}}
\item {\sphinxstyleindexentry{pydash\_app.dashboard.entity}}\sphinxstyleindexpageref{pydash_app.dashboard.entity:\detokenize{module-pydash_app.dashboard.entity}}
\item {\sphinxstyleindexentry{pydash\_app.dashboard.repository}}\sphinxstyleindexpageref{pydash_app.dashboard.repository:\detokenize{module-pydash_app.dashboard.repository}}
\item {\sphinxstyleindexentry{pydash\_app.dashboard.services}}\sphinxstyleindexpageref{pydash_app.dashboard.services:\detokenize{module-pydash_app.dashboard.services}}
\item {\sphinxstyleindexentry{pydash\_app.dashboard.services.fetching}}\sphinxstyleindexpageref{pydash_app.dashboard.services.fetching:\detokenize{module-pydash_app.dashboard.services.fetching}}
\item {\sphinxstyleindexentry{pydash\_app.dashboard.services.seeding}}\sphinxstyleindexpageref{pydash_app.dashboard.services.seeding:\detokenize{module-pydash_app.dashboard.services.seeding}}
\item {\sphinxstyleindexentry{pydash\_app.user}}\sphinxstyleindexpageref{pydash_app.user:\detokenize{module-pydash_app.user}}
\item {\sphinxstyleindexentry{pydash\_app.user.entity}}\sphinxstyleindexpageref{pydash_app.user.entity:\detokenize{module-pydash_app.user.entity}}
\item {\sphinxstyleindexentry{pydash\_app.user.repository}}\sphinxstyleindexpageref{pydash_app.user.repository:\detokenize{module-pydash_app.user.repository}}
\item {\sphinxstyleindexentry{pydash\_app.user.services}}\sphinxstyleindexpageref{pydash_app.user.services:\detokenize{module-pydash_app.user.services}}
\item {\sphinxstyleindexentry{pydash\_app.user.services.pruning}}\sphinxstyleindexpageref{pydash_app.user.services.pruning:\detokenize{module-pydash_app.user.services.pruning}}
\item {\sphinxstyleindexentry{pydash\_app.user.services.seeding}}\sphinxstyleindexpageref{pydash_app.user.services.seeding:\detokenize{module-pydash_app.user.services.seeding}}
\item {\sphinxstyleindexentry{pydash\_app.user.verification}}\sphinxstyleindexpageref{pydash_app.user.verification:\detokenize{module-pydash_app.user.verification}}
\item {\sphinxstyleindexentry{pydash\_app.user.verification\_code}}\sphinxstyleindexpageref{pydash_app.user.verification_code:\detokenize{module-pydash_app.user.verification_code}}
\item {\sphinxstyleindexentry{pydash\_database}}\sphinxstyleindexpageref{pydash_database:\detokenize{module-pydash_database}}
\item {\sphinxstyleindexentry{pydash\_logger}}\sphinxstyleindexpageref{pydash_logger:\detokenize{module-pydash_logger}}
\item {\sphinxstyleindexentry{pydash\_logger.logger}}\sphinxstyleindexpageref{pydash_logger.logger:\detokenize{module-pydash_logger.logger}}
\item {\sphinxstyleindexentry{pydash\_mail}}\sphinxstyleindexpageref{pydash_mail:\detokenize{module-pydash_mail}}
\item {\sphinxstyleindexentry{pydash\_mail.templates}}\sphinxstyleindexpageref{pydash_mail.templates:\detokenize{module-pydash_mail.templates}}
\item {\sphinxstyleindexentry{pydash\_web}}\sphinxstyleindexpageref{pydash_web:\detokenize{module-pydash_web}}
\item {\sphinxstyleindexentry{pydash\_web.api}}\sphinxstyleindexpageref{pydash_web.api:\detokenize{module-pydash_web.api}}
\item {\sphinxstyleindexentry{pydash\_web.api\_routes}}\sphinxstyleindexpageref{pydash_web.api_routes:\detokenize{module-pydash_web.api_routes}}
\item {\sphinxstyleindexentry{pydash\_web.controller}}\sphinxstyleindexpageref{pydash_web.controller:\detokenize{module-pydash_web.controller}}
\item {\sphinxstyleindexentry{pydash\_web.controller.change\_dashboard\_settings}}\sphinxstyleindexpageref{pydash_web.controller.change_dashboard_settings:\detokenize{module-pydash_web.controller.change_dashboard_settings}}
\item {\sphinxstyleindexentry{pydash\_web.controller.change\_password}}\sphinxstyleindexpageref{pydash_web.controller.change_password:\detokenize{module-pydash_web.controller.change_password}}
\item {\sphinxstyleindexentry{pydash\_web.controller.change\_settings}}\sphinxstyleindexpageref{pydash_web.controller.change_settings:\detokenize{module-pydash_web.controller.change_settings}}
\item {\sphinxstyleindexentry{pydash\_web.controller.dashboards}}\sphinxstyleindexpageref{pydash_web.controller.dashboards:\detokenize{module-pydash_web.controller.dashboards}}
\item {\sphinxstyleindexentry{pydash\_web.controller.delete\_dashboard}}\sphinxstyleindexpageref{pydash_web.controller.delete_dashboard:\detokenize{module-pydash_web.controller.delete_dashboard}}
\item {\sphinxstyleindexentry{pydash\_web.controller.delete\_user}}\sphinxstyleindexpageref{pydash_web.controller.delete_user:\detokenize{module-pydash_web.controller.delete_user}}
\item {\sphinxstyleindexentry{pydash\_web.controller.execution\_times\_boxplots}}\sphinxstyleindexpageref{pydash_web.controller.execution_times_boxplots:\detokenize{module-pydash_web.controller.execution_times_boxplots}}
\item {\sphinxstyleindexentry{pydash\_web.controller.execution\_times\_per\_version}}\sphinxstyleindexpageref{pydash_web.controller.execution_times_per_version:\detokenize{module-pydash_web.controller.execution_times_per_version}}
\item {\sphinxstyleindexentry{pydash\_web.controller.login}}\sphinxstyleindexpageref{pydash_web.controller.login:\detokenize{module-pydash_web.controller.login}}
\item {\sphinxstyleindexentry{pydash\_web.controller.logout}}\sphinxstyleindexpageref{pydash_web.controller.logout:\detokenize{module-pydash_web.controller.logout}}
\item {\sphinxstyleindexentry{pydash\_web.controller.register\_dashboard}}\sphinxstyleindexpageref{pydash_web.controller.register_dashboard:\detokenize{module-pydash_web.controller.register_dashboard}}
\item {\sphinxstyleindexentry{pydash\_web.controller.register\_user}}\sphinxstyleindexpageref{pydash_web.controller.register_user:\detokenize{module-pydash_web.controller.register_user}}
\item {\sphinxstyleindexentry{pydash\_web.controller.user\_verification}}\sphinxstyleindexpageref{pydash_web.controller.user_verification:\detokenize{module-pydash_web.controller.user_verification}}
\item {\sphinxstyleindexentry{pydash\_web.controller.utils}}\sphinxstyleindexpageref{pydash_web.controller.utils:\detokenize{module-pydash_web.controller.utils}}
\item {\sphinxstyleindexentry{pydash\_web.controller.visitor\_heatmap}}\sphinxstyleindexpageref{pydash_web.controller.visitor_heatmap:\detokenize{module-pydash_web.controller.visitor_heatmap}}
\item {\sphinxstyleindexentry{pydash\_web.react\_server}}\sphinxstyleindexpageref{pydash_web.react_server:\detokenize{module-pydash_web.react_server}}
\end{sphinxtheindex}

\renewcommand{\indexname}{Index}
\printindex
\end{document}